\documentclass[10pt,a4paper]{article}
\usepackage[utf8]{inputenc}
\usepackage{fullpage}
\usepackage{graphicx}
\usepackage{fancyhdr}
\usepackage{comment}
\usepackage{occi}
\usepackage{lineno}   % adds line numbers, may be removed for non draft versions
\linenumbers          % adds line numbers, may be removed for non draft versions
\usepackage{verbatim} % adds verbatim options
\usepackage{tabularx} % adds extended tabular formatting options
\usepackage{listings}
\usepackage{color}
\definecolor{lightgray}{rgb}{.9,.9,.9}
\definecolor{darkgray}{rgb}{.4,.4,.4}
\definecolor{purple}{rgb}{0.65, 0.12, 0.82}

\lstdefinelanguage{json}{
  ndkeywords={String, Number, Boolean, Null, Object, Array},
  ndkeywordstyle=\itshape
}
\lstset{
   language=json,
   basicstyle=\footnotesize,
}

\setlength{\headheight}{13pt}
\pagestyle{fancy}

% default sans-serif
\renewcommand{\familydefault}{\sfdefault}

% no lines for headers and footers
\renewcommand{\headrulewidth}{0pt}
\renewcommand{\footrulewidth}{0pt}

% header
\fancyhf{}
\lhead{GWD-R}
\rhead{\today}

% footer
\lfoot{occi-wg@ogf.org}
\rfoot{\thepage}

% paragraphs need some space...
\setlength{\parindent}{0pt}
\setlength{\parskip}{1ex plus 0.5ex minus 0.2ex}

% some space between header and text...
\headsep 13pt

\setcounter{secnumdepth}{4}

\begin{document}

% header on first page is different
\thispagestyle{empty}

Draft \hfill Andy Edmons, Zhaw \\
OCCI-WG \hfill Thijs Metsch, Intel\\
\rightline {\today}

\vspace*{0.5in}

\begin{Large}
\textbf{Open Cloud Computing Interface - Text Rendering}
\end{Large}

\vspace*{0.5in}

\underline{Status of this Document}

This document provides information to the community regarding the
specification of the Open Cloud Computing Interface. Distribution is
unlimited.

\underline{Copyright Notice}

Copyright \copyright Open Grid Forum (2015). All Rights Reserved.

\underline{Trademarks}

OCCI is a trademark of the Open Grid Forum.

\underline{Abstract}

This document, part of a document series, produced by the OCCI working
group within the Open Grid Forum (OGF), provides a high-level
definition of a Protocol and API. The document is based upon
previously gathered requirements and focuses on the scope of important
capabilities required to support modern service offerings.

\newpage
\tableofcontents
\newpage

\section{Introduction}
The Open Cloud Computing Interface (OCCI) is a RESTful Protocol and
API for all kinds of Management tasks. OCCI was originally initiated
to create a remote management API for IaaS%
\footnote{Infrastructure as a Service}
model based Services, allowing for the development of interoperable tools for
common tasks including deployment, autonomic scaling and monitoring.
%
It has since evolved into an flexible API with a strong focus on
interoperability while still offering a high degree of extensibility. The
current release of the Open Cloud Computing Interface is suitable to serve many
other models in addition to IaaS, including e.g.~PaaS and SaaS.

In order to be modular and extensible the current OCCI specification is
released as a suite of complimentary documents which together form the complete
specification.
%
The documents are divided into three categories consisting of the OCCI Core,
the OCCI Renderings and the OCCI Extensions.
%
\begin{itemize}
\item The OCCI Core specification consist of a single document defining the
 OCCI Core Model. The OCCI Core Model can be interacted with {\em
 renderings} (including associated behaviours) and expanded through {\em extensions}.
\item The OCCI Rendering specifications consist of multiple documents each
 describing a particular rendering of the OCCI Core Model. Multiple renderings can
 interact with the same instance of the OCCI Core Model and will automatically support
 any additions to the model which follow the extension rules defined in OCCI
 Core.
\item The OCCI Extension specifications consist of multiple documents each
 describing a particular extension of the OCCI Core Model. The extension documents
 describe additions to the OCCI Core Model defined within the OCCI specification
 suite.
\end{itemize}
%
The current specification consist of three documents.
Future releases of OCCI may include additional rendering and extension
specifications. The documents of the current OCCI specification suite are:

\begin{description}
\item[OCCI Core] describes the formal definition of the the OCCI Core Model
\cite{occi:core}.
\item[OCCI HTTP Rendering] defines how to interact with the OCCI Core Model using the
RESTful OCCI API \cite{occi:http_rendering}. The document defines how the OCCI Core Model can
be communicated and thus serialised using the HTTP protocol.
\item[OCCI Infrastructure] contains the definition of the OCCI Infrastructure
extension for the IaaS domain \cite{occi:infrastructure}. The document defines
additional resource types, their attributes and the actions that can be taken
on each resource type.
\end{description}


\section{Notational Conventions}
All these parts and the information within are mandatory for
implementors (unless otherwise specified). The key words "MUST", "MUST
NOT", "REQUIRED", "SHALL", "SHALL NOT", "SHOULD", "SHOULD NOT",
"RECOMMENDED", "MAY", and "OPTIONAL" in this document are to be
interpreted as described in RFC 2119 \cite{rfc2119}.


\section{Security and Authentication}
OCCI does not require that an authentication mechanism be used nor
does it require that client to service communications are secured. It
does RECOMMEND that an authentication mechanism be used and that where
appropriate, communications are encrypted using HTTP over TLS. The
authentication mechanisms that MAY be used with OCCI are those that
can be used with HTTP and TLS. For further discussion see Section
\ref{sec:sec_consid}.

\section{ABNF Definitions}

For the following sections these notations will be used. Implementation MUST hence implement the renderings according to these definitions.

\subsection{Category CRLF}

The following syntax MUST be used for \hl{Category} renderings:

\begin{verbatim}
Category             = "Category" ":" #category-value
  category-value     = term
                      ";" "scheme" "=" <"> scheme <">
                      ";" "class" "=" ( class | <"> class <"> )
                      [ ";" "title" "=" quoted-string ]
                      [ ";" "rel" "=" <"> type-identifier <"> ]
                      [ ";" "location" "=" <"> URI <"> ]
                      [ ";" "attributes" "=" <"> attribute-list <"> ]
                      [ ";" "actions" "=" <"> action-list <"> ]
  term               = LOALPHA *( LOALPHA | DIGIT | "-" | "_" )
  scheme             = URI
  type-identifier    = scheme term
  class              = "action" | "mixin" | "kind"
  attribute-list     = attribute-def
                     | attribute-def *( 1*SP attribute-def)
  attribute-def      = attribute-name
                     | attribute-name
                       "{" attribute-property *( 1*SP attribute-property ) "}"
  attribute-property = "immutable" | "required"
  attribute-name     = attr-component *( "." attr-component )
  attr-component     = LOALPHA *( LOALPHA | DIGIT | "-" | "_" )
  action-list        = action
                     | action *( 1*SP action )
  action             = type-identifier
\end{verbatim}

\subsection{Link CRLF}

The following syntax MUST be used to represent OCCI \hl{Link} type
instance references:

\begin{verbatim}
Link               = "Link" ":" #link-value
  link-value       = "<" URI-Reference ">"
                    ";" "rel" "=" <"> resource-type <">
                    [ ";" "self" "=" <"> link-instance <"> ]
                    [ ";" "category" "=" link-type
                      *( ";" link-attribute ) ]
  term             = LOALPHA *( LOALPHA | DIGIT | "-" | "_" )
  scheme           = URI
  type-identifier  = scheme term
  resource-type    = type-identifier *( 1*SP type-identifier )
  link-type        = type-identifier *( 1*SP type-identifier )
  link-instance    = URI-reference
  link-attribute   = attribute-name "=" ( token | quoted-string )
  attribute-name   = attr-component *( "." attr-component )
  attr-component   = LOALPHA *( LOALPHA | DIGIT | "-" | "_" )
\end{verbatim}

The following syntax MUST be used to represent OCCI \hl{Action}
instance references:

\begin{verbatim}
ActionLink         = "Link" ":" #link-value
  link-value       = "<" action-uri ">"
                    ";" "rel" "=" <"> action-type <">
  term             =  LOALPHA *( LOALPHA | DIGIT | "-" | "_" )
  scheme           = relativeURI
  type-identifier  = scheme term
  action-type      = type-identifier
  action-uri       = URI "?" "action=" term
\end{verbatim}

\subsection{Attribute CRLF}

\begin{verbatim}
Attribute          = "X-OCCI-Attribute" ":" #attribute-repr
  attribute-repr   = attribute-name "=" ( string | number | bool | enum_val )
  attribute-name   = attr-component *( "." attr-component )
  attr-component   = LOALPHA *( LOALPHA | DIGIT | "-" | "_" )	
  string           = quoted-string
  number           = (int | float)
  int              = *DIGIT
  float            = *DIGIT "." *DIGIT
  bool             = ("true" | "false")
  enum_val         = string
\end{verbatim}

\subsection{Location CRLF}

\begin{verbatim}
Location      = "X-OCCI-Location" ":" location-value
  location-value  = URI-reference
\end{verbatim}

\section{Renderings}

\subsection{Entity Instance Rendering}

Entity instances MUST be rendered according to the following definitions.

\subsubsection{Resource Instance Rendering}

A \hl{Resource} instance MUST be rendered using the following definition:

\begin{verbatim}
	resource_rendering = 1*( Category CRLF )
    	                  *( Link CRLF )
    	                  *( ActionLink CRLF )
        	              *( Attribute CRLF )
\end{verbatim}

\paragraph{Action Invocation Rendering}

Upon an \hl{Action} invocation the client MUST send along the following definition:

\begin{verbatim}
	action_definition = 1( Category CRLF )
        	            *( Attribute CRLF )
\end{verbatim}

\subsubsection{Link Instance Rendering}

A \hl{Link} instance MUST be rendered using the following definition:

\begin{verbatim}
	link_rendering = 1*( Category CRLF )
    	              *( ActionLink CRLF )
        	          *( Attribute CRLF )
\end{verbatim}

% HERE I AM

\subsection{Category Instance Rendering}
\label{sec:format_category_instance_rendering}

ABNF

\subsubsection{Kind Instance Rendering}
\label{sec:format_kind}

 look above

\subsubsection{Mixin Instance Rendering}
\label{sec:format_mixin}

look above

\subsubsection{Action Instance Rendering}
\label{sec:format_action}

ABNF

\subsection{Entity Collection Rendering}

\begin{verbatim}
	resource_representations  = *( Location CRLF ) 
\end{verbatim}

\subsubsection{Resource Collection Rendering}

see above

\subsubsection{Link Collection Rendering}

see above

\subsection{Category Collection Rendering}

\begin{verbatim}
	resource\_representations  = *( Category CRLF ) 
\end{verbatim}

\subsubsection{Kind Collection Rendering}

see above

\subsubsection{Mixin Collection Rendering}

see above

\subsubsection{Action Collection Rendering}

see above

\subsection{Attributes Rendering}

\subsubsection{Entity Instance Attribute Rendering Specifics}

\subsubsection{Attribute Description Rendering}
\label{sec:format_attribute_description}

\section{OCCI Text plain rendering}
The OCCI Text plain rendering specifies a rendering of OCCI instance types in a simple text format.

The rendering can be used to render OCCI instances independently of the
protocol being used. Thus messages can be delivered by e.g. the HTTP
protocol as specified in \cite{occi:protocol}.

The following media-types MUST be used for the OCCI Text plain rendering:

	{\tt text/occi+plain}

and

	{\tt text/plain}

TODO: say stuff goes in the body. eithe rmultiple enties sep by "\\n" or by

\section{OCCI header rendering}
The OCCI Text plain rendering specifies a rendering of OCCI instance types in a simple text format.

The rendering can be used to render OCCI instances independently of the
protocol being used. Thus messages can be delivered by e.g. the HTTP
protocol as specified in \cite{occi:protocol}.

The following media-type MUST be used for the OCCI JSON Rendering:

{\tt text/occi}

All stuff goes in headers.

\section{URI listening}
The OCCI Text plain rendering specifies a rendering of OCCI instance types in a simple text format.

The rendering can be used to render OCCI instances independently of the
protocol being used. Thus messages can be delivered by e.g. the HTTP
protocol as specified in \cite{occi:protocol}.

The following media-types MUST be used for the OCCI JSON Rendering:

{\tt text/occi+plain}
{\tt text/plain}

\subsection{Entity Collection Rendering}

\section{Security Considerations}
\label{sec:sec_consid}
The OCCI HTTP rendering assumes HTTP or HTTP-related mechanisms for
security. As such, implementations SHOULD support
TLS \footnote{http://datatracker.ietf.org/wg/tls/} for transport layer
security.

Authentication SHOULD be realized by HTTP authentication mechanisms,
namely HTTP Basic or Digest Auth \cite{rfc2617}, with the former as
default. Additional profiles MAY specify other methods and should
ensure that the selected authentication scheme can be renderable over
the HTTP or HTTP-related protocols.

Authorization is not enforced on the protocol level, but SHOULD be
performed by the implementation. For the authorization decision, the
authentication information as provided by the mechanisms described
above MUST be used.

Protection against potential Denial-of-Service scenarios are out of
scope of this document; the OCCI HTTP Rendering specifications assumes
cooperative clients that SHOULD use selection and filtering as
provided by the Category mechanism wherever possible. Additional
profiles to this document, however, MAY specifically address such
scenarios; in that case, best practices from the HTTP ecosystem and
appropriate mechanisms as part of the HTTP protocol specification
SHOULD be preferred.

As long as specific extensions of the OCCI Core and Model
specification do not impose additional security requirements than the
OCCI Core and Model specification itself, the security considerations
documented above apply to all (existing and future)
extensions. Otherwise, an additional profile to this specification
MUST be provided; this profile MUST express all additional security
considerations using HTTP mechanisms.

\section{Glossary}
\label{sec:glossary}
\begin{tabular}{l|p{11cm}}
Term & Description \\
\hline
\hl{Action} & An OCCI base type. Represent an invocable operation on a \hl{Entity} sub-type instance or collection thereof. \\
\hl{Category} & A type in the OCCI model. The parent type of \hl{Kind}. \\
Client & An OCCI client.\\
Collection & A set of \hl{Entity} sub-type instances all associated to a particular \hl{Kind} instance. \\
\hl{Entity} & An OCCI base type. The parent type of \hl{Resource} and \hl{Link}. \\
\hl{Kind} & A type in the OCCI model. The central piece in the OCCI classification system. \\
\hl{Link} & An OCCI base type. A \hl{Link} instance associate one \hl{Resource} instance with another. \\
Mix-in & A non-structural \hl{Kind}. The ``mix-in like'' concept in OCCI only
  support binding of new attributes and \hl{Action}s at run-time. A
  non-structural \hl{Kind} can only be associated with an {\em instance} of an
  \hl{Entity} sub-type. \\
Non-structural \hl{Kind} & An instance of \hl{Kind} {\em not} used as an unique identifier of an OCCI base type. \\
OCCI & Open Cloud Computing Interface \\
OCCI base type & One of \hl{Entity}, \hl{Resource}, \hl{Link} or \hl{Action}. \\
OGF & Open Grid Forum \\
\hl{Resource} & An OCCI base type. The parent type for all domain-specific resource types. \\
Structural \hl{Kind} & An instance of \hl{Kind} assigned as the unique identifier of an OCCI base type. \\
Tag & A non-structural \hl{Kind} with no attributes or actions defined. \\
URI & Uniform Resource Identifier \\
URL & Uniform Resource Locator \\
URN & Uniform Resource Name \\
\end{tabular}


\section{Contributors}

We would like to thank the following people who contributed to this
document:

\begin{tabular}{l|p{2in}|p{2in}}
Name & Affiliation & Contact \\
\hline
Michael Behrens & R2AD & behrens.cloud at r2ad.com \\
Mark Carlson & Oracle & mark.carlson at oracle.com \\
Andy Edmonds & Intel - SLA@SOI project & andy at edmonds.be \\
Sam Johnston & Google & samj at samj.net \\
Gary Mazzaferro & OCCI Counselour - AlloyCloud, Inc. &  garymazzaferro at gmail.com \\ 
Thijs Metsch & Platform Computing, Sun Microsystems & tmetsch at platform.com \\
Ralf Nyrén & Aurenav & ralf at nyren.net \\
Alexander Papaspyrou & TU Dortmund University & alexander.papaspyrou at tu\-dortmund.de \\
Alexis Richardson & RabbitMQ & alexis at rabbitmq.com \\
Shlomo Swidler & Orchestratus & shlomo.swidler at orchestratus.com \\
Florian Feldhaus & GWDG & florian.feldhaus at gwdg.de \\
Jean Parpaillon & & jean.parpaillon at free.fr \\
\end{tabular}

Next to these individual contributions we value the contributions from
the OCCI working group.


\section{Intellectual Property Statement}
The OGF takes no position regarding the validity or scope of any
intellectual property or other rights that might be claimed to pertain
to the implementation or use of the technology described in this
document or the extent to which any license under such rights might or
might not be available; neither does it represent that it has made any
effort to identify any such rights. Copies of claims of rights made
available for publication and any assurances of licenses to be made
available, or the result of an attempt made to obtain a general
license or permission for the use of such proprietary rights by
implementers or users of this specification can be obtained from the
OGF Secretariat.

The OGF invites any interested party to bring to its attention any
copyrights, patents or patent applications, or other proprietary
rights which may cover technology that may be required to practice
this recommendation. Please address the information to the OGF
Executive Director.


\section{Disclaimer}
\input{include/disclaimer}

\section{Full Copyright Notice}
Copyright \copyright ~Open Grid Forum (2009-2016). All Rights Reserved.

This document and translations of it may be copied and furnished to
others, and derivative works that comment on or otherwise explain it
or assist in its implementation may be prepared, copied, published and
distributed, in whole or in part, without restriction of any kind,
provided that the above copyright notice and this paragraph are
included on all such copies and derivative works. However, this
document itself may not be modified in any way, such as by removing
the copyright notice or references to the OGF or other organizations,
except as needed for the purpose of developing Grid Recommendations in
which case the procedures for copyrights defined in the OGF Document
process must be followed, or as required to translate it into
languages other than English.

The limited permissions granted above are perpetual and will not be
revoked by the OGF or its successors or assignees.


\bibliographystyle{IEEEtran}
\bibliography{references}

\end{document}