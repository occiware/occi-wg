\documentclass[10pt,a4paper]{article}
\usepackage[utf8]{inputenc}
\usepackage{fullpage}
\usepackage{graphicx}
\usepackage{fancyhdr}
\usepackage{occi}
\setlength{\headheight}{13pt}
\pagestyle{fancy}

% default sans-serif
\renewcommand{\familydefault}{\sfdefault}

% no lines for headers and footers
\renewcommand{\headrulewidth}{0pt}
\renewcommand{\footrulewidth}{0pt}

% header
\fancyhf{}
\lhead{GWD-R}
\rhead{\today}

% footer
\lfoot{occi-wg@ogf.org}
\rfoot{\thepage}

% paragraphs need some space...
\setlength{\parindent}{0pt}
\setlength{\parskip}{1ex plus 0.5ex minus 0.2ex}

% some space between header and text...
\headsep 13pt

\setcounter{secnumdepth}{4}

\begin{document}

% header on first page is different
\thispagestyle{empty}

GWD-R \hfill Thijs Metsch, Platform Computing\\
OCCI-WG \hfill Andy Edmonds, Intel\\
\rightline {October 7, 2010}\\
\rightline {Updated: \today}

\vspace*{0.5in}

\begin{Large}
\textbf{Open Cloud Computing Interface - RESTful HTTP Rendering}
\end{Large}

\vspace*{0.5in}

\underline{Status of this Document}

This document provides information to the community regarding the
specification of the Open Cloud Computing Interface. Distribution is
unlimited.

\underline{Copyright Notice}

Copyright \copyright Open Grid Forum (2009-2011). All Rights Reserved.

\underline{Trademarks}

OCCI is a trademark of the Open Grid Forum.

\underline{Abstract}

This document, part of a document series, produced by the OCCI working
group within the Open Grid Forum (OGF), provides a high-level
definition of a Protocol and API. The document is based upon
previously gathered requirements and focuses on the scope of important
capabilities required to support modern service offerings.

\newpage
\tableofcontents
\newpage

\section{Introduction}
The Open Cloud Computing Interface (OCCI) is a RESTful Protocol and
API for all kinds of Management tasks. OCCI was originally initiated
to create a remote management API for IaaS%
\footnote{Infrastructure as a Service}
model based Services, allowing for the development of interoperable tools for
common tasks including deployment, autonomic scaling and monitoring.
%
It has since evolved into an flexible API with a strong focus on
interoperability while still offering a high degree of extensibility. The
current release of the Open Cloud Computing Interface is suitable to serve many
other models in addition to IaaS, including e.g.~PaaS and SaaS.

In order to be modular and extensible the current OCCI specification is
released as a suite of complimentary documents which together form the complete
specification.
%
The documents are divided into three categories consisting of the OCCI Core,
the OCCI Renderings and the OCCI Extensions.
%
\begin{itemize}
\item The OCCI Core specification consist of a single document defining the
 OCCI Core Model. The OCCI Core Model can be interacted with {\em
 renderings} (including associated behaviours) and expanded through {\em extensions}.
\item The OCCI Rendering specifications consist of multiple documents each
 describing a particular rendering of the OCCI Core Model. Multiple renderings can
 interact with the same instance of the OCCI Core Model and will automatically support
 any additions to the model which follow the extension rules defined in OCCI
 Core.
\item The OCCI Extension specifications consist of multiple documents each
 describing a particular extension of the OCCI Core Model. The extension documents
 describe additions to the OCCI Core Model defined within the OCCI specification
 suite.
\end{itemize}
%
The current specification consist of three documents.
Future releases of OCCI may include additional rendering and extension
specifications. The documents of the current OCCI specification suite are:

\begin{description}
\item[OCCI Core] describes the formal definition of the the OCCI Core Model
\cite{occi:core}.
\item[OCCI HTTP Rendering] defines how to interact with the OCCI Core Model using the
RESTful OCCI API \cite{occi:http_rendering}. The document defines how the OCCI Core Model can
be communicated and thus serialised using the HTTP protocol.
\item[OCCI Infrastructure] contains the definition of the OCCI Infrastructure
extension for the IaaS domain \cite{occi:infrastructure}. The document defines
additional resource types, their attributes and the actions that can be taken
on each resource type.
\end{description}


\section{Notational Conventions}
All these parts and the information within are mandatory for
implementors (unless otherwise specified). The key words "MUST", "MUST
NOT", "REQUIRED", "SHALL", "SHALL NOT", "SHOULD", "SHOULD NOT",
"RECOMMENDED", "MAY", and "OPTIONAL" in this document are to be
interpreted as described in RFC 2119 \cite{rfc2119}.


This document uses the Augmented Backus-Naur Form (ABNF) notation of
RFC 2616 \cite{rfc2616}, and explicitly includes the following rules
from it: quoted-string, token, SP (space), LOALPHA, DIGIT.

All examples in this document use one of the following three HTTP
category definitions. An example name-space hierarchy is also
given. Syntax and Semantics is explained in the remaining sections of
the document. These examples do not strive to be complete but to show
the features OCCI has:

\begin{verbatim}

Category: compute;
          scheme="http://schemas.ogf.org/occi/infrastructure#";
          class="kind";
          location=/compute/
    (This is an compute kind)

Category: storage;
          scheme="http://schemas.ogf.org/occi/infrastructure#";
          class="kind";
          location=/storage/
    (This is an storage kind)

Category: my_stuff;
          scheme="http://example.com/occi/my_stuff#";
          class="mixin";
          location=/my_stuff/ 
    (This is a mixin of user1)

The following namespace hierarchy is used in the examples:

http://example.com/-/
http://example.com/vms/user1/vm1
http://example.com/vms/user1/vm2
http://example.com/vms/user2/vm1
http://example.com/disks/user1/disk1
http://example.com/disks/user2/disk1
http://example.com/compute/
http://example.com/storage/
http://example.com/my_stuff/
\end{verbatim}

The following terms \cite{rfc3986} are used when referring to URI
components:

\begin{verbatim}
 http://example.com:8080/over/there?action=stop#xyz
 \__/   \______________/\_________/ \_________/ \_/
  |           |            |            |        |
scheme     authority       path        query   fragment
\end{verbatim}

All examples in this document use the \emph{text/plain} HTTP
Content-Type for posting information. To retrieve information the HTTP
Accept header \emph{text/plain} is used.

This specification is aligned with RFC 3986 \cite{rfc3986}.

\section{OCCI HTTP Rendering}
The OCCI HTTP Rendering document specifies how the OCCI Core Model
\cite{occi:core}, including extensions thereof, is rendered over the
HTTP protocol \cite{rfc2616}. The document describes the general
behavior for all interaction with an OCCI implementation over HTTP
together with three content types to represent the data being
transferred. The content types specified are \textit{text/plain},
\textit{text/occi} and \textit{text/uri-list}. Other data formats such
as e.g.~OVF and JSON will be specified in complimentary documents.

\subsection{A RESTful HTTP Rendering for OCCI}
The OCCI HTTP Rendering uses all the features the HTTP and underlying
protocols offer but also builds upon the Resource Oriented
Architecture (ROA). ROA's use Representation State Transfer (REST)
\cite{REST_Fielding} to cater for client and service
interactions. Interaction with the system is by inspection and
modification of a set of related resources and their states, be it on
the complete state or a sub-set. Resources MUST be uniquely
identified. HTTP is an ideal protocol to use in ROA systems as it
provides the means to uniquely identify individual resources through
URLs as well as operating upon them with a set of general-purpose
methods known as HTTP verbs. These HTTP verbs map loosely to the
resource related operations of Create (POST), Retrieve (GET), Update
(POST/PUT) and Delete (DELETE).

Each resource instance within an OCCI system MUST be uniquely
identified by an URI stored in the \emph{id} attribute of the
\hl{Entity} type \cite{occi:core}.

The structure of these URIs is opaque and the system should not assume
a static, pre-determined scheme for their structure. For example
\emph{Entity::id} can be \emph{http://example.com/vms/user1/vm1}.

\subsection{Behavior of the HTTP Verbs}
As OCCI adopts a ROA, REST-based architecture and uses HTTP as the
foundation protocol the means of interaction with all RESTful resource
instances is through the four main HTTP verbs. OCCI service
implementations MUST, at a minimum, support these verbs as shown in
table \ref{tbl:http_verbs}.

\mytablefloat{\label{tbl:http_verbs}HTTP Verb Behavior}{
\begin{tabular}{p{1in} p{0.7in} p{0.7in} p{0.7in} p{0.7in} p{0.7in} p{0.7in}}
\toprule
Type & GET & POST (create) & POST (action) & PUT (create) & PUT (update) & DELETE \\
\colrule
resource instance & 
Rendering of this resource instance & 
Create a new resource instance & 
Trigger action & 
Create an resource instance at the given path & 
Update an resource instance at an given path & 
Delete this resource instance \\
\colrule
Path in the name-space hierarchy ending with / &	
Listing of all resource instances below this name-space & 
Create a new resource instance & 
N/A & 
N/A & 
N/A & 
Delete all the resource instances below this name-space hierarchy \\
\colrule
Location of an \hl{Mixin} or \hl{Kind} & 
Listing containing locations to all resource instances belonging to this \hl{Mixin} or \hl{Kind} & 
N/A & 
Trigger action (defined for this kind or mixin) on all resource instances belonging to this \hl{Mixin} or \hl{Kind} & 
Add an resource instance to a \hl{Mixin} & 
N/A & 
Remove an resource instance given in the request from a \hl{Mixin} \\
\colrule
Query Interface	& 
Listing of all registered \hl{Kind}s and \hl{Mixin}s & 
N/A & 
N/A & 
Add a user defined \hl{Mixin} & 
N/A & 
Remove a user-defined \hl{Mixin} (defined in the request) \\
\botrule
\end{tabular}
}

\subsection{A RESTful Rendering of OCCI}
The following sections and paragraphs describe how the OCCI model MUST
be implemented by OCCI implementations. Operations which are not
defined are out of scope for this specification and MAY be implemented
. This is the minimal set to ensure interoperability.

\subsubsection{Resource Instance Name-space Hierarchy and Location}
The name-space and the hierarchy are free definable by the Service
Provider. The OCCI implementation needs to implement the location path
feature, which is required by OCCI for discovering capabilities and
operations on \hl{Mixin}s and \hl{Kind}s. Location paths tell the
client where all resource instance of one \hl{Kind} or \hl{Mixin} (in
case the \hl{Mixin} is used as a tag) can be found regardless of the
hierarchy the service provider defines. Location paths are defined
through a HTTP Category rendering and MUST be present for all HTTP
Categories for which resource instance can be created. The location
paths MUST end with a \emph{'/'}. These paths are discoverable by the
client through the Query interface \ref{sec:query}.

\subsection{Various Operations and their Prerequisites and Behaviors}

\subsubsection{Handling the Query Interface}
\label{sec:query}
The query interface MUST be implemented by all OCCI
implementations. It MUST be found at the path \emph{/-/} off the root
of the OCCI implementation. With the help of the query interface it is
possible for the client to determine the capabilites of the OCCI
implementation he refers to. The following operations, listed below,
MUST be implemented by the service.

\begin{description}
\item[Retrieval of all registered \hl{Kind}s and \hl{Mixin}s] The HTTP
  verb GET must be used to retrieve all \hl{Kind}s and \hl{Mixin}s
  the service can handle. This allows the client to discover the
  capabilities of the OCCI implementation. The result MUST contain all
  information about the \hl{Kind}s and \hl{Mixin}s (including
  Attributes and \hl{Actions} assigned).
\begin{verbatim}
> GET /-/ HTTP/1.1
> [...]
 
< HTTP/1.1 200 OK
< [...]
< Category: compute;
<           scheme="http://schemas.ogf.org/occi/infrastructure#";
<           class="kind";
<           rel=http://schemas.ogf.org/occi/core#resource;
<           attributes="occi.compute.cores occi.compute.state{immutable} ..."
<           actions="http://schemas.ogf.org/occi/infrastructure/compute/action#stop ...";
<           location=/compute/;
< Category: my_stuff;
<           scheme="http://example.com/occi/my_stuff#";
<           class="mixin";
<           location=/my_stuff/;
< Category: storage;
<           scheme="http://schemas.ogf.org/occi/infrastructure#";
<           class="kind";
<           rel=http://schemas.ogf.org/occi/core#resource;
<           attributes="...";
<           actions="...";
<           location=/storage/;
 
\end{verbatim}
An OCCI implementation MUST support a filtering mechanism. If a HTTP
Category is provided in the request the server MUST only return the
complete rendering of the requested \hl{Kind} or \hl{Mixin}.

\item[Adding a \hl{Mixin} definition] To add a \hl{Mixin} to the
  service the HTTP PUT verb MUST be used. All possible information for
  the \hl{Mixin} must be defined. At least the HTTP Category term,
  scheme and location MUST be defined. Actions and Attributes are not
  supported:
\begin{verbatim}
> PUT /-/ HTTP/1.1
> [...]
> Category: my_stuff;
>           scheme="http://example.com/occi/my_stuff#";
>           class="mixin";
>           rel=http:/example.com/occi/something_else#mixin;
>           location=/my_stuff/;

< HTTP/1.1 200 OK
< [...]
\end{verbatim}
The service might reject this request if it does not allow
user-defined \hl{Mixin}s to be created. Also on name collisions of the
defined location path the service provider MAY reject this
operation.

\item[Removing a \hl{Mixin} definition] A user defined \hl{Mixin} MAY
  be removed (if allowed) by using the HTTP DELETE verb. The
  information about which \hl{Mixin} should be deleted MUST be
  provided in the request:
\begin{verbatim}
> DELETE /-/ HTTP/1.1
> [...]
> Category: my_stuff; scheme="http://example.com/occi/my_stuff#"; class="mixin";

< HTTP/1.1 200 OK
< [...]
\end{verbatim}
\end{description}

\subsubsection{Operation on Paths in the Name-space}
The following operations are defined when operating on paths in the
name-space hierarchy which are not location paths nor resource
instances. They MUST end with \emph{/} (For example
\emph{http://example.com/vms/user1/}).

\begin{description}
\item[Retrieving All resource instances Below a Path] The HTTP verb
  GET must be used to retrieve all resource instances. The service
  provider MUST return a Listing containing all resource instances
  which are children of the provided URI in the name-space hierarchy:
\begin{verbatim}
> GET /vms/user1/ HTTP/1.1
> [...]
 
< HTTP/1.1 200 OK
< [...]
< 
< X-OCCI-Location: http://example.com/vms/user1/vm1
< X-OCCI-Location: http://example.com/vms/user1/vm2
\end{verbatim}
An OCCI implementations MUST support a filtering mechanism. If a
Category is provided in the request the server MUST only return the
resource instances belonging to the provided \hl{Mixin} or \hl{Kind}.

If an OCCI Entity attribute (X-OCCI-Attribute) is provided in the request the
server MUST only return the resource instances which have a matching attribute
value.

\item[Deletion of all resource instances below a path] \textbf{(Note:
  this is a potentially dangerous operation!)} The HTTP verb DELETE
  must be used to delete all resource instances under a hierarchy:
\begin{verbatim}
> DELETE /vms/user1/ HTTP/1.1
> [...]
 
< HTTP/1.1 200 OK
< [...]
\end{verbatim}
\end{description}

\subsubsection{Operations on \hl{Mixin}s or \hl{Kind}s}
All of the following operations MUST only be be performed on location
paths provided by \hl{Kind}s and \hl{Mixin}s. The path MUST end with
an \emph{/}.

\begin{description}
\item[Retrieving All Resource Instances Belonging to \hl{Mixin} or
  \hl{Kind}] The HTTP verb GET must be used to retrieve all resource
  instances. The service provider MUST return a listing containing all
  resource instances which belong to the requested \hl{Mixin} or
  \hl{Kind}:
\begin{verbatim}
> GET /compute/ HTTP/1.1
> [...]
 
< HTTP/1.1 200 OK
< [...]
< 
< X-OCCI-Location: http://example.com/vms/user1/vm1
< X-OCCI-Location: http://example.com/vms/user1/vm2
< X-OCCI-Location: http://example.com/vms/user2/vm1
\end{verbatim}
An OCCI implementation MUST support a filtering mechanism. If a HTTP
Category is provided in the request the server MUST only return the
resource instances belonging to the provided \hl{Kind} or
\hl{Mixin}. The provided HTTP category definition SHOULD be different
from the \hl{Kind} or \hl{Mixin} definition which defined the location
path used in the request.

If an OCCI Entity attribute (X-OCCI-Attribute) is provided in the request the
server MUST only return the resource instances which have a matching attribute
value.

\item[Triggering Actions on All Instances of a \hl{Mixin} or
  \hl{Kind}] Actions can be triggered on all resource instances of the
  same \hl{Mixin} or \hl{Kind}. The HTTP POST verb MUST be used and
  the request MUST contain the \hl{Category} defining the \hl{Action}.
  Additionally
  the \hl{Action} MUST be defined by the \hl{Kind} or \hl{Mixin} which
  defines the location path which is used in the request:
\begin{verbatim}
> POST /compute/?action=stop HTTP/1.1
> [...]
> Category: stop; scheme="[...]"; class="action";
> X-OCCI-Attribute: method=poweroff

< HTTP/1.1 200 OK
< [...]
\end{verbatim}

\item[Associate resource instances with a \hl{Mixin}] One or multiple
  resource instances can be associated with a \hl{Mixin} using the
  HTTP PUT verb. The URIs which uniquely defined the resource instance
  MUST be provided in the request:
\begin{verbatim}
> PUT /my_stuff/ HTTP/1.1
> [...]
> X-OCCI-Location: http://example.com/vms/user1/vm1
> X-OCCI-Location: http://example.com/vms/user1/vm2
> X-OCCI-Location: http://example.com/disks/user1/disk1

< HTTP/1.1 200 OK
< [...]
\end{verbatim}

\item[Unassociated resource instance(s) from a \hl{Mixin}] One or
  multiple resource instances can be removed from a \hl{Mixin} using
  the HTTP DELETE verb. The URIs which uniquely defined the resource
  instance MUST be provided in the request:
\begin{verbatim}
> DELETE /my_stuff/ HTTP/1.1
> [...]
> X-OCCI-Location: http://example.com/vms/user1/vm1
> X-OCCI-Location: http://example.com/vms/user1/vm2
> X-OCCI-Location: http://example.com/disks/user1/disk1

< HTTP/1.1 200 OK
< [...]
\end{verbatim}
\end{description}

\subsubsection{Operations on Resource Instances}
\label{sec:ops_on_instances}
The following operations MUST be implemented by the OCCI
implementation for operations on resource instances. The resource
instance is uniquely identified by an URI (For example:
\emph{http://example.com/vms/user1/vm1}).\footnote{The path MUST NOT
  end with an '/' - that would mean that a client operates on a path
  in the name-space hierarchy}

\begin{description}
\item[Creating a resource instance] A request to create a resource
  instance MUST contain one and only one HTTP category rendering which
  refers to a specific \hl{Kind} instance. This \hl{Kind}
  instance MUST be used for defining the type of the resource instance.
  A request MAY also contain one or more HTTP category renderings which refer
  to different \hl{Mixin} instances. Any such \hl{Mixin} instances MUST
  be associated (if allowed) to the resource instance.
  Optional information which might be provided by the client
  and if available MUST be used are HTTP Links and HTTP
  X-OCCI-Attributes (mapping to \hl{Link} and the attributes of an
  resource instance). Two ways can be used to create a new resource
  instance - HTTP POST or PUT:
\begin{verbatim}
> POST /compute/ HTTP/1.1
> [...]
> 
> Category: compute; scheme="http://schemas.ogf.org/occi/infrastructure#"; class="kind"; 
> X-OCCI-Attribute: occi.compute.cores=2
> X-OCCI-Attribute: occi.compute.hostname=foobar
> [...]
 
< HTTP/1.1 200 OK
< [...]
< Location: http://example.com/vms/user1/vm1
\end{verbatim}
  The path on which this POST verb is executed MUST be any existing
  path in the hierarchy of the Service provider's name-space. It MUST
  be the location path of the corresponding \hl{Kind}. The OCCI
  implementation MUST return the Location of the newly created
  resource instance.

  HTTP PUT can also be used to create a resource instance. In this
  case the client ask the service provider to create a resource
  instance at a certain path in the name-space hierarchy.\footnote{If
    a Service Provider does not want the user to define the path of a
    resource instance it can return a Bad Request return code - See
    section \ref{sec:return_codes}. Service Providers MUST ensure that
    the paths of REST resources stays unique in their name-space.}
\begin{verbatim}
> PUT /vms/user1/my_first_virtual_machine HTTP/1.1
> [...]
> 
> Category: compute; scheme="http://schemas.ogf.org/occi/infrastructure#"; "class=kind"; 
> X-OCCI-Attribute: occi.compute.cores=2
> X-OCCI-Attribute: occi.compute.hostname=foobar
> [...]
 
< HTTP/1.1 200 OK
< [...]
\end{verbatim}
  The OCCI implementation will return an OK code.
  
  While creating a resource instance the resource instance is added to
  the collection defined by the \hl{Kind}.

\item[Retrieving a resource instance] For retrieval the HTTP GET verb
  is used. It MUST return at least the HTTP category which defines the
  \hl{Kind} of the resource instance. HTTP Links pointing to related
  resource instances, other URI or Actions MUST be included if
  present. Only Actions currently applicable SHOULD be rendered using
  HTTP Links. The Attributes of the resource instance MUST be exposed
  to the client if available.
\begin{verbatim}
> GET /vms/user1/vm1 HTTP/1.1
> [...]
 
< HTTP/1.1 200 OK
< [...]
< Category: compute; scheme="http://schemas.ogf.org/occi/infrastructure#"; class="kind";
< Category: my_stuff; scheme="http://example.com/occi/my_stuff#"; class="mixin";
< X-OCCI-Attribute: occi.compute.cores=2
< X-OCCI-Attribute: occi.compute.hostname=foobar
< Link: [...]
\end{verbatim}

\item[Updating a resource instance] Before updating a resource
  instance it is RECOMMENDED that the client first retrieves the
  resource instance. Updating is done using the HTTP PUT verb. Only
  the information (HTTP Links, HTTP X-OCCI-Attributes or HTTP
  categories), which are updated MUST be provided along with the
  request.\footnote{Changing the type of the resource instance MUST
    not be possible.}
\begin{verbatim}
> PUT /vms/user1/vm1 HTTP/1.1
> [...]
> 
> X-OCCI-Attribute: occi.compute.memory=4.0
> [...]
 
< HTTP/1.1 200 OK
< [...]
\end{verbatim}

\item[Deleting a resource instance] A resource instance can be deleted
  using the HTTP DELETE verb. No other information SHOULD be added to
  the request.\footnote{If the resource instances is a \hl{Link} type
    the source and target must be updated accordingly}
\begin{verbatim}
> DELETE /vms/user1/vm1 HTTP/1.1
> [...]

< HTTP/1.1 200 OK
< [...]
\end{verbatim}

\item[Triggering an Action on a resource instance] To trigger an
  action on a resource instance the request MUST containing the HTTP
  Category defining the \hl{Action}. It MAY include HTTP
  X-OCCI-Attributes which are the parameters of the action. Actions
  are triggered using the HTTP POST verb and by adding a query to the
  URI. This query exposes the term of the \hl{Action}. If an action is
  not available a Bad Request should be returned.
\begin{verbatim}
> POST /vms/user1/vm1?action=stop HTTP/1.1
> [...]
> Category: stop; scheme="[...]"; class="action";
> X-OCCI-Attribute: method=poweroff

< HTTP/1.1 200 OK
< [...]
\end{verbatim}
\end{description}

\subsubsection{Handling \hl{Link}s resource instances}
In general resource instance of the type \hl{Link} and \hl{Resource}
are handled in the same way. Some special handling in the creation and
handling of resource instance of the type \hl{Link} are described in
this section. They MUST be implemented by an OCCI implementation.

\begin{description}
\item[Creation of a \hl{Link} during creation of a \hl{Resource}
  instance] When creating a resource instance of the type
  \hl{Resource}, and \hl{Link}s are defined those \hl{Link}s MUST be
  created implicitly (Resulting in the creation of multiple REST
  resources. Still only the Location of the REST resource which
  represent the requested \hl{Kind} MUST be returned - The URIs of the
  \hl{Link}s can be discovered by retrieving a rendering of the
  resource instance). To render all the \hl{Entity} attributes of the
  \hl{Link} those must be specified in the HTTP Link description
  during creation.
\begin{verbatim}
> POST /compute/ HTTP/1.1
> [...]
> 
> Category: compute; 
            scheme="http://schemas.ogf.org/occi/infrastructure#"; 
            class="kind"; 
> Link: </network/123>;
        rel="http://schemas.ogf.org/occi/infrastructure#network";
        category="http://schemas.ogf.org/occi/infrastructure#networkinterface";
        occi.networkinterface.interface="eth0";
        occi.networkinterface.mac="00:11:22:33:44:55";
> X-OCCI-Attribute: occi.compute.cores=2
> X-OCCI-Attribute: occi.compute.hostname=foobar
> [...]
 
< HTTP/1.1 200 OK
< [...]
< Location: http://example.com/vms/user1/vm1
\end{verbatim}

\item[Retrieval resource instances of the type \hl{Resource} with
  defined \hl{Link}s] When an resource instance of the type
  \hl{Resource} is rendered it MUST expose all the \hl{Link}s which
  are associated with the resource instance. Since \hl{Link}s are
  directed only those originating SHOULD be listed.
\begin{verbatim}
> GET /vms/user1/vm1 HTTP/1.1
> [...]
 
< HTTP/1.1 200 OK
< [...]
< Category: compute; scheme="http://schemas.ogf.org/occi/infrastructure#"; class="kind";
< Category: my_stuff; scheme="http://example.com/occi/my_stuff#"; class="mixin";
< X-OCCI-Attribute: occi.compute.cores=2
< X-OCCI-Attribute: occi.compute.hostname=foobar
< Link: </network/123>;
        rel="http://schemas.ogf.org/occi/infrastructure#network";
        self="/link/networkinterface/456";
        category="http://schemas.ogf.org/occi/infrastructure#networkinterface";
        occi.networkinterface.interface="eth0";
        occi.networkinterface.mac="00:11:22:33:44:55";
        occi.networkinterface.state="active";
\end{verbatim}

\item[Creation of \hl{Link} resource instances] To directly create a
  \hl{Link} between two resource instance the \hl{Kind} as well as a
  source and target attribute MUST be provide during creation of the
  resource instance (Which can be done with HTTP PUT or HTTP POST -
  See section \ref{sec:ops_on_instances} for a complete
  specification).
\begin{verbatim}
> POST /compute/ HTTP/1.1
> [...]
> 
> Category: networkinterface; 
            scheme="http://schemas.ogf.org/occi/infrastructure#"; 
            class="kind"; 
> X-OCCI-Attribute: source=http://example.com/vms/user1/vm1
> X-OCCI-Attribute: target=http://example.com/network/123
> [...]
 
< HTTP/1.1 200 OK
< [...]
< Location: http://example.com/link/networkinterface/456
\end{verbatim}

\item[Retrieval of \hl{Link} resource instances] Retrieval of a
  \hl{Link} is analogue to the retrieval of any other resource
  instance. Please review section \ref{sec:ops_on_instances} for more
  details.
\begin{verbatim}
> GET /link/networkinterface/456 HTTP/1.1
> [...]

< HTTP/1.1 200 OK
< [...]
< Category: networkinterface; scheme="http://schemas.ogf.org/occi/infrastructure#"; class="kind";
< X-OCCI-Attribute: occi.networkinterface.interface="eth0";
< X-OCCI-Attribute: occi.networkinterface.mac="00:11:22:33:44:55";
< X-OCCI-Attribute: occi.networkinterface.state="active";
< X-OCCI-Attribute: source=/vms/user1/vm1
< X-OCCI-Attribute: target=/network/123
\end{verbatim}
\end{description}

\subsection{Syntax and Semantics of the Rendering}
\label{sec:syntax}
All data transferred using the \textit{text/occi} and
\textit{text/plain} content types is encoded with HTTP \cite{rfc2616}
compliant headers. Four specific HTTP headers are used:
\begin{itemize}
\item Category
\item Link
\item X-OCCI-Attribute
\item X-OCCI-Location
\end{itemize}
The \textit{text/occi} content type renders these headers as true HTTP
headers in the header portion of a HTTP request or response. The
\textit{text/plain} content type renders the same headers, with
identical syntax, in the body of the HTTP request/response. See
section \ref{sec:content_type} for more information on the use of
different content types.

Multiple HTTP header field values MUST be supported as defined by
RFC~2616 \cite{rfc2616}. This applies to both the \textit{text/occi}
and \textit{text/plain} content types. RFC 2616 defines two different
methods to render multiple header field values, either a
comma-separated list or multiple header lines. The following two
rendering examples are identical and both formats MUST be supported by
both OCCI client and server to be compliant.

Comma-separated rendering of multiple HTTP header field values:
\begin{verbatim}
X-OCCI-Attribute: occi.compute.memory=2.0, occi.compute.speed=2.33
X-OCCI-Location: /compute/123, /compute/456
\end{verbatim}

Separate header lines for each HTTP header field value:
\begin{verbatim}
X-OCCI-Attribute: occi.compute.memory=2.0
X-OCCI-Attribute: occi.compute.speed=2.33
X-OCCI-Location: /compute/123
X-OCCI-Location: /compute/456
\end{verbatim}

\subsubsection{Rendering of the OCCI Category, Kind and Mixin types}
Instances of the \hl{Category}, \hl{Kind} and \hl{Mixin} types
\cite{occi:core} MUST be rendered using the Category header as defined
by the Web Categories specification%
\footnote{http://tools.ietf.org/html/draft-johnston-http-category-header-01}.

The following syntax applies:

\begin{verbatim}
Category             = "Category" ":" #category-value
  category-value     = term
                      ";" "scheme" "=" <"> scheme <">
                      ";" "class" "=" ( class | <"> class <"> )
                      [ ";" "title" "=" quoted-string ]
                      [ ";" "rel" "=" <"> type-identifier <"> ]
                      [ ";" "location" "=" URI ]
                      [ ";" "attributes" "=" <"> attribute-list <"> ]
                      [ ";" "actions" "=" <"> action-list <"> ]
  term               = LOALPHA *( LOALPHA | DIGIT | "-" | "_" )
  scheme             = URI
  type-identifier    = scheme term
  class              = "action" | "mixin" | "kind"
  attribute-list     = attribute-def
                     | attribute-def *( 1*SP attribute-def)
  attribute-def      = attribute-name
                     | attribute-name
                       "{" attribute-property *( 1*SP attribute-property ) "}"
  attribute-property = "immutable" | "required"
  attribute-name     = attr-component *( "." attr-component )
  attr-component     = LOALPHA *( LOALPHA | DIGIT | "-" | "_" )
  action-list        = action
                     | action *( 1*SP action )
  action             = type-identifier
\end{verbatim}

The following example illustrates a rendering of the \hl{Kind}
instance assigned to the \hl{Storage} type \cite{occi:infrastructure}:

\begin{verbatim}
Category: storage;
    scheme="http://schemas.ogf.org/occi/infrastructure#";
    class="kind";
    title="Storage Resource";
    rel="http://schemas.ogf.org/occi/core#resource";
    location=/storage/;
    attributes="occi.storage.size occi.storage.state{immutable}";
    actions="http://schemas.ogf.org/occi/infrastructure/storage/action#resize ...";
\end{verbatim}

\subsubsection{Rendering of OCCI Link instance references}
The rendering of a \hl{Resource} instance \cite{occi:core} MUST
represent any associated \hl{Link} instances using the HTTP Link
header specified in the Web Linking RFC 5988 \cite{rfc5988}.  For
example, rendering of a \hl{Compute} instance linked to a \hl{Storage}
instance MUST include a Link header displaying the OCCI \hl{Link}
instance of the relation.

The following syntax MUST be used to represent OCCI \hl{Link} type
instance references:

\begin{verbatim}
Link               = "Link" ":" #link-value
  link-value       = "<" URI-Reference ">"
                    ";" "rel" "=" <"> resource-type <">
                    [ ";" "self" "=" <"> link-instance <"> ]
                    [ ";" "category" "=" link-type
                      *( ";" link-attribute ) ]
  term             = LOALPHA *( LOALPHA | DIGIT | "-" | "_" )
  scheme           = URI
  type-identifier  = scheme term
  resource-type    = type-identifier *( 1*SP type-identifier )
  link-type        = type-identifier *( 1*SP type-identifier )
  link-instance    = URI-reference
  link-attribute   = attribute-name "=" ( token | quoted-string )
  attribute-name   = attr-component *( "." attr-component )
  attr-component   = LOALPHA *( LOALPHA | DIGIT | "-" | "_" )
\end{verbatim}

The following example illustrates the rendering of a
\hl{NetworkInterface} \cite{occi:infrastructure} instance linking to a
\hl{Network} resource instance:

\begin{verbatim}
Link: </network/123>;
    rel="http://schemas.ogf.org/occi/infrastructure#network";
    self="/link/networkinterface/456";
    category="http://schemas.ogf.org/occi/infrastructure#networkinterface";
    occi.networkinterface.interface="eth0";
    occi.networkinterface.mac="00:11:22:33:44:55";
    occi.networkinterface.state="active";
\end{verbatim}

\subsubsection{Rendering of references to OCCI Action instances}
The rendering of a \hl{Resource} instance \cite{occi:core} MUST
represent any associated \hl{Action} instances using the HTTP Link
header specified in the Web Linking RFC 5988 \cite{rfc5988}.  For
example, rendering of a \hl{Compute} instance MUST include a Link
header displaying any \hl{Action}s currently applicable to the
resource instance.

The following syntax MUST be used to represent OCCI \hl{Action}
instance references:

\begin{verbatim}
Link               = "Link" ":" #link-value
  link-value       = "<" action-uri ">"
                    ";" "rel" "=" <"> action-type <">
  term             =  LOALPHA *( LOALPHA | DIGIT | "-" | "_" )
  scheme           = relativeURI
  type-identifier  = scheme term
  action-type      = type-identifier
  action-uri       = URI "?" "action=" term
\end{verbatim}

The following example illustrates the rendering of a reference to the
``start'' \hl{Action} defined for the \hl{Compute} type
\cite{occi:infrastructure}. Such a reference would be present in the
rendering of a \hl{Compute} instance.

\begin{verbatim}
Link: </compute/123?action=start>;
    rel="http://schemas.ogf.org/occi/infrastructure/compute/action#start"
\end{verbatim}

\subsubsection{Rendering of OCCI Entity attributes}
Attributes defined for OCCI \hl{Entity} sub-types \cite{occi:core},
i.e.~\hl{Resource} and \hl{Link}, MUST be rendered using the
X-OCCI-Attribute HTTP header. For example the rendering of a
\hl{Compute} instance MUST render the associated attributes, such as
e.g. \texttt{occi.compute.memory}, using X-OCCI-Attribute headers.

The X-OCCI-Attribute header uses a simple key-value format where each
HTTP header field value represent a single attribute. The field value
consist of an attribute name followed by the equal sign (``='') and an
attribute value. The attribute value MUST be quoted if it contains a
separator character as specified in RFC 2616 (page 16) \cite{rfc2616}.

The following syntax MUST be used to represent OCCI \hl{Entity}
attributes:

\begin{verbatim}
Attribute          = "X-OCCI-Attribute" ":" #attribute-repr
  attribute-repr   = attribute-name "=" ( string | number | bool | enum_val )
  attribute-name   = attr-component *( "." attr-component )
  attr-component   = LOALPHA *( LOALPHA | DIGIT | "-" | "_" )	
  string           = quoted-string
  number           = (int | float)
  int              = *DIGIT
  float            = *DIGIT "." *DIGIT
  bool             = ("true" | "false")
  enum_val         = string
\end{verbatim}

Attribute names for the infrastructure types are defined in the OCCI
Infrastructure document \cite{occi:infrastructure}.  The rules for
defining new attribute names can be found in the ``Extensibility''
section of the OCCI Core document \cite{occi:core}.

The following example illustrates a rendering of the attributes
defined by \hl{Compute} type \cite{occi:infrastructure}:

\begin{verbatim}
X-OCCI-Attribute: occi.compute.architechture="x86_64"
X-OCCI-Attribute: occi.compute.cores=2
X-OCCI-Attribute: occi.compute.hostname="testserver"
X-OCCI-Attribute: occi.compute.speed=2.66
X-OCCI-Attribute: occi.compute.memory=3.0
X-OCCI-Attribute: occi.compute.state=active
\end{verbatim}

\subsubsection{Rendering of Location-URIs}
In order to render an OCCI representation solely in the HTTP header,
i.e.~using the \textit{text/occi} content type, the X-OCCI-Location
HTTP header MUST be used to return a list of resource instance
URIs. Each header field value correspond to a single URI. Multiple
resource instance URIs are returned using multiple X-OCCI-Location
headers.

\begin{verbatim}
Location      = "X-OCCI-Location" ":" location-value
  location-value  = URI-reference
\end{verbatim}

The following example illustrates the rendering of a list of
\hl{Compute} resource instances:
\begin{verbatim}
X-OCCI-Location: http://example.com/compute/123
X-OCCI-Location: http://example.com/compute/456
X-OCCI-Location: http://example.com/compute/789
\end{verbatim}

\subsection{General HTTP Behaviors Adopted by OCCI}
The following sections deal with some general HTTP features which are
adopted by OCCI.

\subsubsection{Security and Authentication}
OCCI does not require that an authentication mechanism be used nor
does it require that client to service communications are secured. It
does recommend that an authentication mechanism be used and that where
appropriate, communications are encrypted using HTTP over TLS. The
authentication mechanisms that MAY be used with OCCI are those that
can be used with HTTP and TLS.

\subsubsection{Additional headers (Caching Headers)}
The responses from an OCCI implementation MAY include additional
headers like those for Caching purposes like E-Tags.

\subsubsection{Asynchronous operations}
OCCI implementations MAY implement a way to deal with asynchronous
calls. Upon long-running operations the OCCI implementation MAY return
a temporary resource (e.g. a task resource) using the HTTP 'Location'
header and a corresponding HTTP 202 return code. Clients can query
that resource until the operation finishes. Upon completion of the
operation this temporary result will redirect to the resulting REST
resource using the HTTP 'Location' header and return the HTTP 301
return code signalling the completion.

\subsubsection{Versioning}
Information about what version of OCCI is supported by a OCCI
implementation MUST be advertised to a client on each response to a
client. The version field in the response MUST include the value
OCCI/X.Y, where X is the major version number and Y is the minor
version number of the implemented OCCI specification. In the case of a
HTTP Header Rendering, the server response MUST relay versioning
information using the HTTP header name 'Server'.

\begin{verbatim}
HTTP/1.1 200 OK
Server: occi-server/1.1 (linux) OCCI/1.1
[...]
\end{verbatim}

Complimenting the service-side behavior of an OCCI implementation, a
client SHOULD indicate to the OCCI service implementation the version
it expects to interact with. For the clients, the information SHOULD
be advertised in all requests it issues. A client request SHOULD relay
versioning information in the 'User-Agent' header. The 'User-Agent'
field MUST include the same value (OCCI/X.Y) as supported by the
Server HTTP header.

\begin{verbatim}
GET <Path> HTTP/1.1
Host: example.com
User-Agent: occi-client/1.1 (linux) libcurl/7.19.4 OCCI/1.1
[...]
\end{verbatim}

If a OCCI implementation receives a request from a client that
supplies a version number higher than the service supports, the
service MUST respond back to the client with an exception indicating
that the requested version is not implemented. Where a client
implements OCCI using a HTTP transport, the HTTP code 501, not
implemented, MUST be used.

OCCI implementations which implement this version of the Document MUST
use the version String \emph{OCCI/1.1}.

\subsubsection{Content-type and Accept headers}
\label{sec:content_type}
A server MUST react according to the Accept header the client
provides. If none is given - or \textit{*/*} is used - the service
MUST use the Content-type \emph{text/plain}. This is the fall-back
rendering and MUST be implemented. Otherwise the according rendering
MUST be used. Each Rendering SHOULD expose which Accept and
Content-type header fields it can handle. Overall the service MUST
support the \textit{text/occi} and \textit{text/plain} Content-types.

The server MUST also return the proper Content-type header. If a
client provides information with a Content-Type - the information MUST
be parsed accordingly.

When the Client request a Content-Type that will result in an
incomplete or faulty rendering the Service MUST return the unsupported
media type , 415, HTTP code.

The following examples demonstrate the behavior of an HTTP GET
operations on the resource instance \emph{} using two different HTTP
Accept headers:

\begin{verbatim}
> GET /vms/user1/vm1 HTTP/1.1
> Accept: text/plain
> [...]
 
< HTTP/1.1 200 OK
< [...]
< Category: compute; scheme="http://schemas.ogf.org/occi/infrastructure#"; class="kind";
< Category: my_stuff; scheme="http://example.com/occi/my_stuff#"; class="mixin"; 
< X-OCCI-Attribute: occi.compute.cores=2
< X-OCCI-Attribute: occi.compute.hostname=foobar
< Link: [...]
\end{verbatim}

And with \emph{text/occi} as HTTP Accept header:

\begin{verbatim}
> GET /vms/user1/vm1 HTTP/1.1
> Accept: text/occi
> [...]
 
< HTTP/1.1 200 OK
< Category: compute; scheme="http://schemas.ogf.org/occi/infrastructure#"; class="kind";,
            my_stuff; scheme="http://example.com/occi/my_stuff#"; class="mixin";
< X-OCCI-Attribute: occi.compute.cores=2, occi.compute.hostname=foobar
< Link: [...]
< [...]
OK
\end{verbatim}

\paragraph{The Content-type text/plain}
While using this rendering with the Content-Type \textit{text/plain}
the information described in section \ref{sec:syntax} MUST be placed
in the HTTP Body.

Each rendering of an OCCI base type will be placed in the body. Each
entry consists of a name followed by a colon (":") and the field
value. The format of the field value is specified separately for each
of the three header fields, see section \ref{sec:syntax}.

\paragraph{The Content-type text/occi}
While using this rendering with the Content-Type \textit{text/occi}
the information described in section \ref{sec:syntax} MUST be placed
in the HTTP Header. The body MUST contain the string 'OK' on
successful operations.

The HTTP header fields MUST follow the specification in RFC 2616
\cite{rfc2616}. A header field consists of a name followed by a colon
(":") and the field value. The format of the field value is specified
separately for each of the header fields, see section
\ref{sec:syntax}.

\textbf{Limitations: } HTTP header fields MAY appear multiple times in
a HTTP request or response. In order to be OCCI compliant the
specification of multiple message-header fields according to RFC 2616
MUST be fully supported. In essence there are two valid representation
of multiple HTTP header field values. A header field might either
appear several times or as a single header field with a
comma-separated list of field values. Due to implementation issues in
many web frameworks and client libraries it is RECOMMENDED to use the
comma-separated list format for best interoperability.

HTTP header field values which contain separator characters MUST be
properly quoted according to RFC 2616.

Space in the HTTP header section of a HTTP request is a limited
resource. By this, it is noted that many HTTP servers limit the number
of bytes that can be placed in the HTTP Header area. Implementers MUST
be aware of this limitation in their own implementation and take
appropriate measures so that truncation of header data does NOT occur.

\paragraph{The Content-type text/uri-list}
This Rendering can handle the \textit{text/uri-list} Accept Header. It
will use the Content-type \textit{text/uri-list}.

This rendering cannot render resource instances or \hl{Kind}s or
\hl{Mixin}s directly but just links to them. For concrete rendering of
Kinds and Categories the Content-types \textit{text/occi},
\textit{text/plain} MUST be used. If a request is done with the
\textit{text/uri-list} in the Accept header, while not requesting for
a Listing a Bad Request MUST be returned. Otherwise a list of
resources must be rendered in \emph{text/uri-list} format, which can
be used for listing resource in collections or the namespace of the
OCCI implementation.

\subsubsection{Return codes}
\label{sec:return_codes}
At any point the service provider MAY return any of the following HTTP
Return Codes:

\mytablefloat{\label{tbl:http_codes}HTTP Return Codes}{
\begin{tabular}{l|l|p{3in}}
\toprule
Code & Description & Notes \\
\colrule
200 & OK & \\
202 & Accepted & Used for asynchronous non-blocking calls. \\
400 & Bad Request & For example on parsing errors or missing information \\
401 & Unauthorized & \\
403 & Forbidden & \\
405 & Method Not Allowed & \\
409 & Conflict & \\
410 & Gone & \\
415 & Unsupported Media Type & \\
500 & Internal Server Error & \\
501 & Not Implemented & \\
503 & Service Unavailable & \\
\botrule
\end{tabular}}

\subsection{More complete examples}
Since most examples are not complete due to space limitations this
section will give some more complete examples.

\subsubsection{Creating a compute resource instance}
\begin{verbatim}
> POST / HTTP/1.1
> User-Agent: curl/7.21.0 (x86_64-pc-linux-gnu) libcurl/7.21.0 OpenSSL/0.9.8o zlib/1.2.3.4 libidn/1.18
> Host: localhost:8080
> Accept: */*
> Cookie: pyocci_user=Zm9v|1291753962|7011a4821179ff98ea96d4b44fade0512b1ffc52
> Content-Type: text/occi
> Category: compute; scheme="http://schemas.ogf.org/occi/infrastructure#"; class="kind";
> 
< HTTP/1.1 200 OK
< Content-Length: 2
< Content-Type: text/html; charset=UTF-8
< Location: /users/foo/compute/b9ff813e-fee5-4a9d-b839-673f39746096
< Server: pyocci OCCI/1.1
< 
* Connection #0 to host localhost left intact
* Closing connection #0
OK% 
\end{verbatim}

\subsubsection{Retrieving a compute resource instance}
\begin{verbatim}
> GET /users/foo/compute/b9ff813e-fee5-4a9d-b839-673f39746096 HTTP/1.1
> User-Agent: curl/7.21.0 (x86_64-pc-linux-gnu) libcurl/7.21.0 OpenSSL/0.9.8o zlib/1.2.3.4 libidn/1.18
> Host: localhost:8080
> Accept: */*
> Cookie: pyocci_user=Zm9v|1291753962|7011a4821179ff98ea96d4b44fade0512b1ffc52
> 
< HTTP/1.1 200 OK
< Content-Length: 510
< Etag: "ef485dc7066745cb0fe1e31ecdd4895c356b5bd5"
< Content-Type: text/plain
< Server: pyocci OCCI/1.1
< 
Category: compute;
    scheme="http://schemas.ogf.org/occi/infrastructure#"
    class="kind";
Link: </users/foo/compute/b9ff813e-fee5-4a9d-b839-673f39746096?action=start>;
    rel="http://schemas.ogf.org/occi/infrastructure/compute/action#start"
X-OCCI-Attribute: occi.compute.architecture=x86
X-OCCI-Attribute: occi.compute.state=inactive
X-OCCI-Attribute: occi.compute.speed=1.33
X-OCCI-Attribute: occi.compute.memory=2.0
X-OCCI-Attribute: occi.compute.cores=2
X-OCCI-Attribute: occi.compute.hostname=dummy
\end{verbatim}

\section{Contributors}

We would like to thank the following people who contributed to this
document:

\begin{tabular}{l|p{2in}|p{2in}}
Name & Affiliation & Contact \\
\hline
Michael Behrens & R2AD & behrens.cloud at r2ad.com \\
Mark Carlson & Oracle & mark.carlson at oracle.com \\
Andy Edmonds & Intel - SLA@SOI project & andy at edmonds.be \\
Sam Johnston & Google & samj at samj.net \\
Gary Mazzaferro & OCCI Counselour - AlloyCloud, Inc. &  garymazzaferro at gmail.com \\ 
Thijs Metsch & Platform Computing, Sun Microsystems & tmetsch at platform.com \\
Ralf Nyrén & Aurenav & ralf at nyren.net \\
Alexander Papaspyrou & TU Dortmund University & alexander.papaspyrou at tu\-dortmund.de \\
Alexis Richardson & RabbitMQ & alexis at rabbitmq.com \\
Shlomo Swidler & Orchestratus & shlomo.swidler at orchestratus.com \\
Florian Feldhaus & GWDG & florian.feldhaus at gwdg.de \\
Jean Parpaillon & & jean.parpaillon at free.fr \\
\end{tabular}

Next to these individual contributions we value the contributions from
the OCCI working group.


\section{Glossary}
\label{sec:glossary}
\begin{tabular}{l|p{11cm}}
Term & Description \\
\hline
\hl{Action} & An OCCI base type. Represent an invocable operation on a \hl{Entity} sub-type instance or collection thereof. \\
\hl{Category} & A type in the OCCI model. The parent type of \hl{Kind}. \\
Client & An OCCI client.\\
Collection & A set of \hl{Entity} sub-type instances all associated to a particular \hl{Kind} instance. \\
\hl{Entity} & An OCCI base type. The parent type of \hl{Resource} and \hl{Link}. \\
\hl{Kind} & A type in the OCCI model. The central piece in the OCCI classification system. \\
\hl{Link} & An OCCI base type. A \hl{Link} instance associate one \hl{Resource} instance with another. \\
Mix-in & A non-structural \hl{Kind}. The ``mix-in like'' concept in OCCI only
  support binding of new attributes and \hl{Action}s at run-time. A
  non-structural \hl{Kind} can only be associated with an {\em instance} of an
  \hl{Entity} sub-type. \\
Non-structural \hl{Kind} & An instance of \hl{Kind} {\em not} used as an unique identifier of an OCCI base type. \\
OCCI & Open Cloud Computing Interface \\
OCCI base type & One of \hl{Entity}, \hl{Resource}, \hl{Link} or \hl{Action}. \\
OGF & Open Grid Forum \\
\hl{Resource} & An OCCI base type. The parent type for all domain-specific resource types. \\
Structural \hl{Kind} & An instance of \hl{Kind} assigned as the unique identifier of an OCCI base type. \\
Tag & A non-structural \hl{Kind} with no attributes or actions defined. \\
URI & Uniform Resource Identifier \\
URL & Uniform Resource Locator \\
URN & Uniform Resource Name \\
\end{tabular}


\section{Intellectual Property Statement}
The OGF takes no position regarding the validity or scope of any
intellectual property or other rights that might be claimed to pertain
to the implementation or use of the technology described in this
document or the extent to which any license under such rights might or
might not be available; neither does it represent that it has made any
effort to identify any such rights. Copies of claims of rights made
available for publication and any assurances of licenses to be made
available, or the result of an attempt made to obtain a general
license or permission for the use of such proprietary rights by
implementers or users of this specification can be obtained from the
OGF Secretariat.

The OGF invites any interested party to bring to its attention any
copyrights, patents or patent applications, or other proprietary
rights which may cover technology that may be required to practice
this recommendation. Please address the information to the OGF
Executive Director.


\section{Disclaimer}
\input{include/disclaimer}

\section{Full Copyright Notice}
Copyright \copyright ~Open Grid Forum (2009-2016). All Rights Reserved.

This document and translations of it may be copied and furnished to
others, and derivative works that comment on or otherwise explain it
or assist in its implementation may be prepared, copied, published and
distributed, in whole or in part, without restriction of any kind,
provided that the above copyright notice and this paragraph are
included on all such copies and derivative works. However, this
document itself may not be modified in any way, such as by removing
the copyright notice or references to the OGF or other organizations,
except as needed for the purpose of developing Grid Recommendations in
which case the procedures for copyrights defined in the OGF Document
process must be followed, or as required to translate it into
languages other than English.

The limited permissions granted above are perpetual and will not be
revoked by the OGF or its successors or assignees.


\bibliographystyle{IEEEtran}
\bibliography{references}

\end{document}
