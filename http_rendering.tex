\documentclass[10pt,a4paper]{article}
\usepackage{fullpage}
\usepackage{graphicx}
\usepackage{fancyhdr}
\usepackage{occi}
\setlength{\headheight}{13pt}
\pagestyle{fancy}

% default sans-serif
\renewcommand{\familydefault}{\sfdefault}

% no lines for headers and footers
\renewcommand{\headrulewidth}{0pt}
\renewcommand{\footrulewidth}{0pt}

% header
\fancyhf{}
\lhead{GWD-R}
\rhead{\today}

% footer
\lfoot{occi-wg@ogf.org}
\rfoot{\thepage}

% paragraphs need some space...
\setlength{\parindent}{0pt}
\setlength{\parskip}{1ex plus 0.5ex minus 0.2ex}

% some space between header and text...
\headsep 13pt

\setcounter{secnumdepth}{4}

\begin{document}

% header on first page is different
\thispagestyle{empty}

GWD-R \hfill Thijs Metsch, Platform Computing\\
OCCI-WG \hfill Andy Edmonds, Intel\\
\rightline {October 7, 2010}\\
\rightline {Updated: \today}

\vspace*{0.5in}

\begin{Large}
\textbf{Open Cloud Computing Interface - HTTP Rendering}
\end{Large}

\vspace*{0.5in}

\underline{Status of this Document}

This document provides information to the community regarding the
specification of the Open Cloud Computing Interface. Distribution is
unlimited.

\underline{Copyright Notice}

Copyright \copyright Open Grid Forum (2009-2010). All Rights Reserved.

\underline{Trademarks}

OCCI is a trademark of the Open Grid Forum.

\underline{Abstract}

This document, part of a document series, produced by the OCCI working
group within the Open Grid Forum (OGF), provides a high-level
definition of a Protocol and API. The document is based upon
previously gathered requirements and focuses on the scope of important
capabilities required to support modern service offerings.

\newpage
\tableofcontents
\newpage

\section{Introduction}
The Open Cloud Computing Interface (OCCI) is a RESTful Protocol and
API for all kinds of Management tasks. OCCI was originally initiated
to create a remote management API for IaaS%
\footnote{Infrastructure as a Service}
model based Services, allowing for the development of interoperable tools for
common tasks including deployment, autonomic scaling and monitoring.
%
It has since evolved into an flexible API with a strong focus on
interoperability while still offering a high degree of extensibility. The
current release of the Open Cloud Computing Interface is suitable to serve many
other models in addition to IaaS, including e.g.~PaaS and SaaS.

In order to be modular and extensible the current OCCI specification is
released as a suite of complimentary documents which together form the complete
specification.
%
The documents are divided into three categories consisting of the OCCI Core,
the OCCI Renderings and the OCCI Extensions.
%
\begin{itemize}
\item The OCCI Core specification consist of a single document defining the
 OCCI Core Model. The OCCI Core Model can be interacted with {\em
 renderings} (including associated behaviours) and expanded through {\em extensions}.
\item The OCCI Rendering specifications consist of multiple documents each
 describing a particular rendering of the OCCI Core Model. Multiple renderings can
 interact with the same instance of the OCCI Core Model and will automatically support
 any additions to the model which follow the extension rules defined in OCCI
 Core.
\item The OCCI Extension specifications consist of multiple documents each
 describing a particular extension of the OCCI Core Model. The extension documents
 describe additions to the OCCI Core Model defined within the OCCI specification
 suite.
\end{itemize}
%
The current specification consist of three documents.
Future releases of OCCI may include additional rendering and extension
specifications. The documents of the current OCCI specification suite are:

\begin{description}
\item[OCCI Core] describes the formal definition of the the OCCI Core Model
\cite{occi:core}.
\item[OCCI HTTP Rendering] defines how to interact with the OCCI Core Model using the
RESTful OCCI API \cite{occi:http_rendering}. The document defines how the OCCI Core Model can
be communicated and thus serialised using the HTTP protocol.
\item[OCCI Infrastructure] contains the definition of the OCCI Infrastructure
extension for the IaaS domain \cite{occi:infrastructure}. The document defines
additional resource types, their attributes and the actions that can be taken
on each resource type.
\end{description}


\section{Notational Conventions}
All these parts and the information within are mandatory for
implementors (unless otherwise specified). The key words "MUST", "MUST
NOT", "REQUIRED", "SHALL", "SHALL NOT", "SHOULD", "SHOULD NOT",
"RECOMMENDED", "MAY", and "OPTIONAL" in this document are to be
interpreted as described in RFC 2119 \cite{rfc2119}.


UML activity diagrams do not specify how OCCI should be rendered but
what possible request and outcomes can be.

\section{HTTP Rendering}
The HTTP Protocol is the underlying core fabric of OCCI and uses all
the features of the HTTP and underlying protocols (like self-healing
capabilities of TCP) offer. OCCI also builds upon the Resource
Oriented Architecture (ROA). ROA's use Representation State Transfer
(REST) to cater for client and service interactions. Interaction with
the system is by inspection and modification of a set of related
resources and their states, be it on the complete state or a
sub-set. Resources must be uniquely identified. HTTP is an ideal
protocol to use in ROA systems as it provides the means to uniquely
identify individual resources through URLs as well as operating upon
them with a set of general-purpose HTTP verbs. These HTTP verbs map
loosely to the resource related operations of Create (POST), Retrieve
(GET), Update (POST/PUT) and Delete (DELETE).

The following notations are used when referring to parts or complete
URIs:

\begin{verbatim}
http://www.example.com:8080/foo/bar;action=stop
<   >  <     Authority    >< Path >< Fragment >
  ^
  Scheme
\end{verbatim}

The following section describe the general behavior for all HTTP based
renderings. Later sections will describe the syntax and semantic of
how to render the OCCI Core model with different Content-Types.

\subsection{Identification of Kinds}
Each Kind sub-type instance within an OCCI system must be uniquely
identified by an URI. The structure of these URIs is opaque and the
system should not assume a static, pre-determined scheme for their
structure.

\subsection{Interaction with Kinds}
As OCCI adopts a ROA, REST-based architecture and uses HTTP as the
foundation protocol the means of interaction with all RESTful resource
instances is through the four main HTTP verbs. OCCI service
implementations must, at a minimum, support these verbs:

\begin{tabular}{l|p{3.2in}|p{2in}}
Verb & RFC Definition & Usage in OCCI context \\
\hline
POST & The POST method is used to request that the origin server accept the resource enclosed in the request as a new subordinate of the resource identified by the Request-URI in the Request- Line & This is the verb will be used when creating Kinds. POST can also be used to create a sub-resource of an existing resource type. Action are also triggered using the POST verb. \\
GET & The GET method means retrieve whatever information (in the form of a resource) is identified by the Request-URI & Retrieving one Kind or Category using GET will result in a representation of the resource. Lists of Kinds and Categories are also retrieved through GET (Discovery/Querying). \\
PUT & The PUT method requests that the enclosed resource be stored under the supplied Request-URI & The PUT verb will be used when updating a Kind or when creating an Kind at a specific location\footnote{In difference to POST where sub-resources will be created}. \\
DELETE & The DELETE method requests that the origin server delete the resource identified by the Request-URI & This verb is used to destroy instances of Kinds. \\ 
\end{tabular}

\subsection{Security and Authentication}
OCCI does not require that an authentication mechanism be used nor
does it require that client to service communications are secured. It
does recommend that an authentication mechanism be used and that where
appropriate, communications are encrypted using HTTP over TLS. The
authentication mechanisms that CAN be used with OCCI are those that
can be used with HTTP and TLS, for example Basic [REF], Digest [REF]
and OAuth [REF]. If an OCCI service requires authentication the
response to a request that MUST be authenticated must be a HTTP 401
code that indicates the request is authorized. In response to
authenticate the client MUST set a WWW-Authenticate header field and
through this indicate the authentication mechanism.

\subsection{Versioning}
Information about what version of OCCI is supported by a provider MUST
be advertised to a client on each response to a client. The version
field in the response MUST include the value OCCI/X.Y, where X is the
major version number and Y is the minor version number of the
implemented OCCI specification. In the case of a HTTP Header
Rendering, the server response should relay versioning information
using the HTTP header name 'Server'.

\begin{verbatim}
HTTP/1.1 202 Accepted
Server: occi-server/1.1 (linux) OCCI/1.0
[...]
\end{verbatim}

Complimenting the service-side behavior of an OCCI implementation, a
client MUST indicate to the OCCI service implementation the version it
expects to interact with. For the clients, the information MUST be
advertised in the request it issues. In the case of a HTTP Header
Rendering, the client request should relay versioning information in
the 'User-Agent' header. The 'User-Agent' field must include the same
value (OCCI/X.Y) as specified for the Server HTTP header.

\begin{verbatim}
GET <UDN> HTTP/1.1
Host: example.com
User-Agent: occi-client/1.1 (linux) libcurl/7.19.4 OCCI/1.0
[...]
\end{verbatim}

If a server receives a request from a client that supplies a version
number higher than the service supports, the service MUST respond back
to the client with an exception indicating that the requested version
is not implemented. Where a client implements OCCI using a HTTP
transport, the HTTP code 501, not implemented should be used.

\subsection{Content-type and Accept headers}
A server MUST react according to the Accept header the client
provides. If non is given - or \textit{*/*} is used - the service MUST
use the HTTP Body Rendering described in section
\ref{sec:http_body}. This is the fall-back rendering and MUST be
implemented. Otherwise the according rendering MUST be used. Each
Rendering SHOULD expose which Accept and Content-type header fields it
can handle. Overall the server MUST support the \textit{text/occi},
\textit{text/plain} and \textit{text/uri-list} Content-types.

The server MUST also return the proper Content-type header. If a
client POSTs information with a Content-Type - the information MUST be
parsed accordingly.

When the Client request a Content-Type will will result in an
incomplete or faulty rendering the Service MUST return the 406 Not
Acceptable HTTP code.

\subsection{Discovery and Listing}
When performing an GET operation an server MUST reply with a listing
of matching Kinds. If the request was preformed on a Path of an URI
only the sub-resource SHOULD be returned. If a Category was given in
the request the server MUST only give back Kinds which have this
Category assigned.

If the request was performed on the \textit{/-/} Path a list of
Categories which the server can handle MUST be returned. Categories
used for tagging are not exported via this mechanism.

Each category MUST have an unique location assigned which COULD be the
term of the category. This path MUST NOT be empty. At a minimum it
could be \textit{'/'} It reflects the path under which REST-resources
which relate to this category can be found. Adding and removing
REST-Resources from a category is done by operating on this
path\footnote{The path MUST end with /}.

\subsection{Behavior of HTTP operations}

More information on how OCCI compatible Service should handle the HTTP
verbs is given in these sections. Possible results are demonstrated
here. How to render the results is described later on.

\subsubsection{Return Codes}
The following return codes CAN be returned by an Service implementing
OCCI.

\subsection{Rendering the OCCI model}
\label{sec:model_rendering}
The following subsection demonstrate how the OCCI base types can be
syntactically rendered.

\marginpar{User proper ABNF grammar to describe the header formats.}

\subsubsection{Rendering of an OCCI-Category}

The semantics of the Category in the OCCI context is described in the
OCCI Core \& Models document.\footnote{This rendering follows the
  Category header defined by the Web Categories specification,
  http://tools.ietf.org/html/draft-johnston-http-category-header-01
  and MUST be rendered accordingly.}

\marginpar{Define ``Categary identifier'' somewhere,
  i.e. concatenation of scheme + term.}

\begin{verbatim}
Category: <term>; scheme="<scheme>"
    [;rel=<space-separated list of related Category identifiers>]
    [;attributes=<space-seperated list of attribute names>]
    [;title=<Title of this Category>]    
    ;location=<Parent location>
\end{verbatim}

There is NO order for the optional part.

\subsubsection{Rendering of OCCI-Links and OCCI-Actions}
The semantics of the Link header in the OCCI context is described in
the OCCI Core \& Models document.\footnote{This rendering follows the
  Link header defined by the Web Linking specification,
  http://tools.ietf.org/html/draft-nottingham-http-link-header-10 and
  MUST be rendered accordingly.}

\begin{verbatim}
Link: <Resource URL>;
    rel=<space-separated list of Category identifiers of the target Resource type>
    [;self=<Link instance URL>]
    [;category=<space-separated list of Category identifiers of the Link type>
    [;<attribute name>=<attribute value>] ... ]

or in case it is an Action:

Link: <Resource URL> + ";action=" + <Term of the Action>;
    rel=<Category identifier of the Action>
\end{verbatim}

\subsubsection{Rendering of OCCI-Attributes}
The X-OCCI-Attribute MUST be used to render the attributes associated
with a OCCI Kind. A simple key-value format is used. The field value
consist of an attribute name followed by an equal sign ("=") and the
attribute value. The attribute value must be quoted if it includes a
separator character, see RFC 2616 (page 16).

\begin{verbatim}
X-OCCI-Attribute: <attribute name>=<attribute value>
\end{verbatim}

Valid attribute names for OCCI Kinds are specified in appropriate
Extension documents.

\subsubsection{Rendering of Location-URLs}
To render an OCCI representation solely in the header, the
X-OCCI-Location header MUST be used to return a list of Kind
URLs. Each header field value correspond to a single URL. Multiple
Kind URLs are returned using multiple X-OCCI-Location headers. See RFC
2616 for information on how to render multiple HTTP headers.

\begin{verbatim}
X-OCCI-Location: <URL>
\end{verbatim}

\subsubsection{Fields}
The following setups show how the Core Model MUST be rendered. Shown
are the fields which MUST be available in a request from the Client or
a response from the Server.

\begin{tabular}{l|l|l|l}
Operation & Required HTTP-Header(s) & Optional HTTP-Header(s) & Notes \\
\hline
Rendering of a Category & Category & N/A & \\
Rendering a list of Categories & Category & N/A & \\
Rendering a list of Kinds & X-OCCI-Location & N/A & \\
Rendering of a Resource & Category & X-OCCI-Attribute, Link & \\
Rendering of an Action & Category, Link & X-OCCI-Attribute & \\
Rendering of a Link & Category, Link & X-OCCI-Attribute & \\
\end{tabular}

\subsubsection{Rendering the OCCI model in the HTTP Body}
\label{sec:http_body}
While using this rendering with the Content-Type \textit{text/plain}
the information described in section \ref{sec:model_rendering} MUST be
placed in the HTTP Body.

Each Rendering of an OCCI base type will be placed in the body. Each
entry consists of a name followed by a colon (":") and the field
value. The format of the field value is specified separately for each
of the three header fields, see section \ref{sec:model_rendering}.

\subsubsection{Rendering the OCCI model in the HTTP Header}
While using this rendering with the Content-Type \textit{text/occi}
the information described in section \ref{sec:model_rendering} MUST be
placed in the HTTP Header. The body MUST contain the string 'OK' on
successfull operations.

The HTTP header fields MUST follow the specification in RFC 2616
\cite{rfc2616}. A header field consists of a name followed by a colon
(":") and the field value. The format of the field value is specified
separately for each of the three header fields, see section
\ref{sec:model_rendering}.

\paragraph{Limitations}
HTTP header fields MAY appear multiple times in a HTTP request or
response. In order to be OCCI compliant the specification of multiple
message-header fields according to RFC 2616 MUST be fully
supported. In essence there are two valid representation of multiple
HTTP header field values. A header field might either appear several
times or as a single header field with a comma-separated list of field
values. Due to implementation issues in many web frameworks and client
libraries it is RECOMMENDED to use the comma-separated list format for
best interoperability.

HTTP header field values which contain separator characters MUST be
properly quoted according to RFC 2616.

Space in the HTTP header section of a HTTP request is a limited
resource. By this, it is noted that many HTTP servers limit the number
of bytes that can be placed in the HTTP Header area. Implementers MUST
be aware of this limitation in their own implementation and take
appropriate measures so that truncation of header data does NOT
occure.

\subsubsection{Rendering the OCCI model via URL Listings}

\textbf{This Rendering can handle the \textit{text/uri-list} Accept
  Header. It will \textit{text/uri-list} an empty Content-type}

This rendering cannot render Kinds or Categories directly but just
Links to them. For concrete rendering of Kinds and Categories the
Content-types \textit{text/occi}, \textit{text/plain} MUST be used. If
a request is done with the \textit{text/uri-list} in the Accept
header, while not requesting for a Listing a Bad Request MUST be
returned.

\paragraph{Rendering a list of Categories}

MUST return a list of Category scheme URLs.

\paragraph{Rendering a list of Kinds}

MUST return a list of URLs pointing to the Kinds.

\section{Contributors}

Editors: Andy Edmonds, Thijs Metsch \\
Contributors: Alexander Papaspyrou, Ralf Nyrén, Sam Johnston

\textbf{TBD: Bunch op people missing here - create table\ldots}

\section{Appendix}

\subsection{Examples for rendering the OCCI model in the HTTP Header}

\subsubsection{Creating a Kind}
\begin{verbatim}

\end{verbatim}

\subsubsection{Triggering an Action}

\begin{verbatim}

\end{verbatim}

\subsubsection{Listing Categories}

\begin{verbatim}

\end{verbatim}

\subsubsection{Listing Resources}

\begin{verbatim}

\end{verbatim}

\subsubsection{Request with missing Category rendering}

\begin{verbatim}

\end{verbatim}

\subsubsection{Retrieving a Kind}

\begin{verbatim}

\end{verbatim}

\subsection{Examples for rendering the OCCI model with uri-lists}

\subsubsection{Listing Resources}

\begin{verbatim}

\end{verbatim}

\subsubsection{Listing Categories}

\begin{verbatim}
 
\end{verbatim}

\subsection{Examples for rendering the OCCI model in the HTTP Body}

\subsubsection{Listing of Categories}

\begin{verbatim}

\end{verbatim}

\section{Glossary}
\label{sec:glossary}
\begin{tabular}{l|p{11cm}}
Term & Description \\
\hline
\hl{Action} & An OCCI base type. Represent an invocable operation on a \hl{Entity} sub-type instance or collection thereof. \\
\hl{Category} & A type in the OCCI model. The parent type of \hl{Kind}. \\
Client & An OCCI client.\\
Collection & A set of \hl{Entity} sub-type instances all associated to a particular \hl{Kind} instance. \\
\hl{Entity} & An OCCI base type. The parent type of \hl{Resource} and \hl{Link}. \\
\hl{Kind} & A type in the OCCI model. The central piece in the OCCI classification system. \\
\hl{Link} & An OCCI base type. A \hl{Link} instance associate one \hl{Resource} instance with another. \\
Mix-in & A non-structural \hl{Kind}. The ``mix-in like'' concept in OCCI only
  support binding of new attributes and \hl{Action}s at run-time. A
  non-structural \hl{Kind} can only be associated with an {\em instance} of an
  \hl{Entity} sub-type. \\
Non-structural \hl{Kind} & An instance of \hl{Kind} {\em not} used as an unique identifier of an OCCI base type. \\
OCCI & Open Cloud Computing Interface \\
OCCI base type & One of \hl{Entity}, \hl{Resource}, \hl{Link} or \hl{Action}. \\
OGF & Open Grid Forum \\
\hl{Resource} & An OCCI base type. The parent type for all domain-specific resource types. \\
Structural \hl{Kind} & An instance of \hl{Kind} assigned as the unique identifier of an OCCI base type. \\
Tag & A non-structural \hl{Kind} with no attributes or actions defined. \\
URI & Uniform Resource Identifier \\
URL & Uniform Resource Locator \\
URN & Uniform Resource Name \\
\end{tabular}


\section{Intellectual Property Statement}
The OGF takes no position regarding the validity or scope of any
intellectual property or other rights that might be claimed to pertain
to the implementation or use of the technology described in this
document or the extent to which any license under such rights might or
might not be available; neither does it represent that it has made any
effort to identify any such rights. Copies of claims of rights made
available for publication and any assurances of licenses to be made
available, or the result of an attempt made to obtain a general
license or permission for the use of such proprietary rights by
implementers or users of this specification can be obtained from the
OGF Secretariat.

The OGF invites any interested party to bring to its attention any
copyrights, patents or patent applications, or other proprietary
rights which may cover technology that may be required to practice
this recommendation. Please address the information to the OGF
Executive Director.


\section{Disclaimer}
\input{include/disclaimer}

\section{Full Copyright Notice}
Copyright \copyright ~Open Grid Forum (2009-2016). All Rights Reserved.

This document and translations of it may be copied and furnished to
others, and derivative works that comment on or otherwise explain it
or assist in its implementation may be prepared, copied, published and
distributed, in whole or in part, without restriction of any kind,
provided that the above copyright notice and this paragraph are
included on all such copies and derivative works. However, this
document itself may not be modified in any way, such as by removing
the copyright notice or references to the OGF or other organizations,
except as needed for the purpose of developing Grid Recommendations in
which case the procedures for copyrights defined in the OGF Document
process must be followed, or as required to translate it into
languages other than English.

The limited permissions granted above are perpetual and will not be
revoked by the OGF or its successors or assignees.


\section{References}

\bibliographystyle{IEEEtran}
\bibliography{references}

\end{document}
