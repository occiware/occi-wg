\documentclass[10pt,a4paper]{article}
\usepackage[utf8]{inputenc}
\usepackage{fullpage}
\usepackage{graphicx}
\usepackage{fancyhdr}
\usepackage{comment}
\usepackage{occi}
\setlength{\headheight}{13pt}
\pagestyle{fancy}

% default sans-serif
\renewcommand{\familydefault}{\sfdefault}

% no lines for headers and footers
\renewcommand{\headrulewidth}{0pt}
\renewcommand{\footrulewidth}{0pt}

% header
\fancyhf{}
\lhead{Draft}
\rhead{\today}

% footer
\lfoot{occi-wg@ogf.org}
\rfoot{\thepage}

% paragraphs need some space...
\setlength{\parindent}{0pt}
\setlength{\parskip}{1ex plus 0.5ex minus 0.2ex}

% some space between header and text...
\headsep 13pt

\setcounter{secnumdepth}{4}

\begin{document}

% header on first page is different
\thispagestyle{empty}

Draft \hfill  {Ralf Nyrén, Independent}\\
OCCI-WG \hfill  Andy Edmonds, Zhaw
\rightline {Alexander Papaspyrou, Adesso}
\rightline {Thijs Metsch, Intel}
\rightline {April 7, 2011}\\
\rightline {Update: \today}

\vspace*{0.5in}

\begin{Large}
\textbf{Open Cloud Computing Interface - Core}
\end{Large}

\vspace*{0.5in}

\underline{Status of this Document}

% This document provides information to the community regarding the
specification of the Open Cloud Computing Interface. Distribution is
unlimited.


This document is a \underline{draft} including proposed errata updates to the
OCCI Core \cite{occi:core} specification.

The errata updates are summarized in section~\ref{sec:errata}.

Eventually this document will obsolete GFD-P-R.183. This document is fully backward compatible to \cite{occi:core}.

\underline{Copyright Notice}

Copyright \copyright ~Open Grid Forum (2009-2015). All Rights Reserved.

\underline{Trademarks}

OCCI is a trademark of the Open Grid Forum.

\underline{Abstract}

This document, part of a document series, produced by the OCCI working
group within the Open Grid Forum (OGF), provides a high-level
definition of a Protocol and API. The document is based upon
previously gathered requirements and focuses on the scope of important
capabilities required to support modern service offerings.


\newpage
\tableofcontents
\newpage

\section{Introduction}
The Open Cloud Computing Interface (OCCI) is a RESTful Protocol and
API for all kinds of Management tasks. OCCI was originally initiated
to create a remote management API for IaaS%
\footnote{Infrastructure as a Service}
model based Services, allowing for the development of interoperable tools for
common tasks including deployment, autonomic scaling and monitoring.
%
It has since evolved into an flexible API with a strong focus on
interoperability while still offering a high degree of extensibility. The
current release of the Open Cloud Computing Interface is suitable to serve many
other models in addition to IaaS, including e.g.~PaaS and SaaS.

In order to be modular and extensible the current OCCI specification is
released as a suite of complimentary documents which together form the complete
specification.
%
The documents are divided into three categories consisting of the OCCI Core,
the OCCI Renderings and the OCCI Extensions.
%
\begin{itemize}
\item The OCCI Core specification consist of a single document defining the
 OCCI Core Model. The OCCI Core Model can be interacted with {\em
 renderings} (including associated behaviours) and expanded through {\em extensions}.
\item The OCCI Rendering specifications consist of multiple documents each
 describing a particular rendering of the OCCI Core Model. Multiple renderings can
 interact with the same instance of the OCCI Core Model and will automatically support
 any additions to the model which follow the extension rules defined in OCCI
 Core.
\item The OCCI Extension specifications consist of multiple documents each
 describing a particular extension of the OCCI Core Model. The extension documents
 describe additions to the OCCI Core Model defined within the OCCI specification
 suite.
\end{itemize}
%
The current specification consist of three documents.
Future releases of OCCI may include additional rendering and extension
specifications. The documents of the current OCCI specification suite are:

\begin{description}
\item[OCCI Core] describes the formal definition of the the OCCI Core Model
\cite{occi:core}.
\item[OCCI HTTP Rendering] defines how to interact with the OCCI Core Model using the
RESTful OCCI API \cite{occi:http_rendering}. The document defines how the OCCI Core Model can
be communicated and thus serialised using the HTTP protocol.
\item[OCCI Infrastructure] contains the definition of the OCCI Infrastructure
extension for the IaaS domain \cite{occi:infrastructure}. The document defines
additional resource types, their attributes and the actions that can be taken
on each resource type.
\end{description}


\section{Notational Conventions}
All these parts and the information within are mandatory for
implementors (unless otherwise specified). The key words "MUST", "MUST
NOT", "REQUIRED", "SHALL", "SHALL NOT", "SHOULD", "SHOULD NOT",
"RECOMMENDED", "MAY", and "OPTIONAL" in this document are to be
interpreted as described in RFC 2119 \cite{rfc2119}.


\section{Terms and definitions}
Section \ref{sec:glossary} provides a glossary of all terms and
definitions with a specific meaning to the OCCI specification
suite. However, for reader convenience, a sub-set of the glossary is
provided here as well. The following terminology has specific meaning
in the OCCI context:

\begin{description}
  \item[capabilities] In the context of \hl{Entity} sub-types
    {\bf  capabilities} refer to the OCCI \hl{Attribute}s and OCCI \hl{Action}s
    exposed by a {\bf entity instance}.

  \item[entity instance] An instance of a sub-type of
    \hl{Entity} but not an instance of the \hl{Entity} type itself.
    The OCCI model defines two sub-types of \hl{Entity},
    the \hl{Resource} type and the \hl{Link} type.  However, the term
    {\bf entity instance} is defined to include any instance of a
    {\em sub-type} of \hl{Resource} or \hl{Link} as well.

  \item[mix-in] An instance of the \hl{Mixin} type associated with an
    {\bf entity instance}. The ``mix-in'' concept as used by OCCI
    {\em only} applies to instances, never to \hl{Entity} types.
    See section~\ref{sec:mixin}.

  \item[model attribute] An internal attribute of a the Core Model which is
    {\em not} client discoverable. Examples are \hl{Entity}.id,
    \hl{Link}.source and \hl{Link}.target. A model attribute is {\em not}
    identified by an \hl{Attribute} instance.

  \item[OCCI Attribute] A client discoverable attribute identified by an
    instance of the \hl{Attribute} type. Examples are \hl{occi.core.title}
    and \hl{occi.core.summary}. See section~\ref{sec:attribute}.

  \item[OCCI base type(s)] The OCCI base types are \hl{Entity},
    \hl{Resource} and \hl{Link}.
    See section~\ref{sec:base_types}.

  \item[template] A mechanism to provide default values for a {\bf entity
    instance}. See section~\ref{sec:template}.

  \item[type] A {\bf type} refer to one of those defined by the OCCI
    Core Model.  The OCCI Core Model types are \hl{Category},
    \hl{Attribute},
    \hl{Kind}, \hl{Mixin}, \hl{Action}, \hl{Entity}, \hl{Resource} and
    \hl{Link}.

  \item[concrete type/sub-type] A concrete sub-type is a type that can
    be instantiated.
\end{description}

\section{OCCI Core}
The Open Cloud Computing Interface is a boundary protocol and API that
acts as a service front-end to a provider's internal management
framework. Figure~\ref{fig:placement} shows OCCI's place in a
provider's architecture.

\begin{figure}[h]
  \centering
  \includegraphics[scale=0.5]{figs/occi-intro.pdf}
  \caption{OCCI's place in a provider's architecture.}
  \label{fig:placement}
\end{figure}

Service consumers can be both end-users and other system
instances. OCCI is suitable for both cases. The key feature is that
OCCI can be used as a management API for all kinds of resources while
at the same time maintaining a high level of interoperability.

This document, the OCCI Core specification, defines the OCCI Core
Model. This model is the core of the specification suite and it can be
interacted with by renderings (including associated behaviours) and
expanded through extensions. In itself, the core model is only useful
for a very limited set of use cases. However, it provides the basis
for renderings and extensions to build upon.

\section{OCCI Core Model}
The OCCI Core Model defines a representation of instance types which
can be manipulated through an OCCI protocol and rendering implementations.
It is an abstraction of real-world resources, including the means to identify,
classify, associate and extend those resources.

A fundamental feature of the OCCI Core Model is that it can be
extended in such a way that any extension will be discoverable and
visible to an OCCI client at run-time. An OCCI client can connect to
an OCCI implementation using an extended OCCI Core Model, without
knowing anything in advance, and still be able to discover and
understand, at run-time, the various instance types
supported by that implementation.
For example, a
web-based OCCI client could easily be reused as the management tool
for a wide variety of services.

The OCCI Core Model can be extended through inheritance but also
using a ``mix-in'' like concept.

\begin{quote}
  Mixins first appeared in the Symbolics' object-oriented
  Flavors~\cite{Moon:1986:flavors} system (developed by Howard
  Cannon), which was an approach to object-orientation used in Lisp
  Machine Lisp.%
  \footnote{http://en.wikipedia.org/wiki/Mixin.}
\end{quote}

The mix-in model only applies at the instance level, i.e.~the ``object
level'', and thereby differs from the more common uses of the mix-in
concept. A mix-in in OCCI can never be applied to a type, only to an
instance.

\subsection{Overview}

The UML class diagram shown in figure~\ref{fig:occi_model} gives an
overview of the OCCI Core Model. It must be noted that the UML diagram
in itself is not a complete definition of the model. The diagram is
merely provided as an overview to help understanding the model.

\begin{figure}[!h]
  {\centering \resizebox*{0.9\columnwidth}{!}{%\rotatebox{270}
      {\includegraphics{figs/core_model.pdf}}} \par}
  \caption{UML class diagram of the OCCI Core Model. The diagram
    provides an overview of the OCCI Core Model but is not a
    standalone definition thereof.}
  \label{fig:occi_model}
\end{figure}

The heart of the OCCI Core Model is the \hl{Resource} type. Any
resource exposed through OCCI is a \hl{Resource} or a sub-type
thereof.  A resource can be e.g.~a virtual machine, a job in a job
submission system, a user, etc.

The \hl{Resource} type contains a number of common attributes that
\hl{Resource} sub-types inherit. The \hl{Resource} type is
complemented by the \hl{Link} type which associates one \hl{Resource}
instance with another.
%
The \hl{Link} type contains a number of common attributes that
\hl{Link} sub-types inherit.

\hl{Entity} is an abstract type, which both \hl{Resource} and \hl{Link}
inherit.  Each sub-type of \hl{Entity} is identified by a unique
\hl{Kind} instance.

The \hl{Kind} type is the core of the type classification system built
into the OCCI Core Model. \hl{Kind} is a specialisation of
\hl{Category} and introduces additional capabilities in terms
of \hl{Action}s.  An \hl{Action} identifies an invocable operation
applicable to an entity instance.

\hl{Attribute} describe the name and properties of the OCCI Attributes found in
\hl{Entity} and its sub-types.

The last type defined by the OCCI Core Model is the \hl{Mixin}
type. An instance of \hl{Mixin} can be associated with an entity
instance to ``mix-in'' additional capabilities at run-time.

For compliance with OCCI Core, all of the types defined in the OCCI
Core Model MUST be implemented.  The following sections of the
specification contain the formal definition of the OCCI Core Model.

\subsection{Mutability}
\label{sec:mutability}
Attributes of an OCCI Core Model type instance are either client
mutable or client immutable. If an attribute is noted to be mutable
this MUST be interpreted that a client can create an instance that is
parametrised by the attribute. Likewise, if an attribute is mutable, a
client can update that instance's mutable attribute value and the
server side MUST support this. If an attribute is marked as immutable,
it indicates that the server side implementation MUST manage these
exclusively. Immutable attributes MUST NOT be modifiable by clients
under any circumstance.

\subsection{Classification and Identification}
\label{sec:classification}
The OCCI Core Model provides a built-in type classification system
allowing for safe extension towards domain-specific usage
(e.g. infrastructure). This system is the OCCI type system and offers
the means to be easily and transparently (i.e. no format translation
required) exposed over either a text- or binary-based protocol.

The classification system can be summarised with the following key
features:

\begin{itemize}
  \item Each OCCI base type and extension thereof is assigned a unique
    type identifier (a \hl{Kind} instance), which allow for dynamic
    discovery of available types. All \hl{Entity} sub-types, including
    core model extensions, are assigned a unique \hl{Kind} instance.

  \item The inheritance structure of \hl{Entity}, \hl{Resource} and
    \hl{Link} is client discoverable. This also applies to any
    sub-type of \hl{Resource} and \hl{Link} and therefore an OCCI
    client can discover the type inheritance structure used by a
    particular OCCI implementation. The discovery of the inheritance
    structure is made possible through the relationship of \hl{Kind}
    instances.

  \item The classification system allows \hl{Mixin} instances to be
    associated to entity instances in order to assign additional
    capabilities in terms of \hl{Attribute}s and \hl{Action}s at
    run-time.

  \item Tagging of entity instances is supported through the
    association of \hl{Mixin} instances. A tag is simply a \hl{Mixin}
    instance, which defines no additional capabilities.

  \item A collection of associated entity instances is implicitly
    defined for each \hl{Kind} and \hl{Mixin} instance. That is, all
    entity instances associated with a particular \hl{Kind} or
    \hl{Mixin} instance form a collection.
\end{itemize}

\subsubsection{Category}
\label{sec:category}
The \hl{Category} type is the basis of the type identification
mechanism used by the OCCI classification system. It MUST be
implemented.
There are no instances of the \hl{Category} type itself in the OCCI Core Model.
The \hl{Category} type is only used through its sub-types \hl{Kind}, \hl{Mixin}
and \hl{Action}.
%
Table~\ref{tbl:category} defines the model attributes the \hl{Category} type
MUST implement to be compliant.

\mytablefloat{
  \label{tbl:category}Model attributes defined for the \hl{Category} type.}{
  \begin{tabular}{llllp{2.7in}}
    \toprule
    Model attribute 	& Type 		& Multiplicity 	& Client Mutability 	& Description \\
    \colrule
    term 		& String 	& 1 			& Immutable 			& Unique identifier of the \hl{Category} instance within the categorisation scheme. \\
    scheme 		& URI 		& 1 			& Immutable 			& The categorisation scheme. \\
    title 		& String 	& 0..1 			& Immutable 			& The display name of an instance. \\
    \botrule
  \end{tabular}
}

A \hl{Category} instance is uniquely identified by concatenating the
categorisation scheme with the category term,
e.g.~\textit{http://example.com/category/scheme\#term}.  This is done
to enable discovery of \hl{Category} definitions in text-based
renderings such as HTTP. All renderings MUST make use of and
understand concatenated unique type identifiers of \hl{Category}
instances.
%
Sub-types of \hl{Category} such as \hl{Kind}, \hl{Mixin} and \hl{Action} inherit
this property.

The categorisation schemes defined in the OCCI specification all use
the \textit{http://schemas.ogf.org/occi/} base URL. This base URL is
reserved for OCCI an MUST NOT be used by service provider extensions.

A \hl{Category} instance%
\footnote{Also applies to \hl{Kind}, \hl{Mixin} and \hl{Action} instances.}
have zero or more associated \hl{Attribute} instances.
Each \hl{Attribute}, see section~\ref{sec:attribute},
describes the name and properties of single attribute.


\subsubsection{Attribute}
\label{sec:attribute}

The \hl{Attribute} type has a composite relationship to \hl{Category} and
defines the name and properties of client discoverable OCCI Attributes.
%
Table~\ref{tbl:attribute} defines the model attributes the \hl{Attribute} type
MUST implement to be compliant.

\mytablefloat{
  \label{tbl:attribute}Model attributes defined for the \hl{Attribute} type.}{
  \begin{tabular}{lp{23mm}llp{60mm}}
    \toprule
    Model attribute   &   Type    &   Multiplicity    &   Client Mutability   &   Description \\
    \colrule
    name    &   String    &   1   &   Immutable   & OCCI Attribute name. \\
    type    &   Enum \{Object, List, Hash\}   &   1   &   Immutable   &   OCCI Attribute type. \\
    mutable   &   Boolean   &   1   &   Immutable   &   OCCI Attribute mutability. \\
    required    &   Boolean   &   1   &   Immutable   &   Whether the OCCI Attribute must be supplied by the client at instance creation-time. \\
    pattern   &   String    &   0..1    &   Immutable   &   OCCI Attribute pattern expressed as PCRE \\
    default   &   String    &   0..1    &   Immutable   &   OCCI Attribute default value. \\
    description   &   String    &   0..1    &   Immutable   &   A description of the OCCI Attribute. \\
    \botrule
  \end{tabular}
}

An OCCI Attribute name MUST be defined by \hl{Attribute}.{\tt name}. The
OCCI Attribute namespace is flat and the ``\texttt{occi.}''~prefix is reserved
for the OCCI specification.
Domain-specific OCCI Attribute names MUST NOT contain the
``\texttt{occi.}''~prefix, instead they SHOULD use a prefix consisting of the
provider's reverse domain name. E.g.~``\texttt{com.example.}''.

An \hl{Attribute} MAY specify the following properties in addition to the OCCI
Attribute name. Attribute properties are OPTIONAL but MUST be client
discoverable if used.
\begin{description}
\item[type] The type of the OCCI Attribute. The types supported are ``Object'', ``List'' and ``Hash''.

\item[mutable] Whether a OCCI client can change the OCCI Attribute value. See
  section~\ref{sec:mutability}.

\item[required] If an OCCI Attribute is ``required'' a client MUST specify an
  value at instance creation-time.

\item[pattern] MAY be specified in ERE \cite{ere} format, places additional restrictions on possibles values given. 

\item[default] The default value given to an OCCI Attribute if the client does
  not specify a value at instance creation-time.  The {\em default} property is used to implement templates, see section~\ref{sec:template}.

\item[description] A summarizing description of the OCCI Attribute to
  complement the attribute name. For example, an interactive OCCI client may
  use the description property when presenting the content of an entity
  instance.
\end{description}


\subsubsection{Kind}
\label{sec:kind}

The \hl{Kind} type, together with the \hl{Mixin} type, defines the
classification system of the OCCI Core Model. It MUST be
implemented. The \hl{Kind} type represents the type identification
mechanism for all \hl{Entity} types present in the model.
%
Sub-types MUST NOT be derived from the \hl{Kind} type.

A unique \hl{Kind} {\em instance} MUST be assigned to each and every
\hl{Entity} sub-type defined in an OCCI implementation.

Every instance of \hl{Kind} represents a unique type identifier for a
particular sub-{\em type} of \hl{Entity}.  Consequently, when an \hl{Entity}
sub-type is instantiated the entity instance MUST be associated with
its type identifier, i.e.~the \hl{Kind} instance.  An entity instance
MUST remain associated with its \hl{Kind} instance throughout its
lifetime.
%
For example an instance of \hl{Resource} MUST always be associated
with the \hl{Kind} instance which identifies the \hl{Resource} {\em type}.

In the initial instantiation of the OCCI Core Model, with no core
model extensions, three instances of \hl{Kind} will be present: one
for \hl{Entity}, another for \hl{Resource} and the last one for
\hl{Link}.

\mytablefloat{
  \label{tbl:kind}Model attributes defined for the \hl{Kind} type.}{
  \begin{tabular}{llllp{2.7in}}
    \toprule
    Model attribute	& Type 		& Multiplicity 		& Client Mutability 		& Description \\
    \colrule
    actions 		& \hl{Action} 	& 0..* 			& Immutable 			& Set of \hl{Action} instances defined by the \hl{Kind} instance. \\
    parent 		& \hl{Kind} 	& 0..1 			& Immutable 			& Another \hl{Kind} instance which this \hl{Kind} relates to. \\
    entities 		& \hl{Entity} 	& 0..* 			& Immutable 			& Set of entity instances. Instances of the particular \hl{Entity} sub-type which is uniquely identified by this \hl{Kind} instance. \\
    \botrule
  \end{tabular}
}

The \hl{Kind} type inherits the \hl{Category} type. To be compliant
the \hl{Kind} type MUST implement the model attributes defined in
table~\ref{tbl:kind} and the inherited model attributes defined in
table~\ref{tbl:category}. The following rules apply to all instances
of the \hl{Kind} type:
%
\begin{itemize}
  \item A unique \hl{Kind} instance MUST be assigned to each and every
    sub-type of \hl{Entity}, including \hl{Entity} itself.

  \item A \hl{Kind} instance MUST expose the discoverable attributes defined for
    the \hl{Entity} sub-type it identifies. The \hl{Entity} attributes are
    described by \hl{Attribute} instances stored in the ``{\tt attributes}''
    model attribute inherited from
    \hl{Category}. E.g.~the \hl{Kind} instance identifying the
    \hl{Resource} type has \hl{Kind}.{\tt attributes} populated with a single \hl{Attribute}
    instance where \hl{Attribute}.{\tt name} is {\tt "occi.core.summary"}.

  \item A \hl{Kind} instance MUST expose the \hl{Action}s defined for
    its \hl{Entity} sub-type. \hl{Action}s are exposed through the
    \hl{Kind}.{\tt actions} model attribute which represent the association
    between a \hl{Kind} instance and the \hl{Action} instances it defines.

  \item A \hl{Kind} instance MUST have the \hl{Kind} instance of \hl{Entity}%
    \footnote{\textit{http://schemas.ogf.org/occi/core\#entity}}
    as its direct or indirect parent.
    The \hl{Kind}.{\tt parent} model attribute represent the relationship to
    another \hl{Kind} instance.

  \item If type {\bf B} inherits type {\bf A}, where {\bf A} is a
    sub-type of \hl{Entity}, the \hl{Kind} instance of {\bf B} MUST
    have its {\tt parent} attribute set to the \hl{Kind} instance of {\bf A}.
    See Kind Relationships below.
\end{itemize}

\paragraph*{Kind Relationships}
\hl{Kind} relationships are defined through the ``{\tt parent}''
model attribute present in every \hl{Kind} instance. The ``{\tt parent}''
model attribute define which other \hl{Kind} instance a particular
\hl{Kind} is related to.

A \hl{Kind} instance identifies a unique type, either the \hl{Entity}
type itself or a sub-type thereof.  Each \hl{Kind} instance MUST be
related to the \hl{Kind} of the parent type.

The OCCI base types \hl{Resource} and \hl{Link} both extend
\hl{Entity} and therefore their identifying \hl{Kind} instances MUST
have the \hl{Entity} \hl{Kind} instance as its parent.

These rules imply a hierarchy of related \hl{Kind} instances. The
\hl{Kind} relationships thus mirror the type inheritance structure of
the OCCI Core Model and any extension thereof.

\begin{figure}[!h]
  {\centering \resizebox*{0.9\columnwidth}{!}{\rotatebox{0}
      {\includegraphics{figs/kind_relationships.pdf}}} \par}
  \caption{Object diagram illustrating the \hl{Kind} instances
    involved for the \hl{Entity}, \hl{Resource} and \hl{Compute}
    types. The \hl{Compute} type is an extension to the OCCI Core
    Model defined in the OCCI Infrastructure
    document~\cite{occi:infrastructure}.}
  \label{fig:kind_relationships}
\end{figure}

Figure~\ref{fig:kind_relationships} illustrates the relationship of
the \hl{Kind} instances assigned to the \hl{Entity}, \hl{Resource} and
\hl{Compute}%
\footnote{The \hl{Compute} type is defined in the OCCI Infrastructure
 document~\cite{occi:infrastructure}.}
types.
%
\hl{Compute} inherits \hl{Resource} and therefore the \hl{Kind}
instance assigned to \hl{Compute} has the \hl{Kind} instance
of \hl{Resource} as its parent.
The same applies to the \hl{Resource} type which
inherit \hl{Entity}.

As can be seen in figure~\ref{fig:kind_relationships} the \hl{Kind}
instance relationships mirror the inheritance structure of the types.


\subsubsection{Mixin}
\label{sec:mixin}
The \hl{Mixin} type complements the \hl{Kind} type in defining the
OCCI Core Model type classification system. It MUST be
implemented. The \hl{Mixin} type represent an extension mechanism,
which allows new capabilities to be added to entity
instances both at creation-time and/or run-time.
%
Sub-types MUST NOT be derived from the \hl{Mixin} type.

A \hl{Mixin} {\em instance} can be associated with any existing
entity instance and thereby identify new capabilities,
i.e.~\hl{Attribute}s and \hl{Action}s, for the entity instance.
However, a
\hl{Mixin} can never be applied to a type.  In the initial
instantiation of the OCCI Core Model, with no extensions, no
\hl{Mixin} instances are present.

A \hl{Mixin} instance MAY be associated with an entity instance
either at instance creation-time or at run-time.
Restrictions on which entity instances a particular \hl{Mixin} can be associated
to SHOULD be advertised through the \hl{Mixin}.{\tt applies} model attribute.

When a client attempts to associate a \hl{Mixin} instance to an entity instance
at a stage not supported by a particular provider's OCCI
implementation, the provider MUST notify the client it has issued a
bad request.
%
For example a ``bandwidth'' \hl{Mixin} may only
be applicable to instances of the \hl{Network}%
\footnote{The \hl{Network} type is defined in OCCI
  Infrastructure~\cite{occi:infrastructure}.}  type.
An OCCI provider SHOULD advertise such a restriction by setting
\hl{Mixin}.{\tt applies} to the \hl{Kind} instance of the \hl{Network} type%
\footnote{\textit{http://schemas.ogf.org/occi/infrastructure\#network}}.

\mytablefloat{
  \label{tbl:mixin}Model attributes defined for the \hl{Mixin} type.}{
  \begin{tabular}{llllp{2.7in}}
    \toprule
    Model attribute	& Type 	& Multiplicity 	& Client Mutability 	& Description \\
    \colrule
    actions 	& \hl{Action} 	& 0..* 			& Immutable 			& Set of \hl{Action} instances defined by the \hl{Mixin} instance. \\
    depends 	& \hl{Mixin} 	& 0..* 			& Immutable 	& Set of \hl{Mixin} instances this \hl{Mixin} instance depends on. \\
    applies 	& \hl{Kind} 	& 0..* 			& Immutable 	& Set of \hl{Kind} instances this \hl{Mixin} instance applies to. \\
    entities 	& \hl{Entity} 	& 0..* 			& Mutable 				& Set of entity instances associated with the \hl{Mixin} instance. \\
    \botrule
  \end{tabular}
}

The \hl{Mixin} type inherits the \hl{Category} type. To be compliant
the \hl{Mixin} type MUST implement the model attributes defined in
table~\ref{tbl:mixin} and the inherited model attributes defined in
table~\ref{tbl:category}. The following rules apply to all instances
of the \hl{Mixin} type:
%
\begin{itemize}
  \item A \hl{Mixin} instance MUST only be associated with entity
    {\em instances}, not types, either at creation-time or run-time.

  \item A \hl{Mixin} instance is {\em only} a type identifier. It MUST NOT
    provide the implementation of the new capabilities it introduces.
    For example, a \hl{Mixin} instance never contains the value of an OCCI
    \hl{Attribute}.

  \item A \hl{Mixin} instance MAY introduce additional \hl{Attribute}s
    when applied to an entity instance. The name and properties of those
    OCCI Attributes MUST be exposed through \hl{Mixin}.{\tt attributes}
    inherited from \hl{Category}.  E.g.~a Location
    \hl{Mixin} defining the ``com.example.location'' OCCI Attribute MUST
    have Location.{\tt attributes} populated with a single \hl{Attribute}
    instance where \hl{Attribute}.{\tt name} is {\tt "com.example.location"}.

  \item A \hl{Mixin} instance MAY define \hl{Action} instances that will
    identify additional invocable operations on any entity instance
    associated with the
    \hl{Mixin}.  \hl{Action}s defined by a \hl{Mixin} are exposed
    through the \hl{Mixin}.{\tt actions} model attribute that represent the
    association between a \hl{Mixin} instance and the \hl{Action} instances it
    defines.

  \item A \hl{Mixin} instance MAY depend on another \hl{Mixin}
    instance.  If \hl{Mixin} {\bf B} depends on \hl{Mixin} {\bf A},
    any entity instance associated with \hl{Mixin} {\bf B} will
    receive the capabilities defined by both \hl{Mixin} {\bf
      B} and \hl{Mixin} {\bf A}.  See Mixin Relationships below.

  \item A \hl{Mixin} instance defining no additional
    capabilities is considered to be a tag.

  \item A \hl{Mixin} instance MAY be used as a template. A template
    defines default values for OCCI Attributes to be applied at entity instance
    creation-time. See section \ref{sec:template}.

  \item A \hl{Mixin} instance MAY restrict which \hl{Kind} instances it applies
    to using the {\tt applies} model attribute.  If \hl{Mixin}.{\tt applies}
    is unspecified the \hl{Mixin} may be associated to any entity instance,
    i.e.~equivalent of having \hl{Mixin}.{\tt applies} set to the \hl{Kind}
    instance of \hl{Entity}.

\end{itemize}

\paragraph*{Mixin Relationships}

A \hl{Mixin} instance MAY be depend on other \hl{Mixin} instances.
\hl{Mixin} relationships are implemented using the \hl{Mixin}.{\tt depends}
model attribute.
For example a set of operating
system templates, implemented as \hl{Mixin} instances, could be
related to an ``OS-template'' \hl{Mixin} in order to help
identification.

\hl{Attribute}s and \hl{Action}s defined by different \hl{Mixin} instances
are {\em combined} when \hl{Mixin} relationships are present. Therefore an
entity instance associated with a particular \hl{Mixin} will receive
the additional capabilities defined by any related \hl{Mixin}
instances as well as those defined by the \hl{Mixin} associated.


\subsubsection{Action}
The \hl{Action} type is the final part of the OCCI type classification system
and identifies invocable operations on individual entity instances and collections.
It MUST be implemented.
Each \hl{Action} instance identifies a single invocable operation.
The \hl{Action} instance is only an identifier and does not represent the
implementation of the operation.

The \hl{Action} type inherit the \hl{Category} type. To be compliant
the \hl{Action} type MUST implement the inherited model attributes defined in
table~\ref{tbl:category}.

\mytablefloat{
  \label{tbl:action_example}Example of an \hl{Action} instance which identifies a ``resize'' operation.}{
  \begin{tabular}{ll}
    \toprule
    Model attribute 	& Value \\
    \colrule
    term 		& resize \\
    scheme 		& {\tt http://schemas.ogf.org/occi/infrastructure/storage/action\#} \\
    title 		& Resize virtual disk \\
    attributes 		& Attribute({\tt "resize"}) \\
  \botrule
  \end{tabular}
}

An \hl{Action} instance MUST always bound to either a \hl{Kind} or a \hl{Mixin}
instance through a composite association. An \hl{Action} is considered
to be a capability of the \hl{Kind} or \hl{Mixin} instance it is
associated with.  The operation identified by an \hl{Action} MAY be invoked on
any entity
instance associated with the \hl{Kind} or \hl{Mixin} instance defining
the \hl{Action}. An OCCI implementation MAY however refuse an
the operation from being invoked if currently not applicable.

The operation identified by an \hl{Action} instance MAY be invoked on a
collection of \hl{Entity} sub-type
instances. The \hl{Action} is only considered valid if all entity
instances of the collection are associated with the \hl{Kind} or
\hl{Mixin} defining the \hl{Action} instance.

An \hl{Action} instance MAY identify OCCI Attributes which correspond to
parameters of the invocable operation.
The mechanism to define OCCI Attributes is inherited from \hl{Category}
and follow the same semantics.
The namespace restrictions imposed on entity instance attributes
(see~\ref{sec:attribute}) does however not apply to \hl{Action}s.

Table~\ref{tbl:action_example} shows an example of a ``resize'' operation
defined for a Storage
instance. The operation has a ``size'' parameter which represent the size
argument of the resize operation. In that example the identifying
\hl{Action} instance would have \hl{Action}.{\tt attributes} populated
with an \hl{Attribute} instance where \hl{Attribute}.{\tt name = "size"}.


\subsubsection{Instantiation}
\label{sec:instantiation}
To create an entity instance a client MUST supply the concrete
\hl{Entity} sub-type by a submitting a reference to the
type-identifying \hl{Kind}.  The reference MUST consist of the term
and categorisation scheme which uniquely identify the \hl{Kind}
instance, see section~\ref{sec:category}.  All OCCI implementations
MUST understand these requests.

A client MAY also submit any number of references to \hl{Mixin}
instances to be associated with the instance to be created. A
\hl{Mixin} reference submitted by a client MUST consist of the term
and categorisation scheme which identify the \hl{Mixin} instance, see
section~\ref{sec:category}.

\subsubsection{Templates}
\label{sec:template}

A template is a mechanism to provide default values for entity instances.
OCCI supports templates through \hl{Mixin}s.

A \hl{Mixin} instance associated at entity instance creation-time MAY provide default
values for OCCI Attributes.
Each default value is specified through \hl{Attribute}.{\tt default}.

A \hl{Mixin} instance MAY provide default values for OCCI Attributes already
defined by a \hl{Kind}. A \hl{Mixin}'s \hl{Attribute}.{\tt default} overrides
the default specified by the \hl{Kind}.

\subsubsection{Collections}
\label{sec:collection}
One or more entity instances associated with the same \hl{Kind} or
\hl{Mixin} instance, automatically form a collection.  Each \hl{Kind}
and \hl{Mixin} instance in the system identifies a collection
consisting of all different entity instances associated with the same
\hl{Kind} or \hl{Mixin}.

An entity instance is always a member of the collection indicated by
the \hl{Entity} sub-type's unique \hl{Kind} instance.
The \hl{Kind}.{\tt entities} model attribute implements the collection
of entity instances for a specific \hl{Entity} sub-type.

A \hl{Kind}
instance maintains the collection of all entity instances (of the
type identified by the \hl{Kind}).

Since a \hl{Mixin} instance can be associated to any entity
instance, a collection can contain entity instances of different
\hl{Entity} sub-types.
For example, an instance of the \hl{Resource} type will always be
associated to the \hl{Kind} instance
\textit{http://scheme.ogf.org/occi/core\#resource} and thus part of
the collection implied by that \hl{Kind} instance.
%
\begin{description}
  \item[Adding an entity instance] to a collection is accomplished by
    associating the entity instance to the corresponding \hl{Mixin}
    instance.
  \item[Removing an entity instance] from a collection is
    accomplished by disassociating the entity instance from the
    corresponding \hl{Mixin} instance.
\end{description}
%
An OCCI implementation MUST allow a client to navigate
collections. The following basic navigation operations MUST be
supported:
%
\begin{itemize}
  \item Retrieve the whole collection.
  \item Retrieve a specific item in a collection.
  \item Retrieve a subset of a collection.
\end{itemize}
%
The details of collection navigation is rendering specific.

\subsubsection{Discovery}
\label{sec:discovery}
An OCCI client MUST be able to discover all instances of \hl{Kind},
\hl{Mixin} and \hl{Category} a particular service provider's OCCI
implementation has defined. By examining these instances a client MUST
be able to, at a minimum, deduce the following information:
%
\begin{itemize}
  \item The \hl{Entity} sub-types available from the service provider,
    including core model extensions. This information is provided
    through the \hl{Kind} instances of the OCCI implementation.
  \item The attributes defined for each \hl{Entity} sub-type. The
    identifying \hl{Kind} instance provide this information.
  \item The invocable operations, i.e.~\hl{Action}s, defined for each
    \hl{Entity} sub-type. The identifying \hl{Kind} instance provide
    this information.
  \item Any \hl{Mixin} instances that can be associated to entity
    instances.
  \item Additional capabilities defined by a particular \hl{Mixin}
    instance, i.e.~\hl{Attribute}s and \hl{Action}s.
\end{itemize}
%
The above requirements comprise the OCCI discovery mechanism. It MUST
be implemented.

The details of exactly how the \hl{Category}, \hl{Kind} and \hl{Mixin}
instances are exposed to an OCCI client is specific to the particular
rendering used.
The relevant details can be found in the OCCI Rendering documents.

\subsection{The OCCI Core Base Types}
\label{sec:base_types}
The following sections describe the OCCI base types defined by the
OCCI Core Model.  The base types are \hl{Entity}, \hl{Resource},
\hl{Link}. All base types MUST be implemented.

An instance of the \hl{Resource} type, the \hl{Link} type or one of their
sub-types is called a {\em entity instance}.
Each entity instance within an OCCI system MUST have a unique identifier%
\footnote{An entity instance identifier MUST be unique within the service
provider's name-space. It is RECOMMENDED to use globally unique identifiers.}
stored in the {\tt id} model attribute of the \hl{Entity} type, as defined in
table~\ref{tbl:entity}.
%
The structure of these identifiers is opaque and the system should not assume a
static, pre-determined scheme for their structure other than the rules imposed
by the Uniform Resource Identifier (URI) \cite{rfc3986} syntax.

Although every unique entity instance identifier MUST be valid URI it is
RECOMMENDED to use the Uniform Resource Name (URN) \cite{rfc2141} syntax.

For example \hl{Entity}.{\tt id} could be
{\tt urn:uuid:de7335a7-07e0-4487-9cbd-ed51be7f2ce4}.

\subsubsection{Entity}
\label{sec:entity}
The \hl{Entity} type is an abstract type of the \hl{Resource} type and
the \hl{Link} type. It MUST be implemented.
%
Table~\ref{tbl:entity} defines the model attributes the \hl{Entity} type
MUST implement to be compliant. \todo{How does model attribute 'id' render into 'text/occi' or 'text/plain'? Would OCCI Attribute 'occi.core.id' be a better idea?}
%
\mytablefloat{
  \label{tbl:entity}Model attributes defined for the \hl{Entity} type.}{
  \begin{tabular}{lllp{1.3cm}p{1.0cm}p{7cm}}
    \toprule
    Model attribute	& Type 		& Multiplicity 	& Client Mutability 	& Discover\-able 	& Description \\
    \colrule
    id			& URI 		& 1 		& Immutable 		& Yes 			& A unique identifier (within the service provider's name-space) of the \hl{Entity} sub-type instance. \\
    kind 		& \hl{Kind}	& 1 		& Immutable 		& No 			& The \hl{Kind} instance uniquely identifying the particular \hl{Entity} sub-type of this instance. \\
    mixins 		& \hl{Kind}	& 0..* 		& Mutable 		& No 			& The \hl{Mixin} instances associated to this entity instance. Consumers can expect the \hl{Attribute}s and \hl{Action}s of the associated \hl{Mixin}s to be exposed by the instance. \\
    \botrule
  \end{tabular}
}

\hl{Entity} enforces for all sub-types an optional OCCI Attribute named
\texttt{occi.core.title}, see table~\ref{tbl:entity_attr}.
%
\mytablefloat{
  \label{tbl:entity_attr}OCCI Attributes defined by the \hl{Entity} type.}{
  \begin{tabular}{lllp{1.3cm}p{1.0cm}p{7cm}}
    \toprule
    OCCI Attribute	& Type 		& Multiplicity	& Client Mutability	& Discover\-able 	& Description \\
    \colrule
    occi.core.title 	& String	& 0..1		& Mutable 		& Yes 			& The display name of the instance. \\
    \botrule
  \end{tabular}
}

Every sub-type of \hl{Entity} MUST be assigned a \hl{Kind} instance,
see section~\ref{sec:kind}.
%
\mytablefloat{
  \label{tbl:entity_kind}The \hl{Kind} instance assigned to the \hl{Entity} type.}{
  \begin{tabular}{ll}
    \toprule
    Model attribute 	& Value \\
    \colrule
    term 		& entity \\
    scheme 		& {\tt http://schemas.ogf.org/occi/core\#} \\
    title 		& Entity type \\
    attributes 		& Attribute({\tt "occi.core.title"}) \\
    actions 	& -- \\
  \botrule
  \end{tabular}
}
%
\hl{Entity} itself is assigned the \hl{Kind} instance
\textit{http://schemas.ogf.org/occi/core\#entity} for type
identification, see table~\ref{tbl:entity_kind}.
%
Being an abstract type \hl{Entity} itself can never be instantiated.

An \hl{Entity} sub-type instance, also referred to as an {\em entity instance},
MAY be associated with one or more \hl{Mixin} instances.

An \hl{Entity} sub-type instance MUST expose its identifying \hl{Kind}
instance and any associated \hl{Mixin} instances together with the
\hl{Attribute}s and \hl{Action}s defined by them.

\subsubsection{Resource}
\label{sec:resource}
The \hl{Resource} type inherits \hl{Entity} and describes a concrete
resource that can be inspected and manipulated. It represents a
general object in the OCCI model and MUST be implemented. A
\hl{Resource} is suitable to represent real world resources,
e.g. virtual machines, networks, services, etc.~through
specialisation.

\mytablefloat{
  \label{tbl:resource}Model attributes defined for the \hl{Resource} type.}{
  \begin{tabular}{llllp{8cm}}
    \toprule
    Model attribute 	& Type 		& Multiplicity 	& Client Mutability 	& Description \\
    \colrule
    links 		& \hl{Link} & 0..* 			& Mutable 				& A set of \hl{Link} compositions. Being a composite relation the removal of a \hl{Link} from the set MUST also remove the \hl{Link} instance.\\
    \botrule
  \end{tabular}
}

The \hl{Resource} type MUST implement all model attributes and OCCI Attributes
inherited from
\hl{Entity} as well as the model and OCCI Attributes defined in
table~\ref{tbl:resource} and \ref{tbl:resource_attr} in order to be compliant.

\mytablefloat{
  \label{tbl:resource_attr}OCCI Attributes defined for the \hl{Resource} type.}{
  \begin{tabular}{llllp{8cm}}
    \toprule
    OCCI Attribute 	& Type		& Multiplicity 	& Client Mutability 	& Description \\
    \colrule
    occi.core.summary 	& String 	& 0..1 		& Mutable 		& A summarising description of the \hl{Resource} instance.\\
    \botrule
  \end{tabular}
}

The \hl{Resource} type is assigned the \hl{Kind} instance
\textit{http://schemas.ogf.org/occi/core\#resource}, see
table~\ref{tbl:resource_kind}.
%
\mytablefloat{
  \label{tbl:resource_kind}The \hl{Kind} instance assigned to the \hl{Resource} type.}{
  \begin{tabular}{ll}
    \toprule
    Model attribute 	& Value \\
    \colrule
    term 		& resource \\
    scheme 		& {\tt http://schemas.ogf.org/occi/core\#} \\
    title 		& Resource \\
    attributes	 	& Attribute({\tt occi.core.summary}) \\
    actions 	& -- \\
    \botrule
  \end{tabular}
}

\hl{Resource} enforces the inheritance of a set of common attributes
into sub-types. Moreover, it introduces relationships to other
\hl{Resource} instances through instances of the \hl{Link} type.

The \hl{Resource} type is the first of three entry points to extend
the OCCI Core Model, see section~\ref{sec:extensibility}.

\subsubsection{Link}
\label{sec:link}
An instance of the \hl{Link} type defines a base association between
two \hl{Resource} instances. It MUST be implemented. A \hl{Link}
instance indicates that one \hl{Resource} instance is connected to
another.

The \hl{Link} type MUST implement all attributes inherited from the
\hl{Entity} type together with the model attributes defined in
table~\ref{tbl:link} in order to be compliant. \todo{How does model attribute 'source' render into 'text/occi' or 'text/plain'? Would OCCI Attribute 'occi.core.source' be a better idea?}\\
\todo{How does model attribute 'target' render into 'text/occi' or 'text/plain'? Would OCCI Attribute 'occi.core.target' be a better idea?}

\mytablefloat{
  \label{tbl:link}Model attributes defined for the \hl{Link} type.}{
  \begin{tabular}{llllp{7.5cm}}
    \toprule
    Model attribute	& Type 		& Multiplicity 	& Client Mutability 	& Description \\
    \colrule
    source 		& \hl{Resource} & 1 		& Mutable 		& The \hl{Resource} instances the \hl{Link} instance originates from. \\
    target 		& \hl{URI} & 1 		& Mutable 		& The unique identifier of an \hl{Object} this \hl{Link} instance points to. MAY point outside of the known domain\todo{Formulation? Idea how to draw this in core\_model UML?}.\\
    \botrule
  \end{tabular}
}

The \hl{Link} type is assigned the \hl{Kind} instance
\textit{http://schemas.ogf.org/occi/core\#link}.

\mytablefloat{
  \label{tbl:link_kind}The \hl{Kind} instance assigned to the \hl{Link} type.}{
  \begin{tabular}{ll}
    \toprule
    Model attribute 	& Value \\
    \colrule
    term 		& link \\
    scheme 		& {\tt http://schemas.ogf.org/occi/core\#} \\
    title 		& Link \\
    attributes 	& -- \\
    actions 	& -- \\
    \botrule
  \end{tabular}
}

The {\tt source} and {\tt target} attribute of a \hl{Link} instance
MUST refer to \hl{Resource} {\em instances} within the service provider's
namespace. A \hl{Link} MAY refer to an external \hl{Resource} instance, i.e.~a
resource of which the service provider has no direct control, if and
only if that external resource is mapped into an \hl{Entity} sub-type
instance.

A provider MAY however introduce a sub-type of \hl{Link} with
different semantics, e.g.~having a target attribute containing an URI
and thus the ability of linking with external resources.

The \hl{Link} type is the second of three entry points to extend the
OCCI Core Model, see section~\ref{sec:extensibility}.


\subsection{Extensibility}
\label{sec:extensibility}
The OCCI Core Model has a flexible yet fairly simple extension
mechanism based on the type classification system described in
section~\ref{sec:classification}.

The OCCI Core Model can be extended using two different methods,
sub-typing and mix-in. Custom sub-typing require provider-specific
\hl{Kind} instances and custom mix-ins require provider-specific
\hl{Mixin} instances.  Both methods MAY involve the use of
provider-specific \hl{Action} instances.
The following sections
define the rules for extending the OCCI Core Model.

The rules defined in section~\ref{sec:classification} and
\ref{sec:base_types} are REQUIRED for all extensions of the OCCI Core
Model.

\subsubsection{Category instances}
\label{sec:ext:category}
Provider-specific instances of \hl{Category}, \hl{Kind} and \hl{Mixin}
MAY be introduced by an OCCI implementation. Since \hl{Kind} and
\hl{Mixin} both inherit \hl{Category} the extension rules for
\hl{Category}, defined below, applies to them as well.

A \hl{Category} instance defined outside of the OCCI specification
MUST use a \hl{Category} scheme unique to the provider,
e.g.~\textit{http://example.com/occi\#}. The {\tt term} of a
provider-specific \hl{Category} instance can be any string
corresponding to a ``token'' as defined by the OCCI Rendering documents.

An OCCI Attribute introduced by a provider-specific \hl{Category} MUST use
an attribute name prefix. This prefix MUST NOT be the
``\texttt{occi.}''~prefix which is reserved for the OCCI
specification. Domain-specific OCCI Attribute names SHOULD use a prefix
consisting of the provider's reverse domain name,
e.g.~``\texttt{com.example.}''.

\subsubsection{Sub-typing}
The OCCI Core Model MAY be extended through sub-typing.  Two OCCI
Core Model types MAY be sub-typed, those are \hl{Resource} and \hl{Link}.

In order to define a new sub-type of \hl{Resource} or \hl{Link}, a
provider-specific \hl{Kind} instance MUST be defined and assigned to
the new sub-type.
This provider-specific \hl{Kind} instance MUST have its \hl{Kind}.{\tt parent}
model attribute equal to the \hl{Kind} instance of the type extended.
See figure~\ref{fig:kind_relationships} for an example of \hl{Kind}
relationships.

\subsubsection{Mix-ins}
The OCCI Core Model MAY be extended using a ``mix-in'' like concept by
defining provider-specific \hl{Mixin} instances.  A \hl{Mixin}
instance can be associated with any entity instance although a
provider MAY apply restrictions.

In order to support user-defined tags%
\footnote{A tag is a \hl{Mixin} instance, which does not introduce
  additional capabilities.}
an OCCI implementation must allow custom \hl{Mixin}
instances to be created and destroyed by request of a client.  There
is no limitation in the OCCI Core Model from doing so but it is
RECOMMENDED to assign a separate \hl{Category} scheme for each user's
\hl{Mixin} instances (e.g. per-user schemes).

\section{Security Considerations}
Since the OCCI Core and Model specification describes a model, not an interface
or protocol, no specific security mechanisms are described as part of this
document. However, the elements described by this specification, namely type
instance attribute mutability, Category, Kind, and Mixin instantiations; Entity,
Resource, and Link subtypes,  whether direct or indirect; resource or
collection manipulation; and the discovery mechanism need to implement a proper
authorization scheme, which MUST be part of a concrete OCCI rendering
specification, part of an OCCI specification profile, or part of the specific
OCCI implementation.

Concrete security mechanisms and protection against attacks SHOULD be specified
by OCCI rendering specification. In any case, OCCI rendering specifications MUST
address transport level security and authentication on the protocol level.

All security considerations listed above apply to all (existing and future)
extensions of the OCCI Core and Model specification.

\section{Glossary}
\label{sec:glossary}
\begin{tabular}{l|p{11cm}}
Term & Description \\
\hline
\hl{Action} & An OCCI base type. Represent an invocable operation on a \hl{Entity} sub-type instance or collection thereof. \\
\hl{Category} & A type in the OCCI model. The parent type of \hl{Kind}. \\
Client & An OCCI client.\\
Collection & A set of \hl{Entity} sub-type instances all associated to a particular \hl{Kind} instance. \\
\hl{Entity} & An OCCI base type. The parent type of \hl{Resource} and \hl{Link}. \\
\hl{Kind} & A type in the OCCI model. The central piece in the OCCI classification system. \\
\hl{Link} & An OCCI base type. A \hl{Link} instance associate one \hl{Resource} instance with another. \\
Mix-in & A non-structural \hl{Kind}. The ``mix-in like'' concept in OCCI only
  support binding of new attributes and \hl{Action}s at run-time. A
  non-structural \hl{Kind} can only be associated with an {\em instance} of an
  \hl{Entity} sub-type. \\
Non-structural \hl{Kind} & An instance of \hl{Kind} {\em not} used as an unique identifier of an OCCI base type. \\
OCCI & Open Cloud Computing Interface \\
OCCI base type & One of \hl{Entity}, \hl{Resource}, \hl{Link} or \hl{Action}. \\
OGF & Open Grid Forum \\
\hl{Resource} & An OCCI base type. The parent type for all domain-specific resource types. \\
Structural \hl{Kind} & An instance of \hl{Kind} assigned as the unique identifier of an OCCI base type. \\
Tag & A non-structural \hl{Kind} with no attributes or actions defined. \\
URI & Uniform Resource Identifier \\
URL & Uniform Resource Locator \\
URN & Uniform Resource Name \\
\end{tabular}


\section{Contributors}

We would like to thank the following people who contributed to this
document:

\begin{tabular}{l|p{2in}|p{2in}}
Name & Affiliation & Contact \\
\hline
Michael Behrens & R2AD & behrens.cloud at r2ad.com \\
Mark Carlson & Oracle & mark.carlson at oracle.com \\
Andy Edmonds & Intel - SLA@SOI project & andy at edmonds.be \\
Sam Johnston & Google & samj at samj.net \\
Gary Mazzaferro & OCCI Counselour - AlloyCloud, Inc. &  garymazzaferro at gmail.com \\ 
Thijs Metsch & Platform Computing, Sun Microsystems & tmetsch at platform.com \\
Ralf Nyrén & Aurenav & ralf at nyren.net \\
Alexander Papaspyrou & TU Dortmund University & alexander.papaspyrou at tu\-dortmund.de \\
Alexis Richardson & RabbitMQ & alexis at rabbitmq.com \\
Shlomo Swidler & Orchestratus & shlomo.swidler at orchestratus.com \\
Florian Feldhaus & GWDG & florian.feldhaus at gwdg.de \\
Jean Parpaillon & & jean.parpaillon at free.fr \\
\end{tabular}

Next to these individual contributions we value the contributions from
the OCCI working group.


\section{Intellectual Property Statement}
The OGF takes no position regarding the validity or scope of any
intellectual property or other rights that might be claimed to pertain
to the implementation or use of the technology described in this
document or the extent to which any license under such rights might or
might not be available; neither does it represent that it has made any
effort to identify any such rights. Copies of claims of rights made
available for publication and any assurances of licenses to be made
available, or the result of an attempt made to obtain a general
license or permission for the use of such proprietary rights by
implementers or users of this specification can be obtained from the
OGF Secretariat.

The OGF invites any interested party to bring to its attention any
copyrights, patents or patent applications, or other proprietary
rights which may cover technology that may be required to practice
this recommendation. Please address the information to the OGF
Executive Director.


\section{Disclaimer}
\input{include/disclaimer}

\section{Full Copyright Notice}
Copyright \copyright ~Open Grid Forum (2009-2016). All Rights Reserved.

This document and translations of it may be copied and furnished to
others, and derivative works that comment on or otherwise explain it
or assist in its implementation may be prepared, copied, published and
distributed, in whole or in part, without restriction of any kind,
provided that the above copyright notice and this paragraph are
included on all such copies and derivative works. However, this
document itself may not be modified in any way, such as by removing
the copyright notice or references to the OGF or other organizations,
except as needed for the purpose of developing Grid Recommendations in
which case the procedures for copyrights defined in the OGF Document
process must be followed, or as required to translate it into
languages other than English.

The limited permissions granted above are perpetual and will not be
revoked by the OGF or its successors or assignees.


\bibliographystyle{IEEEtran}
\bibliography{references}

\appendix

\newpage
\section{Errata}
\label{sec:errata}

The corrections introduced by the {\today} errata update are summarized below.
The following sub-sections describe the possible impact of the corrections
on existing implementations and associated dependent specifications such
as OCCI HTTP Rendering \cite{occi:http_rendering} and  OCCI Infrastructure
\cite{occi:infrastructure}.

\begin{itemize}
\item Introduce an explicit \hl{Attribute} type to expose the discoverable
  attribute properties already defined for the OCCI base types \hl{Entity},
  \hl{Resource}, \hl{Link} and their sub-types.

\item Distinguish between discoverable OCCI \hl{Attribute}s and internal model
  attributes.

\item Correct the previously unclear definition of OCCI \hl{Action}. The
  \hl{Action} type inherits \hl{Category} and is only an identifier of
  an invocable operation. It does {\em not} represent the operation itself.
  The \hl{Action} definition now aligns with its use in the OCCI HTTP Rendering
  \cite{occi:http_rendering}.

\item Clarify the format of the unique entity instance identifier defined in
  OCCI \hl{Entity}. Incorporate the description and recommendations from the
  OCCI HTTP Rendering \cite{occi:http_rendering}.

\item Clarify that an OCCI \hl{Mixin} instance is only a type identifier. The
  Core Model does not specify how a mixed-in attribute is implemented. The
  \hl{Mixin} instance only states that the attribute exists.

\item Rename the term {\em resource instance} to {\em entity instance}.
  An {\em entity instance} refers to an instance of either OCCI \hl{Resource},
  OCCI \hl{Link} or a sub-type of either type. The {\em resource instance}
  term, while defined identically, was due to its name a source of
  misinterpretations in the specification.

\item Rename \hl{Kind}.{\tt related} to \hl{Kind}.{\tt parent} and
  \hl{Mixin}.{\tt related} to \hl{Mixin}.{\tt depends}. Clarify the use
  of \hl{Kind} and \hl{Mixin} relationships.

\item Add a new model attribute \hl{Mixin}.{\tt applies} to optionally
  advertise which entity instances a \hl{Mixin} instance may be associated to.
\end{itemize}

\subsection{Introducing the OCCI \hl{Attribute} type}

The \hl{Attribute} type formalizes how attribute properties are represented in
the OCCI Core Model. Since all attribute properties are optional no modifications
are necessary in existing implementations to remain compliant.

OCCI Infrastructure \cite{occi:infrastructure} defines attribute properties for
its sub-types of \hl{Entity}. The errata corrections allows these attribute
properties to be represented in the Core Model. However, the definitions remain the
same.

OCCI HTTP Rendering \cite{occi:http_rendering} already exposes the ``required'' and
``mutable'' attribute properties.

\subsection{OCCI \hl{Attribute}s versus model attributes}

The change is editorial and does not affect existing implementations.
%
The OCCI Infrastructure \cite{occi:infrastructure} specification only defines
discoverable OCCI \hl{Attribute}s although this is not explicitly stated.

\subsection{Action definition}

The corrected definition of OCCI \hl{Action} has no impact on neither discovery
nor invocation of \hl{Action}s in existing implementations. The OCCI HTTP Rendering
\cite{occi:http_rendering} is better aligned with OCCI Core after the corrections
since it already uses {\tt type="action"} in its rendering of categories.

\subsection{Rename ``resource instance'' to ``entity instance''}

The change is editorial and does not affect existing implementations.
%
The glossary contains both terms for compatibility with the OCCI HTTP Rendering
\cite{occi:http_rendering} specification.

\end{document}
