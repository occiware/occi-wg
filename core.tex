\documentclass[10pt,a4paper]{article}
\usepackage[utf8]{inputenc}
\usepackage{fullpage}
\usepackage{graphicx}
\usepackage{fancyhdr}
\usepackage{comment}
\usepackage{occi}
\setlength{\headheight}{13pt}
\pagestyle{fancy}

% default sans-serif
\renewcommand{\familydefault}{\sfdefault}

% no lines for headers and footers
\renewcommand{\headrulewidth}{0pt}
\renewcommand{\footrulewidth}{0pt}

% header
\fancyhf{}
\lhead{GWD-R}
\rhead{\today}

% footer
\lfoot{occi-wg@ogf.org}
\rfoot{\thepage}

% paragraphs need some space...
\setlength{\parindent}{0pt}
\setlength{\parskip}{1ex plus 0.5ex minus 0.2ex}

% some space between header and text...
\headsep 13pt

\setcounter{secnumdepth}{4}

\begin{document}

% header on first page is different
\thispagestyle{empty}

GWD-R \hfill  {Ralf Nyrén, Aurenav}\\
OCCI-WG \hfill  Andy Edmonds, Intel\\
\rightline {Alexander Papaspyrou, TU-Dortmund}
\rightline {Thijs Metsch, Platform Computing}
\rightline {October 14, 2010}\\
\rightline {Updated: \today}

\vspace*{0.5in}

\begin{Large}
\textbf{Open Cloud Computing Interface - Core}
\end{Large}

\vspace*{0.5in}

\underline{Status of this Document}

This document provides information to the community regarding the
specification of the Open Cloud Computing Interface. Distribution is
unlimited.


\underline{Obsoletes}

This document obsoletes all previous versions of this document.

\underline{Copyright Notice}

Copyright \copyright Open Grid Forum (2009-2011). All Rights Reserved.

\underline{Trademarks}

OCCI is a trademark of the Open Grid Forum.

\underline{Abstract}

This document, part of a document series, produced by the OCCI working
group within the Open Grid Forum (OGF), provides a high-level
definition of a Protocol and API. The document is based upon
previously gathered requirements and focuses on the scope of important
capabilities required to support modern service offerings.


\newpage
\tableofcontents
\newpage

\section{Introduction}
The Open Cloud Computing Interface (OCCI) is a RESTful Protocol and
API for all kinds of Management tasks. OCCI was originally initiated
to create a remote management API for IaaS%
\footnote{Infrastructure as a Service}
model based Services, allowing for the development of interoperable tools for
common tasks including deployment, autonomic scaling and monitoring.
%
It has since evolved into an flexible API with a strong focus on
interoperability while still offering a high degree of extensibility. The
current release of the Open Cloud Computing Interface is suitable to serve many
other models in addition to IaaS, including e.g.~PaaS and SaaS.

In order to be modular and extensible the current OCCI specification is
released as a suite of complimentary documents which together form the complete
specification.
%
The documents are divided into three categories consisting of the OCCI Core,
the OCCI Renderings and the OCCI Extensions.
%
\begin{itemize}
\item The OCCI Core specification consist of a single document defining the
 OCCI Core Model. The OCCI Core Model can be interacted with {\em
 renderings} (including associated behaviours) and expanded through {\em extensions}.
\item The OCCI Rendering specifications consist of multiple documents each
 describing a particular rendering of the OCCI Core Model. Multiple renderings can
 interact with the same instance of the OCCI Core Model and will automatically support
 any additions to the model which follow the extension rules defined in OCCI
 Core.
\item The OCCI Extension specifications consist of multiple documents each
 describing a particular extension of the OCCI Core Model. The extension documents
 describe additions to the OCCI Core Model defined within the OCCI specification
 suite.
\end{itemize}
%
The current specification consist of three documents.
Future releases of OCCI may include additional rendering and extension
specifications. The documents of the current OCCI specification suite are:

\begin{description}
\item[OCCI Core] describes the formal definition of the the OCCI Core Model
\cite{occi:core}.
\item[OCCI HTTP Rendering] defines how to interact with the OCCI Core Model using the
RESTful OCCI API \cite{occi:http_rendering}. The document defines how the OCCI Core Model can
be communicated and thus serialised using the HTTP protocol.
\item[OCCI Infrastructure] contains the definition of the OCCI Infrastructure
extension for the IaaS domain \cite{occi:infrastructure}. The document defines
additional resource types, their attributes and the actions that can be taken
on each resource type.
\end{description}


\section{Notational Conventions}
All these parts and the information within are mandatory for
implementors (unless otherwise specified). The key words "MUST", "MUST
NOT", "REQUIRED", "SHALL", "SHALL NOT", "SHOULD", "SHOULD NOT",
"RECOMMENDED", "MAY", and "OPTIONAL" in this document are to be
interpreted as described in RFC 2119 \cite{rfc2119}.


\section{OCCI Core}
The Open Cloud Computing Interface is a boundary protocol and API that
acts as a service front-end to a provider's internal management
framework. Figure~\ref{fig:placement} shows OCCI's place in a
provider's architecture.

\begin{figure}[h]
  \centering
  \includegraphics[scale=0.5]{figs/occi-intro.pdf}
  \caption{OCCI's place in a provider's architecture.}
  \label{fig:placement}
\end{figure}

Service consumers can be both end-users and other system
instances. OCCI is suitable for both cases. The key feature is that
OCCI can be used as a management API for all kinds of resources while
at the same time maintaining a high level of interoperability.

This document, the OCCI Core specification, defines the OCCI Core
Model. This model is the core of the specification suite and it can be
interacted with by renderings (including associated behaviours) and
expanded through extensions. In itself, the core model is only useful
for a very limited set of use cases. However, it provides the basis
for renderings and extensions to build upon.

\section{OCCI Core Model}
The OCCI Core Model defines a representation of instance types which
can be manipulated through an OCCI rendering implementation.  It is an
abstraction of real-world resources, including the means to identify,
classify, associate and extend those resources.

A fundamental feature of the OCCI Core Model is that it can be
extended in such a way that any extension will be discoverable and
visible to an OCCI client at run-time. An OCCI client can connect to
an OCCI implementation using an extended OCCI Core Model, without
knowing anything in advance, and still be able to discover and
understand, at run-time, the various \hl{Resource} and \hl{Link}
sub-types supported by that implementation. What \hl{Mixin}s are
supported is also discoverable in the same fashion. For example, a
web-based OCCI client could easily be reused as the management tool
for a wide variety of services.

The OCCI Core Model can be extended through inheritance but also
using a ``mix-in'' like concept.

\begin{quote}
  Mixins first appeared in the Symbolics' object-oriented
  Flavors~\cite{Moon:1986:flavors} system (developed by Howard
  Cannon), which was an approach to object-orientation used in Lisp
  Machine Lisp.%
  \footnote{http://en.wikipedia.org/wiki/Mixin.}
\end{quote}

The mix-in model only applies at the instance level, i.e.~the ``object
level'', and thereby differs from the more common uses of the mix-in
concept. A mix-in in OCCI can never be applied to a type, only to an
instance.

\subsection{Overview}

The UML class diagram shown in figure~\ref{fig:occi_model} gives an
overview of the OCCI Core Model. It must be noted that the UML diagram
in itself is not a complete definition of the model. The diagram is
merely provided as an overview to help understanding the model.

\begin{figure}[!h]
  {\centering \resizebox*{0.9\columnwidth}{!}{%\rotatebox{270}
      {\includegraphics{figs/core_model.pdf}}} \par}
  \caption{UML class diagram of the OCCI Core Model. The diagram
    provides an overview of the OCCI Core Model but is not a
    standalone definition thereof.}
  \label{fig:occi_model}
\end{figure}

The heart of the OCCI Core Model is the \hl{Resource} type. Any
resource exposed through OCCI is a \hl{Resource} or a sub-type
thereof.  A resource can be e.g.~a virtual machine, a job in a job
submission system, a user, etc.

The \hl{Resource} type contains a number of common attributes that
\hl{Resource} sub-types inherit. The \hl{Resource} type is
complemented by the \hl{Link} type which associates one \hl{Resource}
instance with another.
%
The \hl{Link} type contains a number of common attributes that
\hl{Link} sub-types inherit.

\hl{Entity} is an abstract type, which both \hl{Resource} and \hl{Link}
inherit.  Each sub-type of \hl{Entity} is identified by a unique
\hl{Kind} instance.

The \hl{Kind} type is the core of the type classification system built
into the OCCI Core Model. \hl{Kind} is a specialisation of
\hl{Category} and introduces additional resource capabilities in terms
of \hl{Action}s.  An \hl{Action} represents an invocable operation
applicable to a resource instance.

The last type defined by the OCCI Core Model is the \hl{Mixin}
type. An instance of \hl{Mixin} can be associated with a resource
instance, i.e.~a sub-type of \hl{Entity}, to ``mix-in'' additional
resource capabilities at run-time.

For compliance with OCCI Core, all of the types defined in the OCCI
Core Model MUST be implemented.  The following sections of the
specification contain the formal definition of the OCCI Core Model.

\subsection{Terms and definitions}
Section \ref{sec:glossary} provides a glossary of all terms and
definitions with a specific meaning to the OCCI specification
suite. However, for reader convenience, a sub-set of the glossary is
provided here as well. The following terminology has specific meaning
in the OCCI context:

\begin{description}
  \item[concrete type/sub-type] A concrete sub-type is a type that can
    be instantiated.
  \item[mix-in] An instance of the \hl{Mixin} type associated with a
    {\bf resource instance}. The ``mix-in'' concept as used by OCCI
    {\em only} applies to instances, never to \hl{Entity} types.
  \item[OCCI base type(s)] The OCCI base types are \hl{Entity},
    \hl{Resource}, \hl{Link} and \hl{Action}.
  \item[resource capabilities] Resource capabilities refer to
    attributes and \hl{Action}s exposed by a resource instance.
  \item[resource instance] An instance of a sub-type of
    \hl{Entity}. The OCCI model defines two sub-types of \hl{Entity},
    the \hl{Resource} type and the \hl{Link} type.  However, the term
    {\bf resource instance} is defined to include any instance of a
    {\em sub-type} of \hl{Resource} or \hl{Link} as well.
  \item[type] A {\bf type} refer to one of those defined by the OCCI
    Core Model.  The OCCI Core Model types are \hl{Category},
    \hl{Kind}, \hl{Mixin}, \hl{Action}, \hl{Entity}, \hl{Resource} and
    \hl{Link}.
\end{description}

\subsection{Mutability}
Attributes of an OCCI Core Model type instance are either client
mutable or client immutable. If an attribute is noted to be mutable
this MUST be interpreted that a client can create an instance that is
parametrised by the attribute. Likewise, if an attribute is mutable, a
client can update that instance's mutable attribute value and the
server side MUST support this. If an attribute is marked as immutable,
it indicates that the server side implementation MUST manage these
exclusively. Immutable attributes MUST NOT be modifiable by clients
under any circumstance.

\subsection{Classification and Identification}
\label{sec:classification}
The OCCI Core Model provides a built-in type classification system
allowing for safe extension towards domain-specific usage
(e.g. infrastructure). This system is the OCCI type system and offers
the means to be easily and transparently (i.e. no format translation
required) exposed over either a text- or binary-based protocol.

The classification system can be summarised with the following key
features:

\begin{itemize}
  \item Each OCCI base type and extension thereof is assigned a unique
    type identifier (a \hl{Kind} instance), which allow for dynamic
    discovery of available types. All \hl{Entity} sub-types, including
    core model extensions, are assigned a unique \hl{Kind} instance.
  \item The inheritance structure of \hl{Entity}, \hl{Resource} and
    \hl{Link} is client discoverable. This also applies to any
    sub-type of \hl{Resource} and \hl{Link} and therefore an OCCI
    client can discover the type inheritance structure used by a
    particular OCCI implementation. The discovery of the inheritance
    structure is made possible through the relationship of \hl{Kind}
    instances.
  \item The classification system allows \hl{Mixin} instances to be
    associated to resource instances in order to assign additional
    resource capabilities in terms of attributes and \hl{Action}s at
    run-time.
  \item Tagging of resource instances is supported through the
    association of \hl{Mixin} instances. A tag is simply a \hl{Mixin}
    instance, which defines no additional resource capabilities.
  \item A collection of associated resource instances is implicitly
    defined for each \hl{Kind} and \hl{Mixin} instance. That is, all
    resource instances associated with a particular \hl{Kind} or
    \hl{Mixin} instance form a collection.
\end{itemize}

\subsubsection{Category}
\label{sec:category}
The \hl{Category} type is the basis of the type identification
mechanism used by the OCCI classification system. It MUST be
implemented. Instances of the \hl{Category} type itself are only used
to identify \hl{Action} types. All other uses of \hl{Category}
properties are managed through its sub-types \hl{Kind} and \hl{Mixin}.
%
Table~\ref{tbl:category} defines the attributes the \hl{Category} type
MUST implement to be compliant.

\mytablefloat{
  \label{tbl:category}Attributes defined for the \hl{Category} type.}{
  \begin{tabular}{llllp{2.7in}}
    \toprule
	Attribute 	& Type 		& Multiplicity 	& Client Mutability 	& Description \\
    \colrule
    term 		& String 	& 1 			& Immutable 			& Unique identifier of the \hl{Category} instance within the categorisation scheme. \\
    scheme 		& URI 		& 1 			& Immutable 			& The categorisation scheme. \\
    title 		& String 	& 0..1 			& Immutable 			& The display name of an instance. \\
    attributes 	& String 	& 0..* 			& Immutable 			& The set of resource attribute names defined by the \hl{Category} instance. \\
    \botrule
  \end{tabular}
}

A \hl{Category} instance is uniquely identified by concatenating the
categorisation scheme with the category term,
e.g.~\textit{http://example.com/category/scheme\#term}.  This is done
to enable discovery of \hl{Category} definitions in text-based
renderings such as HTTP. All renderings MUST make use of and
understand concatenated unique type identifiers of \hl{Category}
instances.
%
Sub-types of \hl{Category} such as \hl{Kind} and \hl{Mixin} inherit
this property.

The categorisation schemes defined in the OCCI specification all use
the \textit{http://schemas.ogf.org/occi/} base URL. This base URL is
reserved for OCCI an MUST NOT be used by service provider extensions.

A \hl{Category} instance%
\footnote{Also applies to \hl{Kind} and \hl{Mixin} instances.}
defines the {\em names} of the attributes exposed by any instance associated
with the \hl{Category}.  For example a ``resize'' \hl{Action} having a {\tt
size} attribute would have an identifying \hl{Category} with \hl{Category}.{\tt
attributes = [ size ]}.

\subsubsection{Kind}
\label{sec:kind}

The \hl{Kind} type, together with the \hl{Mixin} type, defines the
classification system of the OCCI Core Model. It MUST be
implemented. The \hl{Kind} type represents the type identification
mechanism for all \hl{Entity} types present in the model.

A unique \hl{Kind} {\em instance} MUST be assigned to each and every
\hl{Entity} sub-type defined in an OCCI implementation.

Every instance of \hl{Kind} represents a unique type identifier for a
particular sub-type of \hl{Entity}.  Consequently, when an \hl{Entity}
sub-type is instantiated the resource instance MUST be associated with
its type identifier, i.e.~the \hl{Kind} instance.  A resource instance
MUST remain associated with its \hl{Kind} instance throughout its
lifetime.
%
For example an instance of \hl{Resource} MUST always be associated
with the \hl{Kind} instance which identifies the \hl{Resource} {\em type}.

In the initial instantiation of the OCCI Core Model, with no core
model extensions, three instances of \hl{Kind} will be present: one
for \hl{Entity}, another for \hl{Resource} and the last one for
\hl{Link}.

\mytablefloat{
  \label{tbl:kind}Attributes defined for the \hl{Kind} type.}{
  \begin{tabular}{llllp{2.7in}}
    \toprule
    Attribute 		& Type 			& Multiplicity 	& Client Mutability 	& Description \\
    \colrule
    actions 		& \hl{Action} 	& 0..* 			& Immutable 			& Set of \hl{Action}s defined by the \hl{Kind} instance. \\
    related 		& \hl{Kind} 	& 0..* 			& Immutable 			& Set of related \hl{Kind} instances. \\
    entity\_type 	& \hl{Entity} 	& 1 			& Immutable 			& \hl{Entity} type uniquely identified by the \hl{Kind} instance. \\
    entities 		& \hl{Entity} 	& 0..* 			& Immutable 			& Set of resource instances, i.e.~\hl{Entity} sub-type instances. Resources instantiated from the \hl{Entity} sub-type which is uniquely identified by this \hl{Kind} instance. \\
    \botrule
  \end{tabular}
}

The \hl{Kind} type inherits the \hl{Category} type. To be compliant
the \hl{Kind} type MUST implement the attributes defined in
table~\ref{tbl:kind} and the inherited attributes defined in
table~\ref{tbl:category}. The following rules apply to all instances
of the \hl{Kind} type:
%
\begin{itemize}
  \item A unique \hl{Kind} instance MUST be assigned to each and every
    sub-type of \hl{Entity}, including \hl{Entity} itself.
  \item A \hl{Kind} instance MUST expose the attribute names of the
    \hl{Entity} sub-type it identifies. The attribute names are
    exposed through the ``{\tt attributes}'' attribute inherited from
    \hl{Category}. E.g.~the \hl{Kind} instance identifying the
    \hl{Link} type has \hl{Kind}.{\tt attributes = [ occi.core.source, occi.core.target ]}.
  \item A \hl{Kind} instance MUST expose the \hl{Action}s defined for
    its \hl{Entity} sub-type. \hl{Action}s are exposed through the
    \hl{Kind}.{\tt actions} attribute which represent the association
    between a \hl{Kind} instance and the \hl{Action}s it defines.
  \item A \hl{Kind} instance MUST be related, either directly or
    indirectly, to the \hl{Kind} instance of \hl{Entity},
    i.e. \textit{http://schemas.ogf.org/occi/core\#entity}.  The
    \hl{Kind}.{\tt related} attribute represent the relationship to
    another \hl{Kind} instance.
  \item If type {\bf B} inherits type {\bf A}, where {\bf A} is a
    sub-type of \hl{Entity}, the \hl{Kind} instance of {\bf B} MUST be
    directly related to the \hl{Kind} instance of {\bf A}. See Kind
    Relationships below.
\end{itemize}

\paragraph*{Kind Relationships}
\hl{Kind} relationships are defined through the {\tt related}
attribute present in every \hl{Kind} instance. The {\tt related}
attribute define which other \hl{Kind} instances a particular
\hl{Kind} is related to.

A \hl{Kind} instance identifies a unique type, either the \hl{Entity}
type itself or a sub-type thereof.  Each \hl{Kind} instance MUST be
related to the \hl{Kind} of the parent type.

The OCCI base types \hl{Resource} and \hl{Link} both extend
\hl{Entity} and therefore their identifying \hl{Kind} instances MUST
be related to \hl{Kind} assigned to the \hl{Entity} type.

These rules imply a hierarchy of related \hl{Kind} instances. The
\hl{Kind} relationships thus mirror the type inheritance structure of
the OCCI Core Model and any extension thereof.

\begin{figure}[!h]
  {\centering \resizebox*{0.9\columnwidth}{!}{\rotatebox{0}
      {\includegraphics{figs/kind_relationships.pdf}}} \par}
  \caption{Object diagram illustrating the \hl{Kind} instances
    involved for the \hl{Entity}, \hl{Resource} and \hl{Compute}
    types. The \hl{Compute} type is an extension to the OCCI Core
    Model defined in the OCCI Infrastructure
    document~\cite{occi:infrastructure}.}
  \label{fig:kind_relationships}
\end{figure}

Figure~\ref{fig:kind_relationships} illustrates the relationship of
the \hl{Kind} instances assigned to the \hl{Entity}, \hl{Resource} and
\hl{Compute}%
\footnote{The \hl{Compute} type is defined in the OCCI Infrastructure
 document~\cite{occi:infrastructure}.}
types.
%
\hl{Compute} inherits \hl{Resource} and therefore the \hl{Kind}
instance assigned to \hl{Compute} is related to the \hl{Kind} instance
of \hl{Resource}.  The same applies to the \hl{Resource} type which
inherit \hl{Entity}.

As can be seen in figure~\ref{fig:kind_relationships} the \hl{Kind}
instance relationships mirror the inheritance structure of the types.

\subsubsection{Mixin}
The \hl{Mixin} type complements the \hl{Kind} type in defining the
OCCI Core Model type classification system. It MUST be
implemented. The \hl{Mixin} type represent an extension mechanism,
which allows new resource capabilities to be added to resource
instances both at creation-time and/or run-time.

A \hl{Mixin} {\em instance} can be associated with any existing
resource instance and thereby add new resource capabilities,
i.e.~attributes and \hl{Action}s, to the resource instance. However, a
\hl{Mixin} can never be applied to a type.  In the initial
instantiation of the OCCI Core Model, with no extensions, no
\hl{Mixin} instances are present.

A \hl{Mixin} instance MAY be associated with any resource instance,
either at instance creation-time or at run-time. Although the OCCI
Core Model has no such restrictions, an OCCI implementation MAY impose
restrictions on which resource instances can be associated with a
particular \hl{Mixin} instance.

When a client attempts to associate a \hl{Mixin} instance to a
resource at a stage not supported by a particular provider's OCCI
implementation, the provider MUST notify the client it has issued a
bad request.
%
For example a ``geographical location'' \hl{Mixin} might by applicable
to all resource instances while a ``bandwidth'' \hl{Mixin} may only
applicable to resources instantiated from the \hl{Network}%
\footnote{The \hl{Network} type is defined in OCCI
  Infrastructure~\cite{occi:infrastructure}.}  type. Such
restrictions, if not otherwise stated, are up to the provider to
implement.

\mytablefloat{
  \label{tbl:mixin}Attributes defined for the \hl{Mixin} type.}{
  \begin{tabular}{llllp{2.7in}}
    \toprule
    Attribute 	& Type 			& Multiplicity 	& Client Mutability 	& Description \\
    \colrule	
    actions 	& \hl{Action} 	& 0..* 			& Immutable 			& Set of \hl{Action}s defined by the \hl{Mixin} instance. \\
    related 	& \hl{Mixin} 	& 0..* 			& Immutable 			& Set of related \hl{Mixin} instances. \\
    entities 	& \hl{Entity} 	& 0..* 			& Mutable 				& Set of resource instances, i.e.~\hl{Entity} sub-type instances, associated with the \hl{Mixin} instance. \\
    \botrule
  \end{tabular}
}

The \hl{Mixin} type inherits the \hl{Category} type. To be compliant
the \hl{Mixin} type MUST implement the attributes defined in
table~\ref{tbl:mixin} and the inherited attributes defined in
table~\ref{tbl:category}. The following rules apply to all instances
of the \hl{Mixin} type:
%
\begin{itemize}
  \item A \hl{Mixin} instance MUST only be associated with resource
    {\em instances}, not types, either at creation-time or run-time.
  \item A \hl{Mixin} instance MAY introduce additional resource
    attributes when applied to a resource instance. The names of those
    attributes MUST be exposed through the \hl{Mixin}.{\tt attributes}
    attribute inherited from \hl{Category}.  E.g.~a Location
    \hl{Mixin} defining the ``com.example.location'' attribute MUST
    have Location.{\tt attributes = [ com.example.location ]}.
  \item A \hl{Mixin} instance MAY define \hl{Action}s that will be
    made applicable to any resource instance associated with the
    \hl{Mixin}.  \hl{Action}s defined by a \hl{Mixin} are exposed
    through the \hl{Mixin}.{\tt actions} attribute that represent the
    association between a \hl{Mixin} instance and the \hl{Action}s it
    defines.
  \item A \hl{Mixin} instance MAY be related to another \hl{Mixin}
    instance.  If \hl{Mixin} {\bf B} is related to \hl{Mixin} {\bf A},
    any resource instance associated with \hl{Mixin} {\bf B} will
    receive the resource capabilities defined by both \hl{Mixin} {\bf
      B} and \hl{Mixin} {\bf A}.  See Mixin Relationships below.
  \item A \hl{Mixin} instance defining no additional resource
    capabilities is considered to be a tag.
  \item A \hl{Mixin} instance applied a resource instantiation time
    MAY cause additional provider-defined side-effects to occur,
    side-effects not visible through the OCCI discovery
    mechanism. Templates that pre-populate certain attributes of a
    resource instance SHOULD be implemented using such \hl{Mixin}
    instances.
\end{itemize}

\paragraph*{Mixin Relationships}

A \hl{Mixin} instance MAY be related to another \hl{Mixin} instance
for type classification purposes. For example a set of operating
system templates, implemented as \hl{Mixin} instances, could be
related to an ``OS-template'' \hl{Mixin} in order to help
identification.

Attributes and \hl{Action}s defined by different \hl{Mixin} instances
are combined when \hl{Mixin} relationships are present. Therefore a
resource instance associated with a particular \hl{Mixin} will receive
the additional capabilities defined by any related \hl{Mixin}
instances as well as those defined by the \hl{Mixin} associated.

\subsubsection{Resource Instantiation}
\label{sec:instantiation}
To create a resource instance a client MUST supply the concrete
\hl{Entity} sub-type by a submitting a reference to the
type-identifying \hl{Kind}.  The reference MUST consist of the term
and categorisation scheme which uniquely identify the \hl{Kind}
instance, see section~\ref{sec:category}.  All OCCI implementations
MUST understand these requests.

A client MAY also submit any number of references to \hl{Mixin}
instances to be associated with the resource to be created. A
\hl{Mixin} reference submitted by a client MUST consist of the term
and categorisation scheme which identify the \hl{Mixin} instance, see
section~\ref{sec:category}.

Associating a \hl{Mixin} at resource instantiation time MAY have
additional provider defined side-effects, side-effects not visible
through the OCCI discovery mechanism. Templates that pre-populate
certain attributes of a resource instance SHOULD be implemented using
such \hl{Mixin} instances.

\subsubsection{Collections}
\label{sec:collection}
One or more resource instances associated with the same \hl{Kind} or
\hl{Mixin} instance, automatically form a collection.  Each \hl{Kind}
and \hl{Mixin} instance in the system identifies a collection
consisting of all different resource instances associated with the same
\hl{Kind} or \hl{Mixin}.

A resource instance is always a member of the collection indicated by
the \hl{Entity} sub-type's unique \hl{Kind} instance. A \hl{Kind}
instance maintains the collection of all resource instances (of the
type identified by the \hl{Kind}).

Since a \hl{Mixin} instance can be associated to any resource
instance, a collection can contain resource instances of different
\hl{Entity} sub-types.
For example, an instance of the \hl{Resource} type will always be
associated to the \hl{Kind} instance
\textit{http://scheme.ogf.org/occi/core\#resource} and thus part of
the collection implied by that \hl{Kind} instance.
%
\begin{description}
  \item[Adding a resource instance] to a collection is accomplished by
    associating the resource instance to the corresponding \hl{Mixin}
    instance.
  \item[Removing a resource instance] from a collection is
    accomplished by disassociating the resource instance from the
    corresponding \hl{Mixin} instance.
\end{description}
%
An OCCI implementation MUST allow a client to navigate
collections. The following basic navigation operations MUST be
supported:
%
\begin{itemize}
  \item Retrieve the whole collection.
  \item Retrieve a specific item in a collection.
  \item Retrieve a subset of a collection.
\end{itemize}
%
The details of collection navigation is rendering specific.

\subsubsection{Discovery}
\label{sec:discovery}
An OCCI client MUST be able to discover all instances of \hl{Kind},
\hl{Mixin} and \hl{Category} a particular service provider's OCCI
implementation has defined. By examining these instances a client MUST
be able to, at a minimum, deduce the following information:
%
\begin{itemize}
  \item The \hl{Entity} sub-types available from the service provider,
    including core model extensions. This information is provided
    through the \hl{Kind} instances of the OCCI implementation.
  \item The attributes defined for each \hl{Entity} sub-type. The
    identifying \hl{Kind} instance provide this information.
  \item The invocable operations, i.e.~\hl{Action}s, defined for each
    \hl{Entity} sub-type. The identifying \hl{Kind} instance provide
    this information.
  \item Any \hl{Mixin} instances that can be associated to resource
    instances.
  \item Additional capabilities defined by a particular \hl{Mixin}
    instance, i.e.~attributes and \hl{Action}s.
\end{itemize}
%
The above requirements comprise the OCCI discovery mechanism. It MUST
be implemented.

The details of exactly how the \hl{Category}, \hl{Kind} and \hl{Mixin}
instances are exposed to an OCCI client is specific to the particular
rendering used.
The relevant details can be found in the OCCI Rendering documents.

\subsection{The OCCI Core Base Types}
\label{sec:base_types}
The following sections describe the OCCI base types defined by the
OCCI Core Model.  The base types are \hl{Entity}, \hl{Resource},
\hl{Link} and \hl{Action}. All base types MUST be implemented.

\subsubsection{Entity}
\label{sec:entity}
The \hl{Entity} type is an abstract type of the \hl{Resource} type and
the \hl{Link} type. It MUST be implemented.
%
Table~\ref{tbl:entity} defines the attributes the \hl{Entity} type
MUST implement to be compliant.
%
\mytablefloat{
  \label{tbl:entity}Attributes defined for the \hl{Entity} type.}{
  \begin{tabular}{lllp{1.3cm}p{1.0cm}p{7cm}}
    \toprule
    Attribute 	& Type 		& Multiplicity 	& Client Mutability 	& Discover\-able 	& Description \\
    \colrule
    occi.core.id		& URI 		& 1 			& Immutable 			& Yes 				& A unique identifier (within the service provider's name-space) of the \hl{Entity} sub-type instance. \\
    occi.core.title 		& String 	& 0..1 			& Mutable 				& Yes 				& The display name of the instance. \\
    kind 		& \hl{Kind} & 1 			& Immutable 			& No 				& The \hl{Kind} instance uniquely identifying the \hl{Entity} sub-type of the resource instance. \\
    mixins 		& \hl{Kind} & 0..* 			& Mutable 				& No 				& The \hl{Mixin} instances associated to this resource instance. Consumers can expect the attributes and \hl{Action}s of the associated \hl{Mixin}s to be exposed by the instance. \\ 
    \botrule
  \end{tabular}
}

\hl{Entity} enforces for all sub-types a required \texttt{occi.core.id}
attribute and an optional \texttt{occi.core.title} attribute.

Every sub-type of \hl{Entity} MUST be assigned a \hl{Kind} instance,
see section~\ref{sec:kind}.
%
\mytablefloat{
  \label{tbl:entity_kind}The \hl{Kind} instance assigned to the \hl{Entity} type.}{
  \begin{tabular}{ll}
    \toprule
    Attribute 	& Value \\
    \colrule
    term 		& entity \\
    scheme 		& {\tt http://schemas.ogf.org/occi/core\#} \\
    title 		& Entity type \\
    attributes 	& occi.core.id, occi.core.title \\
    actions 	& -- \\
  \botrule
  \end{tabular}
}
%
\hl{Entity} itself is assigned the \hl{Kind} instance
\textit{http://schemas.ogf.org/occi/core\#entity} for type
identification, see table~\ref{tbl:entity_kind}.
%
Being an abstract type \hl{Entity} itself can never be instantiated.

An \hl{Entity} sub-type instance, also referred to as a resource instance,
MAY be associated with one or more \hl{Mixin} instances.

An \hl{Entity} sub-type instance MUST expose its identifying \hl{Kind}
instance and any associated \hl{Mixin} instances together with the
attributes and \hl{Action}s defined by them.

\subsubsection{Resource}
\label{sec:resource}
The \hl{Resource} type inherits \hl{Entity} and describes a concrete
resource that can be inspected and manipulated. It represents a
general object in the OCCI model and MUST be implemented. A
\hl{Resource} is suitable to represent real world resources,
e.g. virtual machines, networks, services, etc.~through
specialisation.

The \hl{Resource} type MUST implement all attributes inherited from
\hl{Entity} as well as the attributes defined in
table~\ref{tbl:resource} in order to be compliant.
%
\mytablefloat{
  \label{tbl:resource}Attributes defined for the \hl{Resource} type.}{
  \begin{tabular}{llllp{8cm}}
    \toprule
    Attribute 	& Type 		& Multiplicity 	& Client Mutability 	& Description \\
    \colrule
    occi.core.summary 	& String 	& 0..1 			& Mutable 				& A summarising description of the \hl{Resource} instance.\\
    links 		& \hl{Link} & 0..* 			& Mutable 				& A set of \hl{Link} compositions. Being a composite relation the removal of a \hl{Link} from the set MUST also remove the \hl{Link} instance.\\
    \botrule
  \end{tabular}
}

The \hl{Resource} type is assigned the \hl{Kind} instance
\textit{http://schemas.ogf.org/occi/core\#resource}, see
table~\ref{tbl:resource_kind}.
%
\mytablefloat{
  \label{tbl:resource_kind}The \hl{Kind} instance assigned to the \hl{Resource} type.}{
  \begin{tabular}{ll}
    \toprule
    Attribute 	& Value \\
    \colrule
    term 		& resource \\
    scheme 		& {\tt http://schemas.ogf.org/occi/core\#} \\
    title 		& Resource \\
    attributes 	& occi.core.summary \\
    actions 	& -- \\
\botrule
\end{tabular}}

\hl{Resource} enforces the inheritance of a set of common attributes
into sub-types. Moreover, it introduces relationships to other
\hl{Resource} instances through instances of the \hl{Link} type.

The \hl{Resource} type is the first of three entry points to extend
the OCCI Core Model, see section~\ref{sec:extensibility}.

\subsubsection{Link}
\label{sec:link}
An instance of the \hl{Link} type defines a base association between
two \hl{Resource} instances. It MUST be implemented. A \hl{Link}
instance indicates that one \hl{Resource} instance is connected to
another.

The \hl{Link} type MUST implement all attributes inherited from the
\hl{Entity} type together with the attributes defined in
table~\ref{tbl:link} in order to be compliant.

\mytablefloat{
  \label{tbl:link}Attributes defined for the \hl{Link} type.}{
  \begin{tabular}{llllp{7.5cm}}
    \toprule
    Attribute 	& Type 			& Multiplicity 	& Client Mutability 	& Description \\
    \colrule
    occi.core.source 		& \hl{Resource} & 1 			& Mutable 				& The \hl{Resource} instances the \hl{Link} instance originates from.\\
    occi.core.target 		& \hl{Resource} & 1 			& Mutable 				& The \hl{Resource} instances the \hl{Link} instance points to.\\
    \botrule
  \end{tabular}
}

The \hl{Link} type is assigned the \hl{Kind} instance
\textit{http://schemas.ogf.org/occi/core\#link}.

\mytablefloat{
  \label{tbl:link_kind}The \hl{Kind} instance assigned to the \hl{Link} type.}{
  \begin{tabular}{ll}
    \toprule
    Attribute 	& Value \\
    \colrule
    term 		& link \\
    scheme 		& {\tt http://schemas.ogf.org/occi/core\#} \\
    title 		& Link \\
    attributes 	& occi.core.source, occi.core.target \\
    actions 	& -- \\
    \botrule
  \end{tabular}
}

The {\tt occi.core.source} and {\tt occi.core.target} attribute of a \hl{Link} instance
MUST refer to resource {\em instances} within the service provider's
namespace. A \hl{Link} MAY refer to an external resource, i.e.~a
resource of which the service provider has no direct control, if and
only if that external resource is mapped into a \hl{Entity} sub-type
instance.

A provider MAY however introduce a sub-type of \hl{Link} with
different semantics, e.g.~having a target attribute containing an URI
and thus the ability of linking with external resources.

The \hl{Link} type is the second of three entry points to extend the
OCCI Core Model, see section~\ref{sec:extensibility}.

\subsubsection{Action}
The \hl{Action} type is an abstract type. Each sub-type of \hl{Action}
defines an invocable operation applicable to an \hl{Entity} sub-type
instance or a collection thereof. It MUST be implemented. In general,
\hl{Action}s modify state by, for example,~performing a complex
operation such as rebooting a virtual machine.

Table~\ref{tbl:action} defines the attributes the \hl{Action} type
MUST implement to be compliant.

\mytablefloat{
  \label{tbl:action}Attributes defined for the \hl{Action} type.}{
  \begin{tabular}{llllp{7.5cm}}
    \toprule
    Attribute 	& Type 			& Multiplicity 	& Client Mutability 	& Description \\
    \colrule
    category 	& \hl{Category} & 1 			& Immutable 			& The identifying \hl{Category} of the \hl{Action}. \\
    \botrule
  \end{tabular}
}

An \hl{Action} MUST always bound to either a \hl{Kind} or a \hl{Mixin}
instance through a composite association. An \hl{Action} is considered
to be a capability of the \hl{Kind} or \hl{Mixin} instance it is
associated with.  An \hl{Action} MAY be invoked on any resource
instance associated with the \hl{Kind} or \hl{Mixin} instance defining
the \hl{Action}. An OCCI implementation MAY however refuse an
\hl{Action} from being invoked if currently not applicable.

An \hl{Action} MAY be invoked on a collection of \hl{Entity} sub-type
instances. The \hl{Action} is only considered valid if all resource
instances of the collection are associated with the \hl{Kind} or
\hl{Mixin} defining the \hl{Action}.

\mytablefloat{
  \label{tbl:action_kind}The \hl{Category} instance assigned to the \hl{Action} type.}{
  \begin{tabular}{ll}
    \toprule
    Attribute 	& Value \\
    \colrule
    term 		& action \\
    scheme 		& {\tt http://schemas.ogf.org/occi/core\#} \\
    title 		& Action \\
    attributes 	& -- \\
    \botrule
  \end{tabular}
}

The \hl{Action} type is assigned the \hl{Category} type identifier
\textit{http://schemas.ogf.org/occi/core\#action}, see
table~\ref{tbl:action_kind}.

An \hl{Action} can expose attributes which correspond to arguments of
the invocable operation.  A sub-type of \hl{Action} define the
attributes available for the invocable operation represented. The
names of any such attributes MUST be exposed through
\hl{Category}.{\tt attributes} of the \hl{Action} sub-type's
identifying \hl{Category} instance.

For example, a ``resize'' \hl{Action} sub-type defined for a storage
resource could have a ``size'' attribute which represent the size
argument of the resize operation. In that example the identifying
\hl{Category} instance would have \hl{Category}.{\tt attributes = [
    size ]}.

The \hl{Action} type is the third and last of the entry points to
extend the OCCI Core Model, see section~\ref{sec:extensibility}. Since
\hl{Action} is an abstract type a sub-type is always necessary to
define a specific \hl{Action}.

\subsection{Extensibility}
\label{sec:extensibility}
The OCCI Core Model has a flexible yet fairly simple extension
mechanism based on the type classification system described in
section~\ref{sec:classification}.

The OCCI Core Model can be extended using two different methods,
sub-typing and mix-in. Custom sub-typing require provider-specific
\hl{Kind} instances and custom mix-ins require provider-specific
\hl{Mixin} instances.  Both methods MAY involve the use of
provider-specific \hl{Category} instance since those are REQUIRED for
provider-specific \hl{Action} sub-types.  The following sections
define the rules for extending the OCCI Core Model.

The rules defined in section~\ref{sec:classification} and
\ref{sec:base_types} are REQUIRED for all extensions of the OCCI Core
Model.

\subsubsection{Category instances}
\label{sec:ext:category}
Provider-specific instances of \hl{Category}, \hl{Kind} and \hl{Mixin}
MAY be introduced by an OCCI implementation. Since \hl{Kind} and
\hl{Mixin} both inherit \hl{Category} the extension rules for
\hl{Category}, defined below, applies to them as well.

A \hl{Category} instance defined outside of the OCCI specification
MUST use a \hl{Category} scheme unique to the provider,
e.g.~\textit{http://example.com/occi\#}. The {\tt term} of a
provider-specific \hl{Category} instance can be any string
corresponding to a ``token'' as defined by the OCCI Rendering documents.

An attribute introduced by a provider-specific \hl{Category} MUST use
an attribute name prefix. This prefix MUST NOT be the
``\texttt{occi.}''~prefix which is reserved for the OCCI
specification. Domain-specific attribute names SHOULD use a prefix
consisting of the provider's reverse domain name,
e.g.~``\texttt{com.example.}''.

\subsubsection{Sub-typing}
The OCCI Core Model MAY be extended through sub-typing.  Three OCCI
Core Model types MAY be sub-typed, those are \hl{Resource}, \hl{Link}
and \hl{Action}.

In order to define a sub-type of \hl{Resource} or \hl{Link} a
provider-specific \hl{Kind} instance MUST be defined and assigned to
the sub-type. This \hl{Kind} instance MUST be directly related to the
\hl{Kind} instance of the type extended.

In order to define a sub-type of \hl{Action} a provider-specific
\hl{Category} instance MUST be assigned to the \hl{Action} sub-type as
its unique type identifier.  Furthermore the \hl{Action} sub-type MUST
be associated as a capability of a provider-specific \hl{Kind} or
\hl{Mixin} instance.

\subsubsection{Mix-ins}
The OCCI Core Model MAY be extended using a ``mix-in'' like concept by
defining provider-specific \hl{Mixin} instances.  A \hl{Mixin}
instance can be associated with any resource instance although a
provider MAY apply restrictions.

In order to support user-defined tags%
\footnote{A tag is a \hl{Mixin} instance, which does not introduce
  additional resource capabilities.}
an OCCI implementation must allow custom \hl{Mixin}
instances to be created and destroyed by request of a client.  There
is no limitation in the OCCI Core Model from doing so but it is
RECOMMENDED to assign a separate \hl{Category} scheme for each user's
\hl{Mixin} instances (e.g. per-user schemes).

\section{Security Considerations}
In the scope of this OCCI specification document the security features
are implemented in the Protocol and Renderings. Therefore please see
the \emph{RESTful HTTP Rendering} Document \cite{occi:http_rendering}.

\section{Glossary}
\label{sec:glossary}
\begin{tabular}{l|p{11cm}}
Term & Description \\
\hline
\hl{Action} & An OCCI base type. Represent an invocable operation on a \hl{Entity} sub-type instance or collection thereof. \\
\hl{Category} & A type in the OCCI model. The parent type of \hl{Kind}. \\
Client & An OCCI client.\\
Collection & A set of \hl{Entity} sub-type instances all associated to a particular \hl{Kind} instance. \\
\hl{Entity} & An OCCI base type. The parent type of \hl{Resource} and \hl{Link}. \\
\hl{Kind} & A type in the OCCI model. The central piece in the OCCI classification system. \\
\hl{Link} & An OCCI base type. A \hl{Link} instance associate one \hl{Resource} instance with another. \\
Mix-in & A non-structural \hl{Kind}. The ``mix-in like'' concept in OCCI only
  support binding of new attributes and \hl{Action}s at run-time. A
  non-structural \hl{Kind} can only be associated with an {\em instance} of an
  \hl{Entity} sub-type. \\
Non-structural \hl{Kind} & An instance of \hl{Kind} {\em not} used as an unique identifier of an OCCI base type. \\
OCCI & Open Cloud Computing Interface \\
OCCI base type & One of \hl{Entity}, \hl{Resource}, \hl{Link} or \hl{Action}. \\
OGF & Open Grid Forum \\
\hl{Resource} & An OCCI base type. The parent type for all domain-specific resource types. \\
Structural \hl{Kind} & An instance of \hl{Kind} assigned as the unique identifier of an OCCI base type. \\
Tag & A non-structural \hl{Kind} with no attributes or actions defined. \\
URI & Uniform Resource Identifier \\
URL & Uniform Resource Locator \\
URN & Uniform Resource Name \\
\end{tabular}


\section{Contributors}

We would like to thank the following people who contributed to this
document:

\begin{tabular}{l|p{2in}|p{2in}}
Name & Affiliation & Contact \\
\hline
Michael Behrens & R2AD & behrens.cloud at r2ad.com \\
Mark Carlson & Oracle & mark.carlson at oracle.com \\
Andy Edmonds & Intel - SLA@SOI project & andy at edmonds.be \\
Sam Johnston & Google & samj at samj.net \\
Gary Mazzaferro & OCCI Counselour - AlloyCloud, Inc. &  garymazzaferro at gmail.com \\ 
Thijs Metsch & Platform Computing, Sun Microsystems & tmetsch at platform.com \\
Ralf Nyrén & Aurenav & ralf at nyren.net \\
Alexander Papaspyrou & TU Dortmund University & alexander.papaspyrou at tu\-dortmund.de \\
Alexis Richardson & RabbitMQ & alexis at rabbitmq.com \\
Shlomo Swidler & Orchestratus & shlomo.swidler at orchestratus.com \\
Florian Feldhaus & GWDG & florian.feldhaus at gwdg.de \\
Jean Parpaillon & & jean.parpaillon at free.fr \\
\end{tabular}

Next to these individual contributions we value the contributions from
the OCCI working group.


\section{Intellectual Property Statement}
The OGF takes no position regarding the validity or scope of any
intellectual property or other rights that might be claimed to pertain
to the implementation or use of the technology described in this
document or the extent to which any license under such rights might or
might not be available; neither does it represent that it has made any
effort to identify any such rights. Copies of claims of rights made
available for publication and any assurances of licenses to be made
available, or the result of an attempt made to obtain a general
license or permission for the use of such proprietary rights by
implementers or users of this specification can be obtained from the
OGF Secretariat.

The OGF invites any interested party to bring to its attention any
copyrights, patents or patent applications, or other proprietary
rights which may cover technology that may be required to practice
this recommendation. Please address the information to the OGF
Executive Director.


\section{Disclaimer}
\input{include/disclaimer}

\section{Full Copyright Notice}
Copyright \copyright ~Open Grid Forum (2009-2016). All Rights Reserved.

This document and translations of it may be copied and furnished to
others, and derivative works that comment on or otherwise explain it
or assist in its implementation may be prepared, copied, published and
distributed, in whole or in part, without restriction of any kind,
provided that the above copyright notice and this paragraph are
included on all such copies and derivative works. However, this
document itself may not be modified in any way, such as by removing
the copyright notice or references to the OGF or other organizations,
except as needed for the purpose of developing Grid Recommendations in
which case the procedures for copyrights defined in the OGF Document
process must be followed, or as required to translate it into
languages other than English.

The limited permissions granted above are perpetual and will not be
revoked by the OGF or its successors or assignees.


\bibliographystyle{IEEEtran}
\bibliography{references}

\end{document}
