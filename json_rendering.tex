\documentclass[10pt,a4paper]{article}
\usepackage[utf8]{inputenc}
\usepackage{fullpage}
\usepackage{graphicx}
\usepackage{fancyhdr}
\usepackage{occi}
\setlength{\headheight}{13pt}
\pagestyle{fancy}

% default sans-serif
\renewcommand{\familydefault}{\sfdefault}

% no lines for headers and footers
\renewcommand{\headrulewidth}{0pt}
\renewcommand{\footrulewidth}{0pt}

% header
\fancyhf{}
\lhead{GWD-R}
\rhead{\today}

% footer
\lfoot{occi-wg@ogf.org}
\rfoot{\thepage}

% paragraphs need some space...
\setlength{\parindent}{0pt}
\setlength{\parskip}{1ex plus 0.5ex minus 0.2ex}

% some space between header and text...
\headsep 13pt

\setcounter{secnumdepth}{4}

\begin{document}

% header on first page is different
\thispagestyle{empty}

GWD-R \hfill Ralf Nyrén \\
OCCI-WG \\
\rightline {September 16, 2011}\\
\rightline {Updated: \today}

\vspace*{0.5in}

\begin{Large}
\textbf{Open Cloud Computing Interface - JSON Rendering}
\end{Large}

\vspace*{0.5in}

\underline{Status of this Document}

This document provides information to the community regarding the
specification of the Open Cloud Computing Interface. Distribution is
unlimited.

\underline{Copyright Notice}

Copyright \copyright Open Grid Forum (2011). All Rights Reserved.

\underline{Trademarks}

OCCI is a trademark of the Open Grid Forum.

\underline{Abstract}

This document, part of a document series, produced by the OCCI working
group within the Open Grid Forum (OGF), provides a high-level
definition of a Protocol and API. The document is based upon
previously gathered requirements and focuses on the scope of important
capabilities required to support modern service offerings.

\underline{Comments}
\newcommand{\ralf}[1]{\textcolor{red}{RN: #1}}

\newpage
\tableofcontents
\newpage

\section{Introduction}
%The Open Cloud Computing Interface (OCCI) is a RESTful Protocol and
API for all kinds of Management tasks. OCCI was originally initiated
to create a remote management API for IaaS%
\footnote{Infrastructure as a Service}
model based Services, allowing for the development of interoperable tools for
common tasks including deployment, autonomic scaling and monitoring.
%
It has since evolved into an flexible API with a strong focus on
interoperability while still offering a high degree of extensibility. The
current release of the Open Cloud Computing Interface is suitable to serve many
other models in addition to IaaS, including e.g.~PaaS and SaaS.

In order to be modular and extensible the current OCCI specification is
released as a suite of complimentary documents which together form the complete
specification.
%
The documents are divided into three categories consisting of the OCCI Core,
the OCCI Renderings and the OCCI Extensions.
%
\begin{itemize}
\item The OCCI Core specification consist of a single document defining the
 OCCI Core Model. The OCCI Core Model can be interacted with {\em
 renderings} (including associated behaviours) and expanded through {\em extensions}.
\item The OCCI Rendering specifications consist of multiple documents each
 describing a particular rendering of the OCCI Core Model. Multiple renderings can
 interact with the same instance of the OCCI Core Model and will automatically support
 any additions to the model which follow the extension rules defined in OCCI
 Core.
\item The OCCI Extension specifications consist of multiple documents each
 describing a particular extension of the OCCI Core Model. The extension documents
 describe additions to the OCCI Core Model defined within the OCCI specification
 suite.
\end{itemize}
%
The current specification consist of three documents.
Future releases of OCCI may include additional rendering and extension
specifications. The documents of the current OCCI specification suite are:

\begin{description}
\item[OCCI Core] describes the formal definition of the the OCCI Core Model
\cite{occi:core}.
\item[OCCI HTTP Rendering] defines how to interact with the OCCI Core Model using the
RESTful OCCI API \cite{occi:http_rendering}. The document defines how the OCCI Core Model can
be communicated and thus serialised using the HTTP protocol.
\item[OCCI Infrastructure] contains the definition of the OCCI Infrastructure
extension for the IaaS domain \cite{occi:infrastructure}. The document defines
additional resource types, their attributes and the actions that can be taken
on each resource type.
\end{description}


\section{Notational Conventions}
{\em todo}

\section{Mandatory HTTP headers}
\label{sec:mandatory_headers}

HTTP headers which MUST be present in every JSON-rendered request and response.

\subsection{Mandatory HTTP request headers}
\begin{description}
\item[Accept:] application/occi+json
\item[Content-Type:] application/occi+json
\end{description}

\subsection{Mandatory HTTP response headers}
\begin{description}
\item[Content-Type:] application/occi+json
\end{description}

\section{JSON representation}
\ralf{This section would describe all the details of the JSON objects used to
represent resource instances and collections. IOW lots of ABNF,
MUST/MUST-NOTs, etc... but for now an example will have to do.}

The JSON representation of a Compute instance. Notice that the ``full''
category representation is used here, i.e.~what you would normally only see in
the discovery interface. It has been included here to show how things are
rendered.
\begin{verbatim}
{
  "kind": {
    "term": "compute",
    "scheme": "http://schemas.ogf.org/occi/infrastructure#",
    "title": "Compute Resource",
    "related": "http://schemas.ogf.org/occi/core#resource",
    "attributes": {
      "occi.compute.architecture": {
        "mutable": true,
        "required": false,
        "type": "string"
      },
      "occi.compute.cores": {
        "mutable": true,
        "required": false,
        "type": "integer"
      },
      "occi.compute.hostname": {
        "mutable": true,
        "required": false,
        "type": "string"
      },
      "occi.compute.speed": {
        "mutable": true,
        "required": false,
        "type": "float"
      },
      "occi.compute.memory": {
        "mutable": true,
        "required": false,
        "type": "float"
      },
      "occi.compute.state": {
        "mutable": false,
        "required": false,
        "type": "string"
      }
    },
    "actions": [
      "http://schemas.ogf.org/occi/infrastructure/compute/action#start",
      "http://schemas.ogf.org/occi/infrastructure/compute/action#stop",
      "http://schemas.ogf.org/occi/infrastructure/compute/action#restart",
      "http://schemas.ogf.org/occi/infrastructure/compute/action#suspend"
    ],
    "location": "/api/compute/"
  },
  "actions": [
    {
      "title": "Start Compute Resource",
      "uri": "/api/compute/1154cb39-ea39-4a3a-a9da-c480968252f4?action=start",
      "type": "http://schemas.ogf.org/occi/infrastructure/compute/action#start"
    }
  ],
  "attributes": {
    "occi.core.id": "1154cb39-ea39-4a3a-a9da-c480968252f4",
    "occi.compute.architecture": "x86_64",
    "occi.compute.hostname": "My Server",
    "occi.compute.speed": 2.67,
    "occi.compute.memory": 1.0,
    "occi.compute.state": "inactive"
  }
}
\end{verbatim}

\section{HTTP methods applied to resource instance URLs}

This section describes the HTTP methods used to retrieve and manipulate
individual resource instances. Each HTTP method described is assumed to operate
on an URL referring to a single element in a collection, an URL such as the
following:
\begin{verbatim}
  http://example.com/compute/012d2b48-c334-47f2-9368-557e75249042
\end{verbatim}

An OCCI client MUST supply the mandatory headers described in
section~\ref{sec:mandatory_headers} with every HTTP request.

\subsection{GET resource instance}
The HTTP GET method retrieves the JSON representation of an OCCI resource
instance.

\subsubsection{Client GET request}
The body of the HTTP GET request MUST be empty.
\begin{verbatim}
GET /compute/012d2b48-c334-47f2-9368-557e75249042 HTTP/1.1
Host: example.com
Accept: application/occi+json
User-Agent: occi-client/x.x OCCI/1.1
\end{verbatim}

\subsubsection{Server GET response}
\begin{verbatim}
HTTP/1.1 200 OK
Server: occi-server/x.x OCCI/1.1
Category: compute;
          scheme="http://schemas.ogf.org/occi/infrastructure#";
          class="kind";
          title="Compute Resource"
Content-Type: application/occi+json; charset=utf-8

{
  "kind": { ... },
  "mixins": [ ... ],
  "actions": [ ... ],
  "links": [ ... ],
  "attributes": { ... },
  "location": "http://example.com/compute/012d2b48-c334-47f2-9368-557e75249042"
}
\end{verbatim}

\subsection{PUT resource instance}
The HTTP PUT method creates or replaces the resource instance at the specified
URL. Since the resource identifier is supplied in the request URL an OCCI
server MAY refuse to create a new instance.

\subsubsection{Client PUT request}
The full JSON representation of the resource instance MUST be supplied in the
HTTP body of the request.

\ralf{Including Links in the request breaks the rule of PUT being idempotent.
Simply prohibit Links in PUT requests?}

\begin{verbatim}
PUT /compute/012d2b48-c334-47f2-9368-557e75249042 HTTP/1.1
Host: example.com
Accept: application/occi+json
User-Agent: occi-client/x.x OCCI/1.1
Content-Type: application/occi+json; charset=utf-8

{
  "kind": { ... },
  "mixins": [ ... ],
  "attributes": { ... },
}
\end{verbatim}

\subsubsection{Server PUT response}
Upon success an OCCI server MUST return HTTP status code 200 and a complete
JSON representation of the created/replaced resource instance. The response MUST
be identical%
\footnote{Provided the resource instance was not changed in the meantime.} of the
to that of a subsequent GET request same URL.

\begin{verbatim}
HTTP/1.1 200 OK
Server: occi-server/x.x OCCI/1.1
Category: compute;
          scheme="http://schemas.ogf.org/occi/infrastructure#";
          class="kind";
          title="Compute Resource"
Content-Type: application/occi+json; charset=utf-8

{
  "kind": { ... },
  "mixins": [ ... ],
  "actions": [ ... ],
  "links": [ ... ],
  "attributes": { ... },
  "location": "http://example.com/compute/012d2b48-c334-47f2-9368-557e75249042"
}
\end{verbatim}

\subsection{POST resource instance with``action'' query parameter}
An OCCI Action is invoked using the HTTP POST method together with query
parameter named ``action''.

\subsubsection{Client POST action request}
\begin{verbatim}
POST /compute/012d2b48-c334-47f2-9368-557e75249042?action=stop HTTP/1.1
Host: example.com
Accept: application/occi+json
User-Agent: occi-client/x.x OCCI/1.1
Content-Type: application/occi+json; charset=utf-8

{
  "action": {
    "term": "stop",
    "scheme": "http://schemas.ogf.org/occi/infrastructure/compute/action#",
  },
  "attributes": {
    "method": "graceful"
  },
}
\end{verbatim}

\subsubsection{Server POST action response}

\begin{verbatim}
HTTP/1.1 204 OK
Server: occi-server/x.x OCCI/1.1
\end{verbatim}

\subsection{POST resource instance without any query parameters}
\ralf{This would imply a partial update of the resource instance. While it is
easy to supply only the attributes to be updated the question is if there are any
valid use cases for partial updates using JSON?}

\subsection{DELETE resource instance}
The HTTP DELETE method destroys a resource instance and any OCCI Links
associated with the resource instance.

\subsubsection{Client DELETE request}
\begin{verbatim}
DELETE /compute/012d2b48-c334-47f2-9368-557e75249042 HTTP/1.1
Host: example.com
Accept: application/occi+json
\end{verbatim}

\subsubsection{Server DELETE response}
\begin{verbatim}
HTTP/1.1 204 OK
Server: occi-server/x.x OCCI/1.1
\end{verbatim}


\section{HTTP methods applied to collections URLs}
This section describes the HTTP methods used to manipulate collections. Each
HTTP method described is assumed to operate on an URL referring to a collection
of elements, an URL such as the following:
\begin{verbatim}
  http://example.com/storage/
\end{verbatim}

A collection consist of a set of resource instances and there are three
different types of collections which may be exposed by an OCCI server.  The
request and response format is identical for all three types collections
although the semantics differ slightly for the PUT and POST methods.
\begin{description}
\item[Kind locations] The location associated with an OCCI Kind instance
represents the collection of all resource instances of that particular Kind.
\item[Mixin locations] The location of an OCCI Mixin instance represents the
collection of all resource instances associated with that Mixin.
\item[Arbitrary path] Any path in the URL namespace which is neither a Kind nor
a Mixin location. A typical example is the root URL e.g.~{\tt
http://example.com/}. Such a path combines all collections in the sub-tree
starting at the path. Therefore the root URL is a collection of all resource
instances available.
\end{description}

\subsection{GET collection}
The HTTP GET method retrieves a list of all resource instances in the
collection. Filtering and pagination information is encoded in the query string
of the URL.

\subsubsection{Client GET request}
The query string of the request URL MUST have the following format:
\begin{verbatim}
query-string        = ""
                    | "?" query-parameter *( "&" query-parameter )
  query-parameter   = attribute-filter
                    | category-filter
                    | pagination-start
                    | pagination-count
  attribute-filter  = "q=" attribute-search *( "+" attribute-search )
  attribute-search  = 1*( string-urlencoded |
                          attribute-name "%3D" string-urlencoded )
  category-filter   = "category=" string-urlencoded
  pagination-start  = "start=" 1*( DIGIT )
  pagination-count  = "count=" 1*( DIGIT )
  attribute-name    = attr-component *( "." attr-component )
  attr-component    = LOALPHA *( LOALPHA | DIGIT | "-" | "_" )
  string-urlencoded = *( ALPHA | DIGIT | "-" | "_" | "." | "~" | "%" )

\end{verbatim}

\paragraph*{Filtering} A search filter can be applied to categories and attributes
of resource instances in a collection. An OCCI server SHOULD support filtering.
The query parameters MUST be URL encoded.

Attribute filters are specified using the {\em q} query parameter.  A filter such
as {\tt q=ubuntu+inactive} would match all resource instances whose combined
set of attribute values includes both the word ``ubuntu'' and ``inactive''. It
is also possible to match on specific attributes by preceding the search term
with the attribute name and an equal sign, for example {\tt
occi.core.title\%3Dubuntu+occi.compute.state\%3Dinactive}. 

The category filter is specified using the {\em category} query parameter and
represent a single Kind, Mixin or Action category to be matched. The following
query would include only resource instances of the Compute type:
{\tt category=http\%3A\%2F\%2Fschemas.ogf.org\%2Focci\%2Finfrastructure\%23compute}

\paragraph*{Pagination} An OCCI client MAY request that the server only return
a subset of a collection. This is accomplished using the {\em start} and
{\em count} query parameters.  An OCCI server MUST support pagination.

The {\em start} parameter specifies the offset into the collection. A value of
zero, {\tt start=0} indicates the beginning of the collection.
%
The {\em count} parameter sets the maximum number of elements to include in the
response. For example {\tt ?start=20\&count=10} would indicate the third page
with a limit of 10 elements per page.

\paragraph*{Example request}

\begin{verbatim}
GET /storage/?q=ubuntu+server&start=0&count=20 HTTP/1.1
Host: example.com
Accept: application/occi+json
User-Agent: occi-client/x.x OCCI/1.1
\end{verbatim}

\subsubsection{Server GET response}
\begin{verbatim}
HTTP/1.1 200 OK
Server: occi-server/x.x OCCI/1.1
Content-Type: application/occi+json; charset=utf-8

{
  "start": 0,
  "count": 3,
  "collection": [
    {
      "kind": { ... },
      "mixins": [ ... ],
      "actions": [ ... ],
      "links": [ ... ],
      "attributes": { ... },
      "location": "http://example.com/storage/e3578467-b6ba-448b-8032-5203278d54db"
    },
    { ... },
    { ... }
  ]
}
\end{verbatim}



\subsection{POST collection}
The HTTP POST method is used to create/update one or more resource instances in
a single atomic request. An OCCI server MUST identify existing resource instances
using the {\tt occi.core.id} attribute.

\subsubsection{Client POST request}
\begin{verbatim}
POST /storage/ HTTP/1.1
Host: example.com
Accept: application/occi+json
User-Agent: occi-client/x.x OCCI/1.1
Content-Type: application/occi+json; charset=utf-8

{
  "collection": [
    {
      "kind": { ... },
      "mixins": [ ... ],
      "links": [ ... ],
      "attributes": { ... },
    },
    { ... },
    { ... }
  ]
}
\end{verbatim}

\subsubsection{Server POST response}
\begin{verbatim}
HTTP/1.1 204 OK
Server: occi-server/x.x OCCI/1.1
\end{verbatim}
\ralf{Should we support HTTP 200 returning the whole collection? Or maybe just the resource instances created/updated?}

\subsection{POST collection with ``action'' query parameter}
{\em todo}

\subsection{PUT collection}
Replace the entire collection with a new one. \ralf{Should we support this?}

\subsection{DELETE collection}
Delete the entire collection. \ralf{Should we support this?}

\section{More examples}
The OCCI demo instance of occi-py%
\footnote{\tt http://github.com/nyren/occi-py}
running at {\tt http://www.nyren.net/api/} has an early version of the draft
JSON rendering available.  Feel free to play around with it. However, please
note the following limitations:
\begin{itemize}
\item The Content-Type is {\tt application/json} and not {\tt application/occi+json}
which would be more appropriate.
\item It does not support request data in JSON.
\item Filtering and pagination is not yet supported.
\end{itemize}

A few example queries using curl:
\begin{verbatim}
curl -i -H 'accept: application/json' http://www.nyren.net/api/-/
curl -i -H 'accept: application/json' http://www.nyren.net/api/link/
curl -i -X POST -H 'accept: application/json' http://www.nyren.net/api/compute/
\end{verbatim}

\end{document}
