\documentclass[10pt,a4paper]{article}
\usepackage[utf8]{inputenc}
\usepackage[english]{babel}
\usepackage[activate={true,nocompatibility},final,tracking=true,kerning=true,spacing=true]{microtype}
\usepackage[plainpages=false,pdfpagelabels,unicode]{hyperref}
\usepackage{fullpage}
\usepackage{graphicx}
\usepackage{fancyhdr}
\usepackage{comment}
\usepackage{occi}
\usepackage{lineno}   % adds line numbers, may be removed for non draft versions
\linenumbers          % adds line numbers, may be removed for non draft versions
\usepackage{verbatim} % adds verbatim options
\usepackage{tabularx} % adds extended tabular formatting options
\usepackage{listings}
\usepackage{color}
\definecolor{lightgray}{rgb}{.9,.9,.9}
\definecolor{darkgray}{rgb}{.4,.4,.4}
\definecolor{purple}{rgb}{0.65, 0.12, 0.82}

\lstdefinelanguage{json}{
  ndkeywords={String, Number, Boolean, Null, Object, Array},
  ndkeywordstyle=\itshape
}
\lstset{
   language=json,
   basicstyle=\footnotesize,
}

\setlength{\headheight}{13pt}
\pagestyle{fancy}

% default sans-serif
\renewcommand{\familydefault}{\sfdefault}

% no lines for headers and footers
\renewcommand{\headrulewidth}{0pt}
\renewcommand{\footrulewidth}{0pt}

% header
\fancyhf{}
\lhead{GWD-R}
\rhead{\today}

% footer
\lfoot{occi-wg@ogf.org}
\rfoot{\thepage}

% paragraphs need some space...
\setlength{\parindent}{0pt}
\setlength{\parskip}{1ex plus 0.5ex minus 0.2ex}

% some space between header and text...
\headsep 13pt

\setcounter{secnumdepth}{4}

\begin{document}

% header on first page is different
\thispagestyle{empty}

Draft \hfill Ralf Nyrén, Independent \\
OCCI-WG \hfill Florian Feldhaus, GWDG\\
\rightline {Boris Parák, CESNET}\\
\rightline {February 25, 2011}\\
\rightline {Updated: \today}

\vspace*{0.5in}

\begin{Large}
\textbf{Open Cloud Computing Interface -- JSON Rendering}
\end{Large}

\vspace*{0.5in}

\underline{Status of this Document}

This document is a \underline{draft} providing information to the community regarding the specification of the Open Cloud Computing Interface.

\underline{Copyright Notice}

Copyright \copyright Open Grid Forum (2012-2015). All Rights Reserved.

\underline{Trademarks}

OCCI is a trademark of the Open Grid Forum.

\underline{Abstract}

This document, part of a document series, produced by the OCCI working
group within the Open Grid Forum (OGF), provides a high-level
definition of a Protocol and API. The document is based upon
previously gathered requirements and focuses on the scope of important
capabilities required to support modern service offerings.

\newpage
\tableofcontents
\newpage

\section{Introduction}
The Open Cloud Computing Interface (OCCI) is a RESTful Protocol and
API for all kinds of management tasks. OCCI was originally initiated
to create a remote management API for IaaS%
\footnote{Infrastructure as a Service}
model-based services, allowing for the development of interoperable tools for
common tasks including deployment, autonomic scaling and monitoring.
%
It has since evolved into a flexible API with a strong focus on
interoperability while still offering a high degree of extensibility. The
current release of the Open Cloud Computing Interface is suitable to serve many
other models in addition to IaaS, including PaaS and SaaS.

In order to be modular and extensible the current OCCI specification is
released as a suite of complimentary documents, which together form the complete
specification.
%
The documents are divided into four categories consisting of the OCCI Core,
the OCCI Protocols, the OCCI Renderings and the OCCI Extensions.
%
\begin{itemize}
\item The OCCI Core specification consists of a single document defining the
 OCCI Core Model. The OCCI Core Model can be interacted with through {\em
 renderings} (including associated behaviors) and expanded through {\em extensions}.
\item The OCCI Protocol specifications consist of multiple documents, each
 describing how the model can be interacted with over a particular protocol (e.g. HTTP, AMQP etc.).
 Multiple protocols can interact with the same instance of the OCCI Core Model.
\item The OCCI Rendering specifications consist of multiple documents, each
 describing a particular rendering of the OCCI Core Model. Multiple renderings can
 interact with the same instance of the OCCI Core Model and will automatically support
 any additions to the model which follow the extension rules defined in OCCI
 Core.
\item The OCCI Extension specifications consist of multiple documents,
  each describing a particular extension of the OCCI Core Model. The
  extension documents describe additions to the OCCI Core Model
  defined within the OCCI specification suite.
\end{itemize}
%

The current specification consists of seven documents. This
specification describes version 1.2 of OCCI and is backward compatible with 1.1.
Future releases of OCCI
may include additional protocol, rendering and extension specifications. The specifications to be
implemented (MUST, SHOULD, MAY) are detailed in the table below.

\mytablefloat{
	\label{tbl:occi_compliancy}%
	What OCCI specifications must be implemented for the specific version.
}
{
	\begin{tabular}{lll}
	\toprule
	Document & OCCI 1.1 & OCCI 1.2 \\
	\colrule
	Core Model & MUST & MUST \\
	Infrastructure Model  & SHOULD & SHOULD \\
	Platform Model & MAY & MAY \\
	SLA Model & MAY & MAY \\
	HTTP Protocol & MUST & MUST \\
	Text Rendering& MUST & MUST \\
	JSON Rendering& MAY & MUST \\
	\botrule
	\end{tabular}
}

% hello


\section{Notational Conventions}
\input{include/notational}

\section{OCCI JSON Rendering}
\label{sec:json_format}
The OCCI JSON Rendering specifies a rendering of OCCI instance types in the JSON
data interchange format as defined in \cite{rfc4627}.

The rendering can be used to render OCCI instances independently of the
protocol being used. Thus messages can be delivered by, e.g., the HTTP
protocol as specified in \cite{occi:protocol}.

The following media-type MUST be used for the OCCI JSON Rendering:

{\tt application/occi+json}

The OCCI JSON Rendering consists of a JSON object containing information on the
OCCI Core instances OCCI Kind, OCCI Mixin, OCCI Action,
OCCI Link and OCCI Resource. The rendering also include a JSON object to invoke
the operation identified by OCCI Actions.
The rendering of each OCCI Core instance will be
described in the following sections.

\subsection{Entity Instance Rendering}
\label{sec:format_entity_instance_rendering}

Entity instances MUST be rendered as JSON hashmaps.

\subsubsection{Resource Instance Rendering}
\label{sec:format_resource}

\todo{Section EXAMPLE RESOURCE is missing.}

The OCCI Resource Instance Rendering consists of a JSON object as shown in the
following example. Section \ref{sec:example_resource} contains a detailed
example.
Table~\ref{tbl:format_resource} defines the object members.
\begin{lstlisting}

        {
            "kind": String,
            "mixins": Array,
            "attributes": Object,
            "actions": Array,
            "id": String,
            "links": Array,
            "summary": String,
            "title": String,
        }


\end{lstlisting}
\mytablefloat{
    \label{tbl:format_resource}
    OCCI Resource instance rendered with the following entries:
    } {
    \begin{tabularx}{\textwidth}{llXll}
    \toprule
    Object member & JSON type & Description & Mutability & Multiplicity \\
    \colrule
    kind & String & Type identifier & immutable & 1 \\

    mixins & Array of Strings & List of type identifiers of associated OCCI
Mixins  &
mutable & 0..1 \\

    attributes & Object & Instance Attributes (see
\ref{sec:format_attribute_description}) & mutable & 0..1 \\

    actions & Array of Strings & List of type identifiers of OCCI
Actions applicable to the OCCI Resource instance & mutable & 0..1 \\

    id & String & ID of the OCCI Resource & immutable & 1\\

    links & Array of Strings & List of URIs of OCCI Links & mutable & 0..1\\
    summary & String & Summary text of resource & mutable & 0..1 \\
    title & String & Title of resource & mutable & 0..1 \\
    \botrule
    \end{tabularx}
}

\paragraph{Action Invocation Rendering}
\label{sec:format_action_invocation}

The OCCI Action Invocation Rendering identifies an invocable operation on a OCCI Resource or
OCCI Link instance. To trigger such an operation the OCCI Action Invocation
Rendering is required.

\todo{Section EXAMPLE ACTION INVOCATION is missing.}

The OCCI Action Invocation Rendering consists of a top-level JSON object as shown in the
following example. Section \ref{sec:example_action_invocation} contains a detailed example.
Table~\ref{tbl:format_action_invocation} defines the object members.

\begin{lstlisting}
{
    "action": String,
    "attributes": Object
}
\end{lstlisting}

\mytablefloat{
    \label{tbl:format_action_invocation}
    An OCCI Action invocation is rendered with
 the following entries:
    } {
    \begin{tabularx}{\textwidth}{llXll}
    \toprule
    Object member & JSON type & Description & Mutability & Multiplicity \\
    \colrule
    action & String & Type identifier & immutable & 1 \\

    attributes & Object & Instance attributes (see
\ref{sec:format_attribute_description}) & mutable & 0..1 \\
    \botrule
    \end{tabularx}
}


\subsubsection{Link Instance Rendering}
\label{sec:format_link}

\todo{Section EXAMPLE LINK is missing.}

The OCCI Link Instance Rendering consists of a JSON object as shown in the
following example. Section \ref{sec:example_link} contains a detailed example.
Table~\ref{tbl:format_link} defines the object members.
\begin{lstlisting}

        {
            "kind": String,
            "mixins": Array,
            "attributes": Object,
            "actions": Array,
            "id": String,
            "source": String,
            "source.type": String,
            "target": String,
            "target.type": String,
            "title": String
        }

\end{lstlisting}

\mytablefloat{
    \label{tbl:format_link}
    OCCI Link instances are rendered with the following entries:
    } {
    \begin{tabularx}{\textwidth}{llXll}
    \toprule
    Object member & JSON type & Description & Mutability & Multiplicity \\
    \colrule
    kind & String & Type identifier & immutable & 1 \\

    mixins & Array of Strings & List of type identifiers of associated OCCI
Mixins & mutable & 0..1 \\

    attributes & Object & Instance attributes (see
\ref{sec:format_attribute_description}) & mutable & 0..1 \\

    actions & Array of Strings & List of type identifiers of OCCI
Action Categories applicable to the OCCI Link instance & mutable & 0..1 \\

    id & String & ID of the OCCI Link & immutable & 1\\

    source & String & URI of the source OCCI Resource. If only one OCCI
Resource is rendered in the same collection, this OCCI Resource is the
source of the OCCI Link if this entry is omitted & immutable & 1\\

    source.type & string & Type identifier of the source Resource. & immutable & 0..1 \\

    target & String & URI of the target Resource & immutable & 1\\

    target.type & string & Type identifier of the target Resource, to be supplied if
the target is an OCCI Resource. & immutable & 0..1 \\

	title & String & title of the Link & mutable & 0..1\\
    \botrule
    \end{tabularx}
}


\subsection{Category Instance Rendering}
\label{sec:format_category_instance_rendering}
Category instances MUST be rendered as JSON hashmaps.

\subsubsection{Kind Instance Rendering}
\label{sec:format_kind}

\todo{Section EXAMPLE KIND is missing.}

The OCCI Kind Instance Rendering consists of a JSON object as shown in the
following example. Section \ref{sec:example_kind} contains a detailed example.
Table~\ref{tbl:format_kind} defines the top-level object members.

\mytablefloat{
    \label{tbl:format_kind}
    OCCI Kind instances are rendered with the following entries:
    } {
    \begin{tabularx}{\textwidth}{llXll}
    \toprule
    Object member & JSON type & Description & Mutability & Multiplicity \\
    \colrule
    term & String & Unique identifier within the categorization scheme &
immutable & 1 \\

    scheme & String & Categorization scheme & immutable & 1 \\

    title & String & Title of the OCCI Kind & immutable & 0..1 \\

    attributes & Object & Attribute description, see
~\ref{tbl:format_attribute_description} & immutable & 0..1 \\

    parent & String & OCCI Kind type identifier of the
related ``parent'' \hl{Kind} instance & immutable & 0..1 \\

    actions & Array of Strings & List of OCCI Action type
identifiers & immutable & 0..1 \\

    location & string & Transport protocol specific URI bound to the OCCI Kind
instance. MUST be supplied for the OCCI Kinds of all OCCI Entities except OCCI
Entity itself & immutable & 0..1 \\
    \botrule
    \end{tabularx}
}

\begin{lstlisting}

        {
            "term": String,
            "scheme": String,
            "title": String,
            "attributes": Object,
            "actions": Array,
            "parent": String,
            "location": String
        }

\end{lstlisting}

\subsubsection{Mixin Instance Rendering}
\label{sec:format_mixin}

\todo{Section EXAMPLE MIXIN is missing.}

The OCCI Mixin Instance Rendering consists of a JSON object as shown in the following example. Section \ref{sec:example_mixin} contains a detailed example.
Table~\ref{tbl:format_mixin} defines the top-level object members.

\begin{lstlisting}

        {
            "term": String,
            "scheme": String,
            "title": String,
            "attributes": Object,
            "actions": Array,
            "depends": Array,
            "applies": Array,
            "location": String
        }

\end{lstlisting}

\mytablefloat{
    \label{tbl:format_mixin}
    OCCI Mixin instances are rendered with the following entries:
    } {
    \begin{tabularx}{\textwidth}{llXll}
    \toprule
    Object member & JSON type & Description & Mutability & Multiplicity \\
    \colrule
    term & String & Unique identifier within the categorization scheme &
immutable & 1 \\

    scheme & String & Categorization scheme & immutable & 1 \\

    title & String & Title of the OCCI Mixin & immutable & 0..1 \\

    attributes & Object & Attribute description, see
~\ref{tbl:format_attribute_description} & immutable & 0..1 \\

    depends & Array of Strings & List of type identifiers of the dependent
 \hl{Mixin} instances & immutable & 0..1 \\

    applies & Array of Strings & List of OCCI Kind type identifiers this OCCI
Mixin can be applied to & immutable & 0..1 \\

    actions & Array of Strings & List of OCCI Action type identifiers
& immutable & 0..1 \\

    location & String & Transport protocol specific URI bound to the OCCI Mixin
instance & immutable & 1 \\
    \botrule
    \end{tabularx}
}

\subsubsection{Action Instance Rendering}
\label{sec:format_action}

The OCCI Action Instance Rendering consists of a JSON object as shown in the
following example.
Table~\ref{tbl:format_action} defines the top-level object members.

\mytablefloat{
    \label{tbl:format_action}
    OCCI Actions are rendered inside the
top-level JSON object with name {\em actions} as an array of JSON Objects with
 the following entries:
    } {
    \begin{tabularx}{\textwidth}{llXll}
    \toprule
    Object member & JSON type & Description & Mutability & Multiplicity \\
    \colrule
    term & String & Unique type identifier within the categorization scheme &
immutable & 1 \\

    scheme & String & Categorization scheme & immutable & 1 \\

    title & String & Title of the OCCI Action & immutable & 0..1 \\

    attributes & Object & Attribute description, see
~\ref{tbl:format_attribute_description} & immutable & 0..1 \\
    \botrule
    \end{tabularx}
}

\begin{lstlisting}

        {
            "term": String,
            "scheme": String,
            "title": String,
            "attributes": Object,
        }

\end{lstlisting}

\subsection{Entity Collection Rendering}
Collections of Entity instances MUST be rendered as JSON arrays. The content of that array is a set of entity instance renderings.

That array MUST be a member of a JSON hashmap that is associated with the relevant key name specific to the type of Entity collection being rendered.

\subsubsection{Resource Collection Rendering}

The JSON hashmap key-name associated with the array of resource instances MUST be \hl{resources}.

\begin{lstlisting}
{
    "resources": []
}

\end{lstlisting}

\subsubsection{Link Collection Rendering}

The JSON hashmap key-name associated with the array of link instances MUST be \hl{links}.

\begin{lstlisting}
{
    "links": []
}
\end{lstlisting}

\subsection{Category Collection Rendering}
Collections of Category instances MUST be rendered as JSON arrays. The content of that array is a set of Category instance renderings.

That array MUST be a member of a JSON hashmap that is associated with the relevant key name specific to the type of Category collection being rendered.


\subsubsection{Kind Collection Rendering}

The JSON hashmap key-name associated with the array of kind instances MUST be \hl{kinds}.

\begin{lstlisting}
{
    "kinds": []
}
\end{lstlisting}

\subsubsection{Mixin Collection Rendering}

The JSON hashmap key-name associated with the array of mixin instances MUST be \hl{mixins}.

\begin{lstlisting}
{
    "mixins": []
}
\end{lstlisting}

\subsubsection{Action Collection Rendering}

The JSON hashmap key-name associated with the array of action instances MUST be \hl{actions}.

\begin{lstlisting}
{
    "actions": []
}
\end{lstlisting}


Collections of Category instances are rendered as JSON arrays.

\subsection{Attributes Rendering}

Attribute names consist of alphanumeric characters separated by dots. The dots
define a logical namespace-like hierarchy. This hierarchy is NOT reflected in JSON
objects. As shown in the following example, the attribute name is an opaque
identifier rendered as hashmap \textit{key}. The hashmap \textit{value} contains either a
Number, a String, a Boolean, an Array or an Object (as an attribute value or an attribute
description, following the Attribute Description Rendering, see \ref{sec:format_attribute_description}).
\begin{lstlisting}
{
    "one.two.three": Number|String|Boolean|Array|Object,
    "one.two.four" : Number|String|Boolean|Array|Object
}
\end{lstlisting}

\subsubsection{Attribute Description Rendering}
\label{sec:format_attribute_description}

Attribute Descriptions are rendered as JSON objects as defined in table~\ref{tbl:format_attribute_description}

\mytablefloat{
    \label{tbl:format_attribute_description}
    All properties of the Attribute definition are optional, but may contain
defaults which MUST be used if the Attribute is not present in the instantiated
OCCI Entity.
    } {
    \begin{tabularx}{\textwidth}{llXll}
    \toprule
    Object member & JSON type & Description & Default \\
    \colrule
    mutable & Boolean & Defines if the Attribute is mutable after initialization
& false \\

    required & Boolean & Defines if the Attribute MUST be specified at
instantiation of the OCCI Entity & false \\

    type & String & Type of the Attribute. MUST be either ``string'', ``number'',
``boolean'', ``array'' or ``object''. & string \\

    pattern & string & POSIX Extended Regular Expression as defined in
\cite{iso9945:2009}. For interoperability reasons, POSIX character classes
 (e.g. [:alpha:]) MUST NOT be used. & .* \\

    default & String, Number, Boolean, Array, Object & Attribute default. MUST be the same
type as defined in the type property and MUST  be used if the Attribute is not
present in the instantiated OCCI Entity & \\

    description & String & Description of the attribute & \\
    \botrule
    \end{tabularx}
}
\begin{lstlisting}
{
    "mutable": Boolean,
    "required": Boolean,
    "type": String,
    "default": String | Number | Boolean | Array | Object,
    "description": String,
    "pattern": String
}
\end{lstlisting}

\section{Security Considerations}
OCCI does not require that an authentication mechanism be used nor
does it require that client to service communications are secured. It
does RECOMMEND that an authentication mechanism be used and that where
appropriate, communications are encrypted using HTTP over TLS. The
authentication mechanisms that MAY be used with OCCI are those that
can be used with HTTP and TLS. For further discussion see the
appropiate section in \cite{occi:protocol}.

\section{Glossary}
\label{sec:glossary}
\todo{update glossary}

\begin{tabular}{l|p{12cm}}
Term & Description \\
\hline
\hl{Action} & An OCCI base type. Represents an invocable operation on a \hl{Entity} sub-type instance or collection thereof. \\

\hl{Attribute} & A type in the OCCI Core Model. Describes the name and properties of attributes found in \hl{Entity} types. \\

\hl{Category} & A type in the OCCI Core Model and the basis of the OCCI type identification mechanism. The parent type of \hl{Kind}. \\

capabilities & In the context of \hl{Entity} sub-types {\bf  capabilities} refer
  to the OCCI \hl{Attribute}s and OCCI \hl{Action}s exposed by an {\bf entity
  instance}. \\

\hl{Client} & An OCCI client.\\

\hl{Collection} & A set of \hl{Entity} sub-type instances all associated to a particular \hl{Kind} or \hl{Mixin} instance. \\

\hl{Entity} & An OCCI base type. The parent type of \hl{Resource} and \hl{Link}. \\

entity instance & An instance of a sub-type of \hl{Entity} but not an instance
  of the \hl{Entity} type itself.  The OCCI model defines two sub-types of
  \hl{Entity}, the \hl{Resource} type and the \hl{Link} type.  However, the
  term {\em entity instance} is defined to include any instance of a
  sub-type of \hl{Resource} or \hl{Link} as well. \\

\hl{Kind} & A type in the OCCI Core Model. A core component of the OCCI classification system. \\

\hl{Link} & An OCCI base type. A \hl{Link} instance associates one \hl{Resource} instance with another. \\

\hl{Mixin} & A type in the OCCI Core Model. A core component of the OCCI classification system. \\

mix-in & An instance of the \hl{Mixin} type associated with an {\em entity
 instance}. The ``mix-in'' concept as used by OCCI {\em only} applies to
 instances, never to \hl{Entity} types. \\

model attribute & An internal attribute of a the Core Model which is {\em not}
  client discoverable. \\

\hl{OCCI} & Open Cloud Computing Interface. \\

OCCI base type & One of \hl{Entity}, \hl{Resource}, \hl{Link} or \hl{Action}. \\

OCCI Action & see \hl{Action}. \\
OCCI Attribute & A client discoverable attribute identified by an instance of the \hl{Attribute} type. Examples are \hl{occi.core.title} and \hl{occi.core.summary}. \\
OCCI Category & see \hl{Category}. \\
OCCI Entity & see \hl{Entity}. \\
OCCI Kind & see \hl{Kind}. \\
OCCI Link & see \hl{Link}. \\
OCCI Mixin & see \hl{Mixin}. \\

OGF & Open Grid Forum. \\

\hl{Resource} & An OCCI base type. The parent type for all domain-specific \hl{Resource} sub-types. \\

resource instance & See {\em entity instance}. This term is considered obsolete. \\

tag & A \hl{Mixin} instance with no attributes or actions defined. \\

template & A \hl{Mixin} instance which if associated at instance
creation-time pre-populate certain attributes. \\

type & One of the types defined by the OCCI Core Model.  The Core Model types are
 \hl{Category}, \hl{Attribute},
 \hl{Kind}, \hl{Mixin}, \hl{Action}, \hl{Entity}, \hl{Resource}
 and \hl{Link}. \\

concrete type/sub-type & A concrete type/sub-type is a type that can be instantiated.\\

URI & Uniform Resource Identifier. \\
URL & Uniform Resource Locator. \\
URN & Uniform Resource Name. \\
\end{tabular}


\section{Contributors}

We would like to thank the following people who contributed to this
document:

\begin{tabular}{l|p{2in}|p{2in}}
Name & Affiliation & Contact \\
\hline
Michael Behrens & R2AD & behrens.cloud at r2ad.com \\
Andy Edmonds & Intel - SLA@SOI project & andy at edmonds.be \\
Sam Johnston & Google & samj at samj.net \\
Gary Mazzaferro & OCCI Counselour - Exxia, Inc. &  garymazzaferro at gmail.com \\ 
Thijs Metsch & Platform Computing, Sun Microsystems & tmetsch at platform.com \\
Ralf Nyrén & Aurenav & ralf at nyren.net \\
Alexander Papaspyrou & TU-Dortmund & alexander.papaspyour at tu\-dortmund.de \\
Shlomo Swidler & Orchestratus & shlomo.swidler at orchestratus.com \\
\end{tabular}

Next to these individual contributions we value the contributions from
the OCCI working group.



% FIXME: Insert an up-to-date table here!

\section{Intellectual Property Statement}
\input{include/ip}

\section{Disclaimer}
This document and the information contained herein is provided on an
``As Is'' basis and the OGF disclaims all warranties, express or
implied, including but not limited to any warranty that the use of the
information herein will not infringe any rights or any implied
warranties of merchantability or fitness for a particular purpose.


\section{Full Copyright Notice}
Copyright \copyright ~Open Grid Forum (2009-2016). All Rights Reserved.

This document and translations of it may be copied and furnished to
others, and derivative works that comment on or otherwise explain it
or assist in its implementation may be prepared, copied, published and 
distributed, in whole or in part, without restriction of any kind,
provided that the above copyright notice and this paragraph are
included as references to the derived portions on all such copies
and derivative works. The published OGF document from which such works
are derived, however, may not be modified in any way, such as by removing
the copyright notice or references to the OGF or other organizations,
except as needed for the purpose of developing new or updated OGF documents
in conformance with the procedures defined in the OGF Document Process,
or as required to translate it into languages other than English. OGF,
with the approval of its board, may remove this restriction for inclusion
of OGF document content for the purpose of producing standards in cooperation
with other international standards bodies. 

The limited permissions granted above are perpetual and will not be
revoked by the OGF or its successors or assignees.


\bibliographystyle{IEEEtran}
\bibliography{references}

\end{document}
