\documentclass[10pt,a4paper]{article}
\usepackage[utf8]{inputenc}
\usepackage{fullpage}
\usepackage{graphicx}
\usepackage{fancyhdr}
\usepackage{occi}
\setlength{\headheight}{13pt}
\pagestyle{fancy}

% default sans-serif
\renewcommand{\familydefault}{\sfdefault}

% no lines for headers and footers
\renewcommand{\headrulewidth}{0pt}
\renewcommand{\footrulewidth}{0pt}

% header
\fancyhf{}
\lhead{GFD-P-R.185}
\rhead{\today}

% footer
\lfoot{occi-wg@ogf.org}
\rfoot{\thepage}

% paragraphs need some space...
\setlength{\parindent}{0pt}
\setlength{\parskip}{1ex plus 0.5ex minus 0.2ex}

% some space between header and text...
\headsep 13pt

\setcounter{secnumdepth}{4}

\begin{document}

% header on first page is different
\thispagestyle{empty}

GFD-P-R.185 \hfill Thijs Metsch, Platform Computing\\
OCCI-WG \hfill Andy Edmonds, Intel\\
\rightline {October 7, 2010}\\
\rightline {Updated: \today}

\vspace*{0.5in}

\begin{Large}
\textbf{Open Cloud Computing Interface - RESTful HTTP Rendering}
\end{Large}

\vspace*{0.5in}

\underline{Status of this Document}

This document provides information to the community regarding the
specification of the Open Cloud Computing Interface. Distribution is
unlimited.

\underline{Copyright Notice}

Copyright \copyright ~Open Grid Forum (2009-2011). All Rights Reserved.

\underline{Trademarks}

OCCI is a trademark of the Open Grid Forum.

\underline{Abstract}

This document, part of a document series, produced by the OCCI working
group within the Open Grid Forum (OGF), provides a high-level
definition of a Protocol and API. The document is based upon
previously gathered requirements and focuses on the scope of important
capabilities required to support modern service offerings.

\newpage
\tableofcontents
\newpage

\section{Introduction}
The Open Cloud Computing Interface (OCCI) is a RESTful Protocol and
API for all kinds of management tasks. OCCI was originally initiated
to create a remote management API for IaaS%
\footnote{Infrastructure as a Service}
model-based services, allowing for the development of interoperable tools for
common tasks including deployment, autonomic scaling and monitoring.
%
It has since evolved into a flexible API with a strong focus on
interoperability while still offering a high degree of extensibility. The
current release of the Open Cloud Computing Interface is suitable to serve many
other models in addition to IaaS, including PaaS and SaaS.

In order to be modular and extensible the current OCCI specification is
released as a suite of complimentary documents, which together form the complete
specification.
%
The documents are divided into four categories consisting of the OCCI Core,
the OCCI Protocols, the OCCI Renderings and the OCCI Extensions.
%
\begin{itemize}
\item The OCCI Core specification consists of a single document defining the
 OCCI Core Model. The OCCI Core Model can be interacted with through {\em
 renderings} (including associated behaviors) and expanded through {\em extensions}.
\item The OCCI Protocol specifications consist of multiple documents, each
 describing how the model can be interacted with over a particular protocol (e.g. HTTP, AMQP etc.).
 Multiple protocols can interact with the same instance of the OCCI Core Model.
\item The OCCI Rendering specifications consist of multiple documents, each
 describing a particular rendering of the OCCI Core Model. Multiple renderings can
 interact with the same instance of the OCCI Core Model and will automatically support
 any additions to the model which follow the extension rules defined in OCCI
 Core.
\item The OCCI Extension specifications consist of multiple documents,
  each describing a particular extension of the OCCI Core Model. The
  extension documents describe additions to the OCCI Core Model
  defined within the OCCI specification suite.
\end{itemize}
%

The current specification consists of seven documents. This
specification describes version 1.2 of OCCI and is backward compatible with 1.1.
Future releases of OCCI
may include additional protocol, rendering and extension specifications. The specifications to be
implemented (MUST, SHOULD, MAY) are detailed in the table below.

\mytablefloat{
	\label{tbl:occi_compliancy}%
	What OCCI specifications must be implemented for the specific version.
}
{
	\begin{tabular}{lll}
	\toprule
	Document & OCCI 1.1 & OCCI 1.2 \\
	\colrule
	Core Model & MUST & MUST \\
	Infrastructure Model  & SHOULD & SHOULD \\
	Platform Model & MAY & MAY \\
	SLA Model & MAY & MAY \\
	HTTP Protocol & MUST & MUST \\
	Text Rendering& MUST & MUST \\
	JSON Rendering& MAY & MUST \\
	\botrule
	\end{tabular}
}

% hello


\section{Notational Conventions}
\input{include/notational}

\todo{update rfc}

This document uses the Augmented Backus-Naur Form (ABNF) notation of
RFC 2616 \cite{rfc2616}, and explicitly includes the following rules
from it: quoted-string, token, SP (space), LOALPHA, DIGIT.

The following terms \cite{rfc3986} are used when referring to URI
components:

\begin{verbatim}
 http://example.com:8080/over/there?action=stop#xyz
 \__/   \______________/\_________/ \_________/ \_/
  |           |            |            |        |
scheme     authority       path        query   fragment
\end{verbatim}

\section{OCCI HTTP Rendering}
The OCCI HTTP Rendering document specifies how the OCCI Core Model
\cite{occi:core}, including extensions thereof, is rendered over the
HTTP protocol \cite{rfc2616}. 

\subsection{Introduction}
The OCCI HTTP Rendering uses many of the features the HTTP and
underlying protocols offer and builds upon the Resource Oriented
Architecture (ROA). ROA's use Representation State Transfer (REST)
\cite{REST_Fielding} to cater for client and service
interactions. Interaction with the system is by inspection and
modification of a set of related resources and their states, be it on
the complete state or a sub-set. Resources MUST be uniquely
identified. HTTP is an ideal protocol to use in ROA systems as it
provides the means to uniquely identify individual resources through
URLs as well as operating upon them with a set of general-purpose
methods known as HTTP verbs. These HTTP verbs map loosely to the
resource related operations of Create (POST), Retrieve (GET), Update
(POST/PUT) and Delete (DELETE).

Each resource instance within an OCCI system MUST have a unique
identifier stored in the \emph{occi.core.id} attribute of the
\hl{Entity} type \cite{occi:core}.
It is RECOMMENDED to use a Uniform Resource Name (URN) as the
identifier stored in \emph{occi.core.id}.

The structure of these identifiers is opaque and the system should not
assume a static, pre-determined scheme for their structure. For
example \emph{occi.core.id} could be
\emph{urn:uuid:de7335a7-07e0-4487-9cbd-ed51be7f2ce4}.

The following sections and paragraphs describe how the OCCI model MUST
be implemented by OCCI implementations. Operations which are not
defined are out of scope for this specification and MAY be implemented. 
This is the minimal set to ensure interoperability.

\subsection{Namespace}

The OCCI HTTP Protocol maps the OCCI Core model into the URL hierarchy by binding
\hl{Kind} and \hl{Mixin} instances to unique URL paths. Such a URL path is called
the {\em location} of the \hl{Kind} or \hl{Mixin}.
A provider is free to choose the {\em location} as long as it is unique
within the service provider's URL namespace. 

Location paths are defined through specific renderings and MUST be present for all \hl{Kind} or \hl{Mixin} instances. The location
paths MUST end with a \emph{'/'}. The {\em location} MAY be realized by applying URI templates defined by \cite{rfc6570}.

A \hl{Kind} instance whose associated type cannot be instantiated MUST NOT be
bound to an URL path. This applies to the \hl{Kind} instance for OCCI Entity
which, according to OCCI Core, cannot be instantiated \cite{occi:core}.

\subsubsection{unbound/bound}

Since a limited set of URL paths are bound to \hl{Kind} and \hl{Mixin}
instances the URL hierarchy consists of both {\em bound} and {\em unbound}
paths. A bound URL path is the {\em location} of a \hl{Kind} or \hl{Mixin} collection.

An unbound URL path MAY represent the union of all \hl{Kind} and \hl{Mixin}
collection ``below'' the unbound path.

\subsection{HTTP}

\subsubsection{General Status codes}

\subsubsection{Versioning}

\subsubsection{Media types}

\paragraph{text uri-list (rendering)}

\subsubsection{Well known QI}

\subsubsection{Async op}

\subsubsection{Import Export}

\begin{itemize}
\item realize as Category
\item HTTP POST user defined mixin
\item HTTP POST against that collection with Content-Type…
\item HTTP GET against that collection with Accept...
\end{itemize}

\subsection{Operations}

\subsubsection{Entity instances}
POST
partial-update
action
PUT
create
full-update
GET
retrieval
DELETE
delete
Link instances
POST
inline creation 
\subsubsection{Collections}
POST
create entity instance
add entity to mixin
action
PUT
full-update on mixins...not kinds.
full-update on kind collection -> not defined
GET + filter/query?
retrieval
DELETE
delete all or single (!) entity
remove entity of mixins 
\subsubsection{QI}
POST
user defined mixin
PUT
N/A
GET + filter/query!
retrieve capabilities
DELETE
Removal user defined mixin

\section{Security Considerations}
\label{sec:sec_consid}
The OCCI HTTP rendering assumes HTTP or HTTP-related mechanisms for
security. As such, implementations SHOULD support
TLS \footnote{http://datatracker.ietf.org/wg/tls/} for transport layer
security.

Authentication SHOULD be realized by HTTP authentication mechanisms,
namely HTTP Basic or Digest Auth \cite{rfc2617}, with the former as
default. Additional profiles MAY specify other methods and should
ensure that the selected authentication scheme can be renderable over
the HTTP or HTTP-related protocols.

Authorization is not enforced on the protocol level, but SHOULD be
performed by the implementation. For the authorization decision, the
authentication information as provided by the mechanisms described
above MUST be used.

Protection against potential Denial-of-Service scenarios are out of
scope of this document; the OCCI HTTP Rendering specifications assumes
cooperative clients that SHOULD use selection and filtering as
provided by the Category mechanism wherever possible. Additional
profiles to this document, however, MAY specifically address such
scenarios; in that case, best practices from the HTTP ecosystem and
appropriate mechanisms as part of the HTTP protocol specification
SHOULD be preferred.

As long as specific extensions of the OCCI Core and Model
specification do not impose additional security requirements than the
OCCI Core and Model specification itself, the security considerations
documented above apply to all (existing and future)
extensions. Otherwise, an additional profile to this specification
MUST be provided; this profile MUST express all additional security
considerations using HTTP mechanisms.

\section{Glossary}
\label{sec:glossary}
\todo{update glossary}

\begin{tabular}{l|p{12cm}}
Term & Description \\
\hline
\hl{Action} & An OCCI base type. Represents an invocable operation on a \hl{Entity} sub-type instance or collection thereof. \\

\hl{Attribute} & A type in the OCCI Core Model. Describes the name and properties of attributes found in \hl{Entity} types. \\

\hl{Category} & A type in the OCCI Core Model and the basis of the OCCI type identification mechanism. The parent type of \hl{Kind}. \\

capabilities & In the context of \hl{Entity} sub-types {\bf  capabilities} refer
  to the OCCI \hl{Attribute}s and OCCI \hl{Action}s exposed by an {\bf entity
  instance}. \\

\hl{Client} & An OCCI client.\\

\hl{Collection} & A set of \hl{Entity} sub-type instances all associated to a particular \hl{Kind} or \hl{Mixin} instance. \\

\hl{Entity} & An OCCI base type. The parent type of \hl{Resource} and \hl{Link}. \\

entity instance & An instance of a sub-type of \hl{Entity} but not an instance
  of the \hl{Entity} type itself.  The OCCI model defines two sub-types of
  \hl{Entity}, the \hl{Resource} type and the \hl{Link} type.  However, the
  term {\em entity instance} is defined to include any instance of a
  sub-type of \hl{Resource} or \hl{Link} as well. \\

\hl{Kind} & A type in the OCCI Core Model. A core component of the OCCI classification system. \\

\hl{Link} & An OCCI base type. A \hl{Link} instance associates one \hl{Resource} instance with another. \\

\hl{Mixin} & A type in the OCCI Core Model. A core component of the OCCI classification system. \\

mix-in & An instance of the \hl{Mixin} type associated with an {\em entity
 instance}. The ``mix-in'' concept as used by OCCI {\em only} applies to
 instances, never to \hl{Entity} types. \\

model attribute & An internal attribute of a the Core Model which is {\em not}
  client discoverable. \\

\hl{OCCI} & Open Cloud Computing Interface. \\

OCCI base type & One of \hl{Entity}, \hl{Resource}, \hl{Link} or \hl{Action}. \\

OCCI Action & see \hl{Action}. \\
OCCI Attribute & A client discoverable attribute identified by an instance of the \hl{Attribute} type. Examples are \hl{occi.core.title} and \hl{occi.core.summary}. \\
OCCI Category & see \hl{Category}. \\
OCCI Entity & see \hl{Entity}. \\
OCCI Kind & see \hl{Kind}. \\
OCCI Link & see \hl{Link}. \\
OCCI Mixin & see \hl{Mixin}. \\

OGF & Open Grid Forum. \\

\hl{Resource} & An OCCI base type. The parent type for all domain-specific \hl{Resource} sub-types. \\

resource instance & See {\em entity instance}. This term is considered obsolete. \\

tag & A \hl{Mixin} instance with no attributes or actions defined. \\

template & A \hl{Mixin} instance which if associated at instance
creation-time pre-populate certain attributes. \\

type & One of the types defined by the OCCI Core Model.  The Core Model types are
 \hl{Category}, \hl{Attribute},
 \hl{Kind}, \hl{Mixin}, \hl{Action}, \hl{Entity}, \hl{Resource}
 and \hl{Link}. \\

concrete type/sub-type & A concrete type/sub-type is a type that can be instantiated.\\

URI & Uniform Resource Identifier. \\
URL & Uniform Resource Locator. \\
URN & Uniform Resource Name. \\
\end{tabular}

 
\section{Contributors}

We would like to thank the following people who contributed to this
document:

\begin{tabular}{l|p{2in}|p{2in}}
Name & Affiliation & Contact \\
\hline
Michael Behrens & R2AD & behrens.cloud at r2ad.com \\
Andy Edmonds & Intel - SLA@SOI project & andy at edmonds.be \\
Sam Johnston & Google & samj at samj.net \\
Gary Mazzaferro & OCCI Counselour - Exxia, Inc. &  garymazzaferro at gmail.com \\ 
Thijs Metsch & Platform Computing, Sun Microsystems & tmetsch at platform.com \\
Ralf Nyrén & Aurenav & ralf at nyren.net \\
Alexander Papaspyrou & TU-Dortmund & alexander.papaspyour at tu\-dortmund.de \\
Shlomo Swidler & Orchestratus & shlomo.swidler at orchestratus.com \\
\end{tabular}

Next to these individual contributions we value the contributions from
the OCCI working group.




\section{Intellectual Property Statement}
\input{include/ip}

\section{Disclaimer}
This document and the information contained herein is provided on an
``As Is'' basis and the OGF disclaims all warranties, express or
implied, including but not limited to any warranty that the use of the
information herein will not infringe any rights or any implied
warranties of merchantability or fitness for a particular purpose.


\section{Full Copyright Notice}
Copyright \copyright ~Open Grid Forum (2009-2016). All Rights Reserved.

This document and translations of it may be copied and furnished to
others, and derivative works that comment on or otherwise explain it
or assist in its implementation may be prepared, copied, published and 
distributed, in whole or in part, without restriction of any kind,
provided that the above copyright notice and this paragraph are
included as references to the derived portions on all such copies
and derivative works. The published OGF document from which such works
are derived, however, may not be modified in any way, such as by removing
the copyright notice or references to the OGF or other organizations,
except as needed for the purpose of developing new or updated OGF documents
in conformance with the procedures defined in the OGF Document Process,
or as required to translate it into languages other than English. OGF,
with the approval of its board, may remove this restriction for inclusion
of OGF document content for the purpose of producing standards in cooperation
with other international standards bodies. 

The limited permissions granted above are perpetual and will not be
revoked by the OGF or its successors or assignees.


\bibliographystyle{IEEEtran}
\bibliography{references}

\end{document}