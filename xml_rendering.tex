\documentclass[10pt,a4paper]{article}
\usepackage[utf8]{inputenc}
\usepackage{fullpage}
\usepackage{graphicx}
\usepackage{fancyhdr}
\usepackage{comment}
\usepackage{occi}
%\usepackage{lineno}   % adds line numbers, may be removed for non draft versions
%\linenumbers          % adds line numbers, may be removed for non draft versions
\usepackage{verbatim} % adds verbatim options
\usepackage{tabularx} % adds extended tabular formatting options
\usepackage{listings}
\usepackage{color}
\definecolor{lightgray}{rgb}{.9,.9,.9}
\definecolor{darkgray}{rgb}{.4,.4,.4}
\definecolor{purple}{rgb}{0.65, 0.12, 0.82}

\lstdefinelanguage{json}{
  ndkeywords={String, Number, Boolean, Null, Object, Array},
  ndkeywordstyle=\itshape
}
\lstset{
   language=json,
   basicstyle=\footnotesize,
}

\setlength{\headheight}{13pt}
\pagestyle{fancy}

% default sans-serif
\renewcommand{\familydefault}{\sfdefault}

% no lines for headers and footers
\renewcommand{\headrulewidth}{0pt}
\renewcommand{\footrulewidth}{0pt}

% header
\fancyhf{}
\lhead{GWD-R}
\rhead{\today}

% footer
\lfoot{occi-wg@ogf.org}
\rfoot{\thepage}

% paragraphs need some space...
\setlength{\parindent}{0pt}
\setlength{\parskip}{1ex plus 0.5ex minus 0.2ex}

% some space between header and text...
\headsep 13pt

\setcounter{secnumdepth}{4}

\begin{document}

% header on first page is different
\thispagestyle{empty}

GWD-R \hfill Jean Parpaillon\\
OCCI-WG \\
\rightline {September 26, 2013}\\
\rightline {Updated: \today}

\vspace*{0.5in}

\begin{Large}
\textbf{Open Cloud Computing Interface - XML Rendering}
\end{Large}

\vspace*{0.5in}

\underline{Status of this Document}

This document provides information to the community regarding the
specification of the Open Cloud Computing Interface. Distribution is
unlimited.

\underline{Copyright Notice}

Copyright \copyright Open Grid Forum (2013). All Rights Reserved.

\underline{Trademarks}

OCCI is a trademark of the Open Grid Forum.

\underline{Abstract}

This document, part of a document series, produced by the OCCI working
group within the Open Grid Forum (OGF), provides a high-level
definition of a Protocol and API. The document is based upon
previously gathered requirements and focuses on the scope of important
capabilities required to support modern service offerings.

\underline{Comments}
\newcommand{\ralf}[1]{\textcolor{red}{RN: #1}}
\newcommand{\andy}[1]{\textcolor{green}{AE: #1}}
\newcommand{\florian}[1]{\textcolor{blue}{FF: #1}}
\newcommand{\jean}[1]{\textcolor{purple}{JP: #1}}

\newpage
\tableofcontents
\newpage

\section{Introduction}
The Open Cloud Computing Interface (OCCI) is a RESTful Protocol and
API for all kinds of Management tasks. OCCI was originally initiated
to create a remote management API for IaaS%
\footnote{Infrastructure as a Service}
model based Services, allowing for the development of interoperable tools for
common tasks including deployment, autonomic scaling and monitoring.
%
It has since evolved into an flexible API with a strong focus on
interoperability while still offering a high degree of extensibility. The
current release of the Open Cloud Computing Interface is suitable to serve many
other models in addition to IaaS, including e.g.~PaaS and SaaS.

In order to be modular and extensible the current OCCI specification is
released as a suite of complimentary documents which together form the complete
specification.
%
The documents are divided into three categories consisting of the OCCI Core,
the OCCI Renderings and the OCCI Extensions.
%
\begin{itemize}
\item The OCCI Core specification consist of a single document defining the
 OCCI Core Model. The OCCI Core Model can be interacted with {\em
 renderings} (including associated behaviours) and expanded through {\em extensions}.
\item The OCCI Rendering specifications consist of multiple documents each
 describing a particular rendering of the OCCI Core Model. Multiple renderings can
 interact with the same instance of the OCCI Core Model and will automatically support
 any additions to the model which follow the extension rules defined in OCCI
 Core.
\item The OCCI Extension specifications consist of multiple documents each
 describing a particular extension of the OCCI Core Model. The extension documents
 describe additions to the OCCI Core Model defined within the OCCI specification
 suite.
\end{itemize}
%
The current specification consist of three documents.
Future releases of OCCI may include additional rendering and extension
specifications. The documents of the current OCCI specification suite are:

\begin{description}
\item[OCCI Core] describes the formal definition of the the OCCI Core Model
\cite{occi:core}.
\item[OCCI HTTP Rendering] defines how to interact with the OCCI Core Model using the
RESTful OCCI API \cite{occi:http_rendering}. The document defines how the OCCI Core Model can
be communicated and thus serialised using the HTTP protocol.
\item[OCCI Infrastructure] contains the definition of the OCCI Infrastructure
extension for the IaaS domain \cite{occi:infrastructure}. The document defines
additional resource types, their attributes and the actions that can be taken
on each resource type.
\end{description}


\section{Notational Conventions}
All these parts and the information within are mandatory for
implementors (unless otherwise specified). The key words "MUST", "MUST
NOT", "REQUIRED", "SHALL", "SHALL NOT", "SHOULD", "SHOULD NOT",
"RECOMMENDED", "MAY", and "OPTIONAL" in this document are to be
interpreted as described in RFC 2119 \cite{rfc2119}.


\section{OCCI XML Rendering}

The OCCI XML Rendering specifies a rendering of OCCI instance types in
the eXtensible Markup Language (XML), as defined in \cite{w3c:xml11}.

The Rendering can be used to render OCCI instances independently of the
transport mechanism being used. Thus messages can be delivered by e.g. the HTTP
protocol as specified in \cite{occi:http_rendering} or by using text files with
the .xml file extension as defined in \cite{w3c:xml11}.

The following media-type MUST be used for the OCCI XML Rendering:

{\tt application/occi+xml}

\section{XML Format}
\label{sec:xml_format}

The OCCI XML Rendering consists of a sequence of XML element containing
information on the OCCI Core instances OCCI Kind, OCCI Mixin, OCCI
Action, OCCI Link and OCCI Resource. The rendering also include an XML
element to invoke the operation identified by OCCI Actions.
The rendering of each OCCI Core instance will be described in the
following sections.

\jean{Planned, not present in XML schema}

OCCI XML rendering include XML element to represent the
CRUD\footnote{Create, Retrieve, Update, Delete} actions on OCCI types.

\subsection{Resource Instance Format}
\label{sec:format_resource}

The OCCI Resource Instance Format consists of a XML object as shown in the
following example. Section \ref{sec:example_resource} contains a detailed
example.
Table~\ref{tbl:format_resource} defines the object members.
%FIXME : attribute? id title href
\begin{lstlisting}
<xs:complexType name="resourceType">
	<xs:complexContent>
		<xs:extension base="occi:entityType">
			<xs:sequence>
				<xs:element name="summary" type="xs:string" minOccurs="0" maxOccurs="1" />
				<xs:element name="link" type="occi:resourceLinkType" minOccurs="0" maxOccurs="unbounded" />
			</xs:sequence>
		</xs:extension>
	</xs:complexContent>
</xs:complexType>

Example:
<resources id="xs:anyURI" title="xs:string" href="xl:href">
	<kind type="occi:categoryType"/>
	<mixins type="occi:categoryType"/>
        <attributes type="occi:attributeType"/>
        <actions: type="occi:attributeType"/>
        <links type="occi:resourceLinkType"> 
</resources>
\end{lstlisting}

\mytablefloat{
    \label{tbl:format_resource}
    OCCI Resource instances are rendered inside the top-level XML object with
name
{\em resources}
    as a xs:complexType of XML objects with the following entries:
    } {
    \begin{tabularx}{\textwidth}{llXll}
    \toprule
    Object member & XML type & Description & Mutability & Multiplicity \\
    \colrule
    kind & occi:categoryType & Type identifier & immutable & 1 \\

    mixins & occi:categoryType & List of type identifiers of associated OCCI
Mixins  &
mutable & 0..* \\

    attributes & occi:attributeType & Instance Attributes (see
\ref{sec:format_attribute_description}) & mutable & 0..* \\
    
    actions & complexType & List of type identifiers of OCCI
Actions applicable to the OCCI Resource instance & mutable & 0..* \\
    
    id & xs:anyURI & ID of the OCCI Resource & immutable & 1\\

	summary & xs:string &  & immutable & 1\\
            
    links & occi:resourceLinkType & List of URIs of OCCI Links & mutable & 0..*\\
    \botrule
    \end{tabularx}
}

\subsection{Link Instance Format}
\label{sec:format_link}

The OCCI Link Instance Format consists of a XML object as shown in the
following example. Section \ref{sec:example_link} contains a detailed example.
Table~\ref{tbl:format_link} defines the object members.
\begin{lstlisting}
<xs:complexType name="linkType">
	<xs:complexContent>
		<xs:extension base="occi:entityType" />
	</xs:complexContent>
</xs:complexType>

Example:
<link id="xs:anyURI" title="xs:string" href="xl:href">
	<kind type="occi:categoryType"/>
	<mixins type="occi:categoryType"/>
        <attributes type="occi:attributeType"/>
        <actions: type="occi:attributeType"/>
        <links type="occi:resourceLinkType">
<link>
\end{lstlisting}

\mytablefloat{
    \label{tbl:format_link}
    OCCI Resource instances are rendered inside the top-level XML object with
name
{\em resources}
    as a xs:compleType of XML objects with the following entries:
    } {
    \begin{tabularx}{\textwidth}{llXll}
    \toprule
    Object member & XML type & Description & Mutability & Multiplicity \\
    \colrule
    kind & occi:categoryType & Type identifier & immutable & 1 \\

    mixins & occi:categoryType & List of type identifiers of associated OCCI
Mixins  &
mutable & 0..* \\

    attributes & occi:attributeType & Instance Attributes (see
\ref{sec:format_attribute_description}) & mutable & 0..* \\
    
    actions & complexType & List of type identifiers of OCCI
Actions applicable to the OCCI Resource instance & mutable & 0..* \\
    
    id & xs:anyURI & ID of the OCCI Resource & immutable & 1\\
            
    links & occi:resourceLinkType & List of URIs of OCCI Links & mutable & 0..*\\
    \botrule
    \end{tabularx}
}

\subsection{Action Invocation Format}
\label{sec:format_action_invocation}

The OCCI Action Invocation  Format identifies an invocable operation on a OCCI Resource or
OCCI Link instance. To trigger such an operation the OCCI Action Invocation
Format is required.

The OCCI Action Invocation Format consists of a top-level XML object as shown in the
following example. Section \ref{sec:example_action_invocation} contains a detailed example.
Table~\ref{tbl:format_action_invocation} defines the object members.
%FIXME : categoryGroup?
\begin{lstlisting}
<xs:complexType name="actionType">
	<xs:sequence>
		<xs:element name="attribute" type="occi:attributeType"></xs:element>
	</xs:sequence>
	<xs:attributeGroup ref="occi:categoryGroup"></xs:attributeGroup>
</xs:complexType>

Example:
<action>
	<attributes type="occi:attributeType">
</action>
\end{lstlisting}

\mytablefloat{
    \label{tbl:format_action_invocation}
    An OCCI Action invocation is rendered as top-level XML object with
 the following entries:
    } {
    \begin{tabularx}{\textwidth}{llXll}
    \toprule
    Object member & XML type & Description & Mutability & Multiplicity \\
    \colrule
    action & String & Type identifier & immutable & 1 \\

    attributes & occi:attributeType & Instance attributes (see
\ref{sec:format_attribute_description}) & mutable & 0..* \\
    \botrule
    \end{tabularx}
}

\subsection{Kind Instance Format}
\label{sec:format_kind}

The OCCI Kind Instance Format consists of a XML object as shown in the
following example. Section \ref{sec:example_kind} contains a detailed example.
Table~\ref{tbl:format_kind} defines the top-level object members.
%FIXME : categoryGroup?
\begin{lstlisting}
<xs:complexType name="kindType">
	<xs:sequence>
		<xs:element name="parent" type="occi:categoryType" minOccurs="1" maxOccurs="1" />
		<xs:element name="attribute" type="occi:attributeSpecType" minOccurs="0" maxOccurs="unbounded" />
		<xs:element name="action" type="occi:actionSpecType" minOccurs="0" maxOccurs="unbounded" />
	</xs:sequence>
	<xs:attributeGroup ref="occi:categoryGroup"></xs:attributeGroup>
	<xs:attribute name="location" type="xs:anyURI"></xs:attribute>
</xs:complexType>

Example:
<kinds location="xs:anyURI">
	<parent type="occi:categoryType"/>
	<attribute type="occi:attributeSpecType"/>
	<action type="occi:actionSpecType"/>
</kind>
\end{lstlisting}

\mytablefloat{
    \label{tbl:format_kind}
    OCCI Kind instances are rendered inside the top-level XML object with name
{\em kinds} as a xs:complexType of XML objects with the following entries:
    } {
    \begin{tabularx}{\textwidth}{llXll}
    \toprule
    Object member & XML type & Description & Mutability & Multiplicity \\
    \colrule

    parent & occi:categoryType & OCCI Kind type identifier of the
related ``parent'' \hl{Kind} instance & immutable & 1..1 \\

    attributes & occi:attributeSpecType & Attribute description, see
~\ref{tbl:format_attribute_description} & immutable & 0..* \\

    actions & occi:actionSpecType & List of OCCI Action type
identifiers & immutable & 0..* \\

    location & xs:anyURI & Transport protocol specific URI bound to the OCCI Kind
instance. MUST be supplied for the OCCI Kinds of all OCCI Entities except OCCI
Entity itself & immutable & 0..1 \\
    \botrule
    \end{tabularx}
}

\subsection{Mixin Instance Format}
\label{sec:format_mixin}

The OCCI Mixin Instance Format consists of a XML object as shown in the following example. Section \ref{sec:example_mixin} contains a detailed example.
Table~\ref{tbl:format_mixin} defines the top-level object members.
%FIXME : attributeGroup? and location?
\mytablefloat{
    \label{tbl:format_mixin}
    OCCI Mixin instances are rendered inside the top-level XML object with name
{\em mixins} as a xs:complexType of XML objects with the following entries:
    } {
    \begin{tabularx}{\textwidth}{llXll}
    \toprule
    Object member & XML type & Description & Mutability & Multiplicity \\
    \colrule
    attributes & occi:attributeSpecType & Attribute description, see
~\ref{tbl:format_attribute_description} & immutable & 0..* \\

    depends & occi:categoryType & List of type identifiers of the dependent
 \hl{Mixin} instances & immutable & 0..* \\
 
    applies & occi:categoryType & List of OCCI Kind type identifiers this OCCI 
Mixin can be applied to \\

    actions & occi:actionSpecType & List of OCCI Action type identifiers
& immutable & 0..* \\

    location & xs:anyURI & Transport protocol specific URI bound to the OCCI Mixin
instance & immutable & 1 \\
    \botrule
    \end{tabularx}
}

\begin{lstlisting}
<xs:complexType name="mixinType">
	<xs:sequence>
		<xs:element name="depends" type="occi:categoryType" minOccurs="0" maxOccurs="unbounded" />
		<xs:element name="applies" type="occi:categoryType" minOccurs="0" maxOccurs="unbounded" />
		<xs:element name="attribute" type="occi:attributeSpecType" minOccurs="0" maxOccurs="unbounded" />
		<xs:element name="action" type="occi:actionSpecType" minOccurs="0" maxOccurs="unbounded" />
	</xs:sequence>
	<xs:attributeGroup ref="occi:categoryGroup"></xs:attributeGroup>
	<xs:attribute name="location" type="xs:anyURI"></xs:attribute>
</xs:complexType>

Example:
<mixins location="xs:anyURI">
	<depends type="occi:categoryType"/>
	<applies type="occi:categoryType"/>
	<attribute type="occi:attributeSpecType"/>
	<action type="occi:actionSpecType"/>
</mixins>
\end{lstlisting}

\subsection{Action Instance Format}
\label{sec:format_action}

The OCCI Action Instance Format consists of a XML object as shown in the
following example.
Table~\ref{tbl:format_action} defines the top-level object members.
%FIXME : categoryGroup?
\begin{lstlisting}
<xs:complexType name="actionSpecType">
	<xs:sequence>
		<xs:element name="attribute" type="occi:attributeSpecType" minOccurs="0" maxOccurs="unbounded"/>
	</xs:sequence>
	<xs:attributeGroup ref="occi:categoryGroup"></xs:attributeGroup>
</xs:complexType>

Example:
<action>
	<attributes type="occi:attributeSpecType">
</action>
\end{lstlisting}

\mytablefloat{
    \label{tbl:format_action}
    An OCCI Action invocation is rendered as top-level XML object with
 the following entries:
    } {
    \begin{tabularx}{\textwidth}{llXll}
    \toprule
    Object member & XML type & Description & Mutability & Multiplicity \\
    \colrule

    attributes & occi:attributeSpecType & Instance attributes (see
\ref{sec:format_attribute_description}) & mutable & 0..* \\
    \botrule
    \end{tabularx}
}

\subsection{Attribute Description Format}
\label{sec:format_attribute_description}

OCCI Attribute Descriptions are rendered as XML objects as defined in table~\ref{tbl:format_attribute_description}

\begin{lstlisting}
<xs:complexType name="attributeSpecType">
	<xs:complexContent>
		<xs:extension base="xs:attribute">
			<xs:attribute name="immutable" type="xs:boolean" default="false">
				<xs:annotation>
					<xs:documentation>
						If 'immutable' is true, the attribute can
						not be
						modified by user.
					</xs:documentation>
				</xs:annotation>
			</xs:attribute>
		</xs:extension>
	</xs:complexContent>
</xs:complexType>

Example:
<immutable type="xs:boolean">
\end{lstlisting}

\mytablefloat{
    \label{tbl:format_attribute_description}
    All properties of the OCCI Attribute definition are optional, but may contain
defaults which MUST be used if the Attribute is not present in the instantiated
OCCI Entity.
    } {
    \begin{tabularx}{\textwidth}{llXll}
    \toprule
    Object member & XML type & Description & Default \\
    \colrule
    immutable & boolean & Defines if the Attribute is mutable after initialization
& false \\

%    Pattern & string & Posix Extended Regular Expression as defined in
%\cite{iso9945:2009}. For interoperability reasons, POSIX character classes
% (e.g. [:alpha:]) MUST NOT be used. & .* \\

    \botrule
    \end{tabularx}
}

\section{Appendix: Full XML schema}
\label{sec:full_xml_schema}

\subsection{OCCI XML schema}
\label{sec:occi_xml_schema}

\lstinputlisting[language=XML]{schemas/occi.xsd}

\section{Detailed Examples}
\label{sec:examples}

%FIXME: Check that examples are correct and cover most aspects

\subsection{Resource Instance Format Example}
\label{sec:example_resource}

\begin{lstlisting}
<resource xmlns='http://schemas.ogf.org/occi' xmlns:xl='http://www.w3.org/2008/06/xlink' title='Machine a toto'>
	<kind scheme='http://schemas.ogf.org/occi/infrastructure#'term='compute'/>
	<attribute name='occi.compute.speed' value='4.0e3'/>
	<attribute name='occi.compute.memory' value='5.0'/>
	<attribute name='occi.compute.hostname' value='pc_toto'/>
	<attribute name='occi.compute.cores' value='2'/>
	<attribute name='occi.compute.architecture' value='x86'/>
	<link id='xmpp+occi:user-1@localhost/store/mylinks/json/networkinterface/0008413e-4091-42a8-9e01-f93c8c40ec21'>
	<kind scheme='http://schemas.ogf.org/occi/infrastructure#'term='networkinterface'/>
		<attribute name='occi.core.target' xl:href='xmpp+occi:user-1@localhost/store/myresources/xml/network/6a7d9971-214d-44c1-9da2-af1229584844'/>
		<attribute name='occi.networkinterface.address' value='192.168.3.4'/>
		<attribute name='occi.networkinterface.allocation' value='dynamic'/>
		<attribute name='occi.networkinterface.gateway' value='192.168.3.0'/>
		<attribute name='occi.networkinterface.interface' value='eth0'/>
		<attribute name='occi.networkinterface.mac' value='00:80:41:ae:fd:32'/>
		<mixin scheme='http://schemas.ogf.org/occi/infrastructure/networkinterface#' term='ipnetworkinterface'/>
	</link>
</resource>
\end{lstlisting}

\subsection{Link Instance Format Example}
\label{sec:example_link}

\begin{lstlisting}
<link xmlns='http://schemas.ogf.org/occi' xmlns:xl='http://www.w3.org/2008/06/xlink' id='xmpp+occi:user-1@localhost/store/mylinks/xml/networkinterfaces/id9e295664-afff-419b-8a1d-e092bdf6f718'>
	<attribute name='occi.core.source' xl:href='xmpp+occi:user-1@localhost/store/myresources
	/xml/compute/f322e40a-6c9f-42b0-8a63-231f7ac7f3e4'/>
	<kind scheme='http://schemas.ogf.org/occi/infrastructure#' term='networkinterface'/>
	<attribute name='occi.core.target' xl:href='xmpp+occi:user-1@localhost/store/myresources/xml/network/6a7d9971-214d-44c1-9da2-af1229584844'/>
	<attribute name='occi.networkinterface.address' value='192.168.3.4'/>
	<attribute name='occi.networkinterface.allocation' value='dynamic'/>
	<attribute name='occi.networkinterface.gateway' value='192.168.3.0'/>
	<attribute name='occi.networkinterface.interface' value='eth0'/>
	<attribute name='occi.networkinterface.mac' value='00:80:41:ae:fd:32'/>
	<mixin scheme='http://schemas.ogf.org/occi/infrastructure/networkinterface#' term='ipnetworkinterface'/>
</link>
\end{lstlisting}

\subsection{Action Invocation Format Example}
\label{sec:example_action_invocation}
%FIXME : A voir
\begin{lstlisting}
<occi:action xmlns="http://schemas.ogf.org/occi"
term="stop" scheme="http://schemas.ogf.org/occi/infrastructure/compute/action#" >
	<occi:attribute name="method" value="graceful"/>
</occi:action>
\end{lstlisting}

\subsection{Kind Format Example}
\label{sec:example_kind}
%FIXME : A voir
\begin{lstlisting}
<occi:kind scheme="http://schemas.ogf.org/occi/infrastructure#" term="compute" title="Compute Ressource">
	<occi:attribute name="occi.compute.hostname.mutable" value="true"/>
	<occi:attribute name="occi.compute.hostname.required" value="false"/>
	<occi:attribute name="occi.compute.hostname.type" value="xs:string"/>
	<occi:attribute name="occi.compute.hostname.description" value="Hostname of the compute ressource"/>
	<occi:attribute name="occi.compute.state.mutable" value="false"/>
	<occi:attribute name="occi.compute.hostname.required" value="false"/>
	<occi:attribute name="occi.compute.hostname.type" value="xs:string"/>
	<occi:attribute name="occi.compute.hostname.default" value="inactive"/>
	<occi:attribute name="occi.compute.hostname.description" value="State the compute ressource is in"/>
	<occi:action sechme="http://schemas.ogf.org/occi/infrastructure/compute/action#start"/>
	<occi:action sechme="http://schemas.ogf.org/occi/infrastructure/compute/action#stop"/>
	<occi:action sechme="http://schemas.ogf.org/occi/infrastructure/compute/action#restart"/>
	<occi:action sechme="http://schemas.ogf.org/occi/infrastructure/compute/action#suspend"/>
</occi:kind>
\end{lstlisting}

\subsection{Mixin Format Example}
\label{sec:example_mixin}
%FIXME : A regarder
\begin{lstlisting}
<occi:mixin scheme="http://schemas.ogf.org/occi/infrastructure#" term="ressource_tpl" title="Medium VM">
	<occi:attribute name="occi.compute.speed.type" value="number"/>
	<occi:attribute name="occi.compute.speed.default" value="2.8">		
</occi:mixin>
\end{lstlisting}

\subsection{Action Format Example}
\label{sec:example_action}
%FIXME : A regarder 
\begin{lstlisting}
<occi:action scheme="http://schemas.ogf.org/occi/infrastructure/compute/action#" term="stop" title="Stop Compute instance">
	<occi:attribute name="method.mutable" value="true"/>
	<occi:attribute name="method.required" value="false"/>
	<occi:attribute name="method.type" value="string"/>
	<occi:attribute name="method.default" value="poweroff"/>
</occi:action>
\end{lstlisting}

\section{Glossary}
\label{sec:glossary}
\begin{tabular}{l|p{11cm}}
Term & Description \\
\hline
\hl{Action} & An OCCI base type. Represent an invocable operation on a \hl{Entity} sub-type instance or collection thereof. \\
\hl{Category} & A type in the OCCI model. The parent type of \hl{Kind}. \\
Client & An OCCI client.\\
Collection & A set of \hl{Entity} sub-type instances all associated to a particular \hl{Kind} instance. \\
\hl{Entity} & An OCCI base type. The parent type of \hl{Resource} and \hl{Link}. \\
\hl{Kind} & A type in the OCCI model. The central piece in the OCCI classification system. \\
\hl{Link} & An OCCI base type. A \hl{Link} instance associate one \hl{Resource} instance with another. \\
Mix-in & A non-structural \hl{Kind}. The ``mix-in like'' concept in OCCI only
  support binding of new attributes and \hl{Action}s at run-time. A
  non-structural \hl{Kind} can only be associated with an {\em instance} of an
  \hl{Entity} sub-type. \\
Non-structural \hl{Kind} & An instance of \hl{Kind} {\em not} used as an unique identifier of an OCCI base type. \\
OCCI & Open Cloud Computing Interface \\
OCCI base type & One of \hl{Entity}, \hl{Resource}, \hl{Link} or \hl{Action}. \\
OGF & Open Grid Forum \\
\hl{Resource} & An OCCI base type. The parent type for all domain-specific resource types. \\
Structural \hl{Kind} & An instance of \hl{Kind} assigned as the unique identifier of an OCCI base type. \\
Tag & A non-structural \hl{Kind} with no attributes or actions defined. \\
URI & Uniform Resource Identifier \\
URL & Uniform Resource Locator \\
URN & Uniform Resource Name \\
\end{tabular}


\section{Contributors}

We would like to thank the following people who contributed to this
document:

\begin{tabular}{l|p{2in}|p{2in}}
Name & Affiliation & Contact \\
\hline
Michael Behrens & R2AD & behrens.cloud at r2ad.com \\
Mark Carlson & Oracle & mark.carlson at oracle.com \\
Andy Edmonds & Intel - SLA@SOI project & andy at edmonds.be \\
Sam Johnston & Google & samj at samj.net \\
Gary Mazzaferro & OCCI Counselour - AlloyCloud, Inc. &  garymazzaferro at gmail.com \\ 
Thijs Metsch & Platform Computing, Sun Microsystems & tmetsch at platform.com \\
Ralf Nyrén & Aurenav & ralf at nyren.net \\
Alexander Papaspyrou & TU Dortmund University & alexander.papaspyrou at tu\-dortmund.de \\
Alexis Richardson & RabbitMQ & alexis at rabbitmq.com \\
Shlomo Swidler & Orchestratus & shlomo.swidler at orchestratus.com \\
Florian Feldhaus & GWDG & florian.feldhaus at gwdg.de \\
Jean Parpaillon & & jean.parpaillon at free.fr \\
\end{tabular}

Next to these individual contributions we value the contributions from
the OCCI working group.

% FIXME: Insert an up-to-date table here!

\section{Intellectual Property Statement}
The OGF takes no position regarding the validity or scope of any
intellectual property or other rights that might be claimed to pertain
to the implementation or use of the technology described in this
document or the extent to which any license under such rights might or
might not be available; neither does it represent that it has made any
effort to identify any such rights. Copies of claims of rights made
available for publication and any assurances of licenses to be made
available, or the result of an attempt made to obtain a general
license or permission for the use of such proprietary rights by
implementers or users of this specification can be obtained from the
OGF Secretariat.

The OGF invites any interested party to bring to its attention any
copyrights, patents or patent applications, or other proprietary
rights which may cover technology that may be required to practice
this recommendation. Please address the information to the OGF
Executive Director.


\section{Disclaimer}
\input{include/disclaimer}

\section{Full Copyright Notice}
Copyright \copyright ~Open Grid Forum (2009-2016). All Rights Reserved.

This document and translations of it may be copied and furnished to
others, and derivative works that comment on or otherwise explain it
or assist in its implementation may be prepared, copied, published and
distributed, in whole or in part, without restriction of any kind,
provided that the above copyright notice and this paragraph are
included on all such copies and derivative works. However, this
document itself may not be modified in any way, such as by removing
the copyright notice or references to the OGF or other organizations,
except as needed for the purpose of developing Grid Recommendations in
which case the procedures for copyrights defined in the OGF Document
process must be followed, or as required to translate it into
languages other than English.

The limited permissions granted above are perpetual and will not be
revoked by the OGF or its successors or assignees.


\bibliographystyle{IEEEtran}
\bibliography{references}

\end{document}
