\documentclass[10pt,a4paper]{article}
\usepackage[utf8]{inputenc}
\usepackage{fullpage}
\usepackage{graphicx}
\usepackage{fancyhdr}
\usepackage{comment}
\usepackage{occi}
%\usepackage{lineno}   % adds line numbers, may be removed for non draft versions
%\linenumbers          % adds line numbers, may be removed for non draft versions
\usepackage{verbatim} % adds verbatim options
\usepackage{tabularx} % adds extended tabular formatting options
\usepackage{listings}
\usepackage{color}
\usepackage{placeins}
\definecolor{lightgray}{rgb}{.9,.9,.9}
\definecolor{darkgray}{rgb}{.4,.4,.4}
\definecolor{purple}{rgb}{0.65, 0.12, 0.82}

\lstdefinelanguage{json}{
  ndkeywords={String, Number, Boolean, Null, Object, Array},
  ndkeywordstyle=\itshape
}
\lstset{
   language=json,
   basicstyle=\footnotesize,
}

\setlength{\headheight}{13pt}
\pagestyle{fancy}

% default sans-serif
\renewcommand{\familydefault}{\sfdefault}

% no lines for headers and footers
\renewcommand{\headrulewidth}{0pt}
\renewcommand{\footrulewidth}{0pt}

% header
\fancyhf{}
\lhead{GWD-R}
\rhead{\today}

% footer
\lfoot{occi-wg@ogf.org}
\rfoot{\thepage}

% paragraphs need some space...
\setlength{\parindent}{0pt}
\setlength{\parskip}{1ex plus 0.5ex minus 0.2ex}

% some space between header and text...
\headsep 13pt

\setcounter{secnumdepth}{4}

\begin{document}

% header on first page is different
\thispagestyle{empty}

GWD-R \hfill Jean Parpaillon\\
OCCI-WG \\
\rightline {September 26, 2013}\\
\rightline {Updated: \today}

\vspace*{0.5in}

\begin{Large}
\textbf{Open Cloud Computing Interface - XML Rendering}
\end{Large}

\vspace*{0.5in}

\underline{Status of this Document}

This document provides information to the community regarding the
specification of the Open Cloud Computing Interface. Distribution is
unlimited.

\underline{Copyright Notice}

Copyright \copyright Open Grid Forum (2013). All Rights Reserved.

\underline{Trademarks}

OCCI is a trademark of the Open Grid Forum.

\underline{Abstract}

This document, part of a document series, produced by the OCCI working
group within the Open Grid Forum (OGF), provides a high-level
definition of a Protocol and API. The document is based upon
previously gathered requirements and focuses on the scope of important
capabilities required to support modern service offerings.

\underline{Comments}
\newcommand{\ralf}[1]{\textcolor{red}{RN: #1}}
\newcommand{\andy}[1]{\textcolor{green}{AE: #1}}
\newcommand{\florian}[1]{\textcolor{blue}{FF: #1}}
\newcommand{\jean}[1]{\textcolor{purple}{JP: #1}}

\newpage
\tableofcontents
\newpage

\section{Introduction}
The Open Cloud Computing Interface (OCCI) is a RESTful Protocol and
API for all kinds of management tasks. OCCI was originally initiated
to create a remote management API for IaaS%
\footnote{Infrastructure as a Service}
model-based services, allowing for the development of interoperable tools for
common tasks including deployment, autonomic scaling and monitoring.
%
It has since evolved into a flexible API with a strong focus on
interoperability while still offering a high degree of extensibility. The
current release of the Open Cloud Computing Interface is suitable to serve many
other models in addition to IaaS, including PaaS and SaaS.

In order to be modular and extensible the current OCCI specification is
released as a suite of complimentary documents, which together form the complete
specification.
%
The documents are divided into four categories consisting of the OCCI Core,
the OCCI Protocols, the OCCI Renderings and the OCCI Extensions.
%
\begin{itemize}
\item The OCCI Core specification consists of a single document defining the
 OCCI Core Model. The OCCI Core Model can be interacted with through {\em
 renderings} (including associated behaviors) and expanded through {\em extensions}.
\item The OCCI Protocol specifications consist of multiple documents, each
 describing how the model can be interacted with over a particular protocol (e.g. HTTP, AMQP etc.).
 Multiple protocols can interact with the same instance of the OCCI Core Model.
\item The OCCI Rendering specifications consist of multiple documents, each
 describing a particular rendering of the OCCI Core Model. Multiple renderings can
 interact with the same instance of the OCCI Core Model and will automatically support
 any additions to the model which follow the extension rules defined in OCCI
 Core.
\item The OCCI Extension specifications consist of multiple documents,
  each describing a particular extension of the OCCI Core Model. The
  extension documents describe additions to the OCCI Core Model
  defined within the OCCI specification suite.
\end{itemize}
%

The current specification consists of seven documents. This
specification describes version 1.2 of OCCI and is backward compatible with 1.1.
Future releases of OCCI
may include additional protocol, rendering and extension specifications. The specifications to be
implemented (MUST, SHOULD, MAY) are detailed in the table below.

\mytablefloat{
	\label{tbl:occi_compliancy}%
	What OCCI specifications must be implemented for the specific version.
}
{
	\begin{tabular}{lll}
	\toprule
	Document & OCCI 1.1 & OCCI 1.2 \\
	\colrule
	Core Model & MUST & MUST \\
	Infrastructure Model  & SHOULD & SHOULD \\
	Platform Model & MAY & MAY \\
	SLA Model & MAY & MAY \\
	HTTP Protocol & MUST & MUST \\
	Text Rendering& MUST & MUST \\
	JSON Rendering& MAY & MUST \\
	\botrule
	\end{tabular}
}

% hello


\section{Notational Conventions}
\input{include/notational}

\section{OCCI XML Rendering}

The OCCI XML Rendering specifies a rendering of OCCI instance types in
the eXtensible Markup Language (XML), as defined in \cite{w3c:xml11}.

The Rendering can be used to render OCCI instances independently of the
transport mechanism being used. Thus messages can be delivered by e.g. the HTTP
protocol as specified in \cite{occi:http_rendering} or by using text files with
the .xml file extension as defined in \cite{w3c:xml11}.

The following media-type MUST be used for the OCCI XML Rendering:

{\tt application/occi+xml}

\section{XML Format}
\label{sec:xml_format}

The OCCI XML Rendering consists of a sequence of XML element containing
information on the OCCI Core instances OCCI Kind, OCCI Mixin, OCCI
Action, OCCI Link and OCCI Resource. The rendering also include an XML
element to invoke the operation identified by OCCI Actions.
The rendering of each OCCI Core instance will be described in the
following sections.

\jean{Planned, not present in XML schema}

OCCI XML rendering include XML element to represent the
CRUD\footnote{Create, Retrieve, Update, Delete} actions on OCCI types.

\subsection{Category Format}
\label{sec:format_category}
The OCCI Category Format consists of a XML object.
Table~\ref{tbl:format_category} defines the object members.

\begin{lstlisting}
<xs:complexType name="categoryType">	
	<xs:attributeGroup ref="occi:categoryGroup"></xs:attributeGroup>
</xs:complexType>

<xs:attributeGroup name="categoryIdGroup" >
	<xs:attribute name="scheme" type="xs:anyURI"></xs:attribute>
	<xs:attribute name="term" type="xs:string" use="required"></xs:attribute>
</xs:attributeGroup>

<xs:attributeGroup name="categoryGroup">	
	<xs:attributeGroup ref="occi:categoryIdGroup"></xs:attributeGroup>
	<xs:attribute name="title" type="xs:string"></xs:attribute>
	<xs:attribute name="use">
		<xs:simpleType>
			<xs:restriction base="xs:string">
				<xs:enumeration value="required"></xs:enumeration>
				<xs:enumeration value="optional"></xs:enumeration>
			</xs:restriction>
		</xs:simpleType>
	</xs:attribute>
</xs:attributeGroup>
\end{lstlisting}

\mytablefloat{
    \label{tbl:format_category}
	Attribute defined for the Categorie type}
	{
    \begin{tabularx}{\textwidth}{llXll}
    \toprule
    Attribute & XML type & Description & Required \\
    \colrule
    term & xs:string & Unique identifier of the Category instance within the categorisation scheme & yes \\
    scheme & xs:anyURI & The categorisation scheme & yes \\
    title & xs:string & The display name of an instance & no \\
    use & required & Whether the OCCI Attribute must be supplied by the client at instance creation time & no \\
    \botrule
    \end{tabularx}
}
\FloatBarrier

\subsection{Attribute Description Format}
\label{sec:format_attribute_description}

OCCI Attribute Descriptions are rendered as XML objects as defined in table~\ref{tbl:format_attribute_description} and table~\ref{tbl:format_attribute_description_Att}.

\begin{lstlisting}
<xs:complexType name="attributeSpecType">
	<xs:complexContent>
		<xs:extension base="xs:attribute">
			<xs:attribute name="immutable" type="xs:boolean" default="false">
				<xs:annotation>
					<xs:documentation>
						If 'immutable' is true, the attribute can
						not be
						modified by user.
					</xs:documentation>
				</xs:annotation>
			</xs:attribute>
		</xs:extension>
	</xs:complexContent>
</xs:complexType>

<xs:complexType name="attributeType">
	<xs:sequence>
		<xs:element name="value" type="xs:anyURI" minOccurs="0" maxOccurs="unbounded" />
	</xs:sequence>
	<xs:attribute name="name" type="xs:string"></xs:attribute>
	<xs:attribute name="value" type="xs:string"></xs:attribute>
</xs:complexType>
\end{lstlisting}


\mytablefloat{
    \label{tbl:format_attribute_description}
    Element defined for the Attribute type
    } {
    \begin{tabularx}{\textwidth}{llXll}
    \toprule
    Element & XML type & Description & Mutability & Multiplicity \\
    \colrule
    value & xs:anyURI & Notion of array & Immutable & 0..* \\
    \botrule
    \end{tabularx}
}
\mytablefloat{	   
    \label{tbl:format_attribute_description_Att}
     Attribute defined for the Attribute type
     }{
     \begin{tabularx}{\textwidth}{llXll}
    \toprule
     Attribute & XML type & Description & Required \\
    \colrule
    name & xs:string & OCCI Attribute name & yes\\
    value & xs:string & Description & no\\
    type & XML type & OCCI Attribute type & no\\
    title & String & A descriptin of the OCCI Attribute & no \\
    use & required & Whether the OCCI Attribute must be supplied by the client at instance creation time & no\\
    default & XML type & OCCI Attribute default value & no\\
    immutable & xs:boolean & OCCI Attribute mutability & no \\
    \botrule
    \end{tabularx}
}
\FloatBarrier

\begin{lstlisting}
Example:
<occi:attribute name="occi.compute.state" use="required"
		default="inactive" immutable="true"
		title="System state" type="xs:string" />
\end{lstlisting}

\subsection{Entity Format}
\label{sec:format_entity}

The OCCI Entity Format consists of a XML object.
Table~\ref{tbl:format_entity} and table~\ref{tbl:format_entity_Att} define the object members.

\begin{lstlisting}
<xs:complexType name="entityType">
	<xs:sequence>
		<xs:element name="kind" type="occi:categoryType" minOccurs="1" maxOccurs="1" />
		<xs:element name="mixin" type="occi:categoryType" minOccurs="0" maxOccurs="unbounded" />
		<xs:element name="attribute" type="occi:attributeType" minOccurs="0" maxOccurs="unbounded" />
		<xs:element name="action" type="occi:categoryType" minOccurs="0" maxOccurs="unbounded" />
	</xs:sequence>
	<xs:attribute name="id" type="xs:anyURI" ></xs:attribute>
	<xs:attribute name="title" type="xs:string" ></xs:attribute>
	<xs:attribute name="href" type="xl:href" />
</xs:complexType>
\end{lstlisting}

\mytablefloat{
    \label{tbl:format_entity}
    Element defined for the Entity type
    } {
    \begin{tabularx}{\textwidth}{llXll}
    \toprule
    Element & XML type & Description & Mutability & Multiplicity \\
    \colrule
    kind & occi:categoryType & Type identifier & immutable & 1 \\

    mixins & occi:categoryType & List of type identifiers of associated OCCI Mixins & mutable & 0..* \\

    attributes & occi:attributeType & Instance Attributes (see
\ref{sec:format_attribute_description}) & mutable & 0..* \\
    
    actions & occi:category & List of type identifiers of OCCI
Actions applicable to the OCCI Resource instance & mutable & 0..* \\
    \botrule
    \end{tabularx}
}
\mytablefloat{	   
    \label{tbl:format_entity_Att}
     Attribute defined for the Entity type
     }{    
    \begin{tabularx}{\textwidth}{llXll}
    \toprule
    Attribute & XML type & Description & Required \\
    \colrule
    id & xs:anyURI & ID of the OCCI Resource & yes\\

    title & xs:string & A Description of the OCCI Entity  & yes\\
            
    href & xl:href & Path of the OCCI Entity & yes\\
    \botrule
    \end{tabularx}
}
\FloatBarrier

\subsection{Resource Instance Format}
\label{sec:format_resource}

The OCCI Resource Instance Format consists of a XML object as shown in the
following example. Section \ref{sec:example_resource} contains a detailed
example.
Table~\ref{tbl:format_resource} and table~\ref{tbl:format_resource_Att} define the object members.

\begin{lstlisting}
<xs:complexType name="resourceType">
	<xs:complexContent>
		<xs:extension base="occi:entityType">
			<xs:sequence>
				<xs:element name="summary" type="xs:string" minOccurs="0" maxOccurs="1" />
				<xs:element name="link" type="occi:resourceLinkType" minOccurs="0" maxOccurs="unbounded" />
			</xs:sequence>
		</xs:extension>
	</xs:complexContent>
</xs:complexType>
\end{lstlisting}

\mytablefloat{
    \label{tbl:format_resource}
    OCCI Resource instances are rendered inside the top-level XML object with
name
{\em resources}
    as a ressourceType of XML objects with the following entries:
    } {
    \begin{tabularx}{\textwidth}{llXll}
    \toprule
    Element member & XML type & Description & Mutability & Multiplicity \\
    \colrule
    entity & occi:entityType & Set of entity instances & Mutable & 0..* \\
    \botrule
    \end{tabularx}
}
\mytablefloat{	   
    \label{tbl:format_resource_Att}
     Attribute defined for the Entity type
     }{      
    \begin{tabularx}{\textwidth}{llXll}
    \toprule
    Attribute & XML type & Description & Required \\
    \colrule
    summary & xs:string & A summarising description of the Ressource instance & no\\
            
    links & occi:resourceLinkType & List of URIs of OCCI Links & no\\
    \botrule
    \end{tabularx}
}
\FloatBarrier

\begin{lstlisting}
Example:
<resources id="xs:anyURI" title="xs:string" href="xl:href">
	<kind type="occi:categoryType"/>
	<mixins type="occi:categoryType"/>
        <attributes type="occi:attributeType"/>
        <actions: type="occi:attributeType"/>
        <links type="occi:resourceLinkType"> 
</resources>
\end{lstlisting}

\subsection{Link Instance Format}
\label{sec:format_link}

The OCCI Link Instance Format consists of a XML object as shown in the
following example. Section \ref{sec:example_link} contains a detailed example.
Table~\ref{tbl:format_link} defines the object members.

\begin{lstlisting}
<xs:complexType name="linkType">
	<xs:complexContent>
		<xs:extension base="occi:entityType" />
	</xs:complexContent>
</xs:complexType>
\end{lstlisting}

\mytablefloat{
    \label{tbl:format_link}
    OCCI Resource instances are rendered inside the top-level XML object with
name
{\em resources}
    as a linkType of XML objects with the following entries:
    } {
    \begin{tabularx}{\textwidth}{llXll}
    \toprule
    Element & XML type & Description & Mutability & Multiplicity \\
    \colrule
    entity & occi:entityType & Set of entity instances & Mutable & 0..* \\
    \botrule
    \end{tabularx}
}
\FloatBarrier

\begin{lstlisting}
Example:
<link id="xs:anyURI" title="xs:string" href="xl:href">
	<kind type="occi:categoryType"/>
	<mixins type="occi:categoryType"/>
        <attributes type="occi:attributeType"/>
        <actions: type="occi:attributeType"/>
        <links type="occi:resourceLinkType">
<link>
\end{lstlisting}

\subsection{Kind Instance Format}
\label{sec:format_kind}

The OCCI Kind Instance Format consists of a XML object as shown in the
following example. Section \ref{sec:example_kind} contains a detailed example.
Table~\ref{tbl:format_kind} and table~\ref{tbl:format_kind_Att} define the top-level object members.

\begin{lstlisting}
<xs:complexType name="kindType">
	<xs:sequence>
		<xs:element name="parent" type="occi:categoryType" minOccurs="1" maxOccurs="1" />
		<xs:element name="attribute" type="occi:attributeSpecType" minOccurs="0" maxOccurs="unbounded" />
		<xs:element name="action" type="occi:actionSpecType" minOccurs="0" maxOccurs="unbounded" />
	</xs:sequence>
	<xs:attributeGroup ref="occi:categoryGroup"></xs:attributeGroup>
	<xs:attribute name="location" type="xs:anyURI"></xs:attribute>
</xs:complexType>
\end{lstlisting}

\mytablefloat{
    \label{tbl:format_kind}
    OCCI Kind instances are rendered inside the top-level XML object with name
{\em kinds} as a kindType of XML objects with the following entries:
    } {
    \begin{tabularx}{\textwidth}{llXll}
    \toprule
    Element & XML type & Description & Mutability & Multiplicity \\
    \colrule
    parent & occi:categoryType & OCCI Kind type identifier of the
related ``parent'' \hl{Kind} instance & immutable & 1 \\

    attributes & occi:attributeSpecType & Attribute description & immutable & 0..* \\

    actions & occi:actionSpecType & List of OCCI Action type
identifiers & immutable & 0..* \\
    \botrule
    \end{tabularx}
}
\mytablefloat{	   
    \label{tbl:format_kind_Att}
     Attribute defined for the Entity type
     }{     
    \begin{tabularx}{\textwidth}{llXll}
    \toprule
    Attribute & XML type & Description & Required \\
    \colrule
    location & xs:anyURI & Transport protocol specific URI bound to the OCCI Kind
instance. MUST be supplied for the OCCI Kinds of all OCCI Entities except OCCI
Entity itself & no \\
   
    attributeGroup & occi:attributeGroup & Set of category & yes \\
    \botrule
    \end{tabularx}
}
\FloatBarrier

\begin{lstlisting}
Example:
<kinds location="xs:anyURI">
	<parent type="occi:categoryType"/>
	<attribute type="occi:attributeSpecType"/>
	<action type="occi:actionSpecType"/>
</kind>
\end{lstlisting}

\subsection{Mixin Instance Format}
\label{sec:format_mixin}

The OCCI Mixin Instance Format consists of a XML object as shown in the following example. Section \ref{sec:example_mixin} contains a detailed example.
Table~\ref{tbl:format_mixin} and table~\ref{tbl:format_mixin} define the top-level object members.
 
\begin{lstlisting}
<xs:complexType name="mixinType">
	<xs:sequence>
		<xs:element name="depends" type="occi:categoryType" minOccurs="0" maxOccurs="unbounded" />
		<xs:element name="applies" type="occi:categoryType" minOccurs="0" maxOccurs="unbounded" />
		<xs:element name="attribute" type="occi:attributeSpecType" minOccurs="0" maxOccurs="unbounded" />
		<xs:element name="action" type="occi:actionSpecType" minOccurs="0" maxOccurs="unbounded" />
	</xs:sequence>
	<xs:attributeGroup ref="occi:categoryGroup"></xs:attributeGroup>
	<xs:attribute name="location" type="xs:anyURI"></xs:attribute>
</xs:complexType>
\end{lstlisting}

\mytablefloat{
    \label{tbl:format_mixin}
    OCCI Mixin instances are rendered inside the top-level XML object with name
{\em mixins} as a mixinType of XML objects with the following entries:
    } {
    \begin{tabularx}{\textwidth}{llXll}
    \toprule
    Element & XML type & Description & Mutability & Multiplicity \\
    \colrule
    attributes & occi:attributeSpecType & Attribute description, see
~\ref{tbl:format_attribute_description} & immutable & 0..* \\

    depends & occi:categoryType & List of type identifiers of the dependent
 \hl{Mixin} instances & immutable & 0..* \\
 
    applies & occi:categoryType & List of OCCI Kind type identifiers this OCCI 
Mixin can be applied to \\

    actions & occi:actionSpecType & List of OCCI Action type identifiers
& immutable & 0..* \\
    \botrule
    \end{tabularx}
}
\mytablefloat{	   
    \label{tbl:format_mixin_Att}
     Attribute defined for the Entity type
     }{  
    \begin{tabularx}{\textwidth}{llXll}
    \toprule
    Attribute & XML type & Description & Required \\
    \colrule
    location & xs:anyURI & Transport protocol specific URI bound to the OCCI Mixin
instance & no \\

    attributeGroup & occi:attributeGroup & Set of category & yes \\
    \botrule
    \end{tabularx}
}
\FloatBarrier

\begin{lstlisting}
Example:
<mixins location="xs:anyURI">
	<depends type="occi:categoryType"/>
	<applies type="occi:categoryType"/>
	<attribute type="occi:attributeSpecType"/>
	<action type="occi:actionSpecType"/>
</mixins>
\end{lstlisting}

\subsection{Action Instance Format}
\label{sec:format_action}

The OCCI Action Instance Format consists of a XML object as shown in the
following example.
Table~\ref{tbl:format_action} and table~\ref{tbl:format_action_Att} define the top-level object members.

\begin{lstlisting}
<xs:complexType name="actionSpecType">
	<xs:sequence>
		<xs:element name="attribute" type="occi:attributeSpecType" minOccurs="0" maxOccurs="unbounded"/>
	</xs:sequence>
	<xs:attributeGroup ref="occi:categoryGroup"></xs:attributeGroup>
</xs:complexType>
\end{lstlisting}

\mytablefloat{
    \label{tbl:format_action}
    An OCCI Action invocation is rendered as top-level XML object with
 the following entries:
    } {
    \begin{tabularx}{\textwidth}{llXll}
    \toprule
    Element & XML type & Description & Mutability & Multiplicity \\
    \colrule
    
    attributes & occi:attributeSpecType & Instance attributes & mutable & 0..* \\ 
    \botrule
    \end{tabularx}
}
\mytablefloat{	   
    \label{tbl:format_action_Att}
     Attribute defined for the Entity type
     }{      
    \begin{tabularx}{\textwidth}{llXll}
    \toprule
    Attribute & XML type & Description & Required \\
    \colrule
    attributeGroup & occi:attributeGroup & Set of category & yes\\
    \botrule
    \end{tabularx}
}
\FloatBarrier

\begin{lstlisting}
Example:
<occi:action term="suspend"
scheme="http://schemas.ogf.org/occi/infrastructure/compute/action#"
title="Suspend the system (hibernate or in RAM)" >
	<occi:attribute name="method" type="xs:string" />
</occi:action>
\end{lstlisting}

\subsection{Action Invocation Format}
\label{sec:format_action_invocation}

The OCCI Action Invocation  Format identifies an invocable operation on a OCCI Resource or
OCCI Link instance. To trigger such an operation the OCCI Action Invocation
Format is required.

The OCCI Action Invocation Format consists of a top-level XML object as shown in the
following example. Section \ref{sec:example_action_invocation} contains a detailed example.
Table~\ref{tbl:format_action_invocation} and table~\ref{tbl:format_action_invocation_Att} define the object members.

\begin{lstlisting}
<xs:complexType name="actionType">
	<xs:sequence>
		<xs:element name="attribute" type="occi:attributeType"></xs:element>
	</xs:sequence>
	<xs:attributeGroup ref="occi:categoryGroup"></xs:attributeGroup>
</xs:complexType>
\end{lstlisting}

\mytablefloat{
    \label{tbl:format_action_invocation}
    An OCCI Action invocation is rendered as top-level XML object with
 the following entries:
    } {
    \begin{tabularx}{\textwidth}{llXll}
    \toprule
    Element & XML type & Description & Mutability & Multiplicity \\
    \colrule
    
    attribute & occi:attributeType & Instance attributes & mutable & 0..* \\
    \botrule
    \end{tabularx}
}
\mytablefloat{	   
    \label{tbl:format_action_invocation_Att}
     Attribute defined for the Entity type
     }{      
    \begin{tabularx}{\textwidth}{llXll}
    \toprule
    Attribute & XML type & Description & Required \\
    \colrule
    attributeGroup & occi:categoryGroup & Set of category & yes \\
    
    \botrule
    \end{tabularx}
}
\FloatBarrier

\begin{lstlisting}
Example:
<action xmlns="http://schemas.ogf.org/occi"
term="stop" scheme="http://schemas.ogf.org/occi/infrastructure/compute/action#" >
	<attribute name="method" value="graceful" />
</action>
\end{lstlisting}

\section{Appendix: Full XML schema}
\label{sec:full_xml_schema}

\subsection{OCCI XML schema}
\label{sec:occi_xml_schema}

\lstinputlisting[language=XML]{schemas/occi.xsd}

\section{Detailed Examples}
\label{sec:examples}

%FIXME: Check that examples are correct and cover most aspects

\subsection{Resource Instance Format Example}
\label{sec:example_resource}

\begin{lstlisting}
<resource xmlns='http://schemas.ogf.org/occi' xmlns:xl='http://www.w3.org/2008/06/xlink' title='Machine a toto'>
	<kind scheme='http://schemas.ogf.org/occi/infrastructure#'term='compute'/>
	<attribute name='occi.compute.speed' value='4.0e3'/>
	<attribute name='occi.compute.memory' value='5.0'/>
	<attribute name='occi.compute.hostname' value='pc_toto'/>
	<attribute name='occi.compute.cores' value='2'/>
	<attribute name='occi.compute.architecture' value='x86'/>
	<link id='xmpp+occi:user-1@localhost/store/mylinks/json/networkinterface/0008413e-4091-42a8-9e01-f93c8c40ec21'>
	<kind scheme='http://schemas.ogf.org/occi/infrastructure#'term='networkinterface'/>
		<attribute name='occi.core.target' xl:href='xmpp+occi:user-1@localhost/store/myresources/xml/network/6a7d9971-214d-44c1-9da2-af1229584844'/>
		<attribute name='occi.networkinterface.address' value='192.168.3.4'/>
		<attribute name='occi.networkinterface.allocation' value='dynamic'/>
		<attribute name='occi.networkinterface.gateway' value='192.168.3.0'/>
		<attribute name='occi.networkinterface.interface' value='eth0'/>
		<attribute name='occi.networkinterface.mac' value='00:80:41:ae:fd:32'/>
		<mixin scheme='http://schemas.ogf.org/occi/infrastructure/networkinterface#' term='ipnetworkinterface'/>
	</link>
</resource>
\end{lstlisting}

\subsection{Link Instance Format Example}
\label{sec:example_link}

\begin{lstlisting}
<link xmlns='http://schemas.ogf.org/occi' xmlns:xl='http://www.w3.org/2008/06/xlink' id='xmpp+occi:user-1@localhost/store/mylinks/xml/networkinterfaces/id9e295664-afff-419b-8a1d-e092bdf6f718'>
	<attribute name='occi.core.source' xl:href='xmpp+occi:user-1@localhost/store/myresources
	/xml/compute/f322e40a-6c9f-42b0-8a63-231f7ac7f3e4'/>
	<kind scheme='http://schemas.ogf.org/occi/infrastructure#' term='networkinterface'/>
	<attribute name='occi.core.target' xl:href='xmpp+occi:user-1@localhost/store/myresources/xml/network/6a7d9971-214d-44c1-9da2-af1229584844'/>
	<attribute name='occi.networkinterface.address' value='192.168.3.4'/>
	<attribute name='occi.networkinterface.allocation' value='dynamic'/>
	<attribute name='occi.networkinterface.gateway' value='192.168.3.0'/>
	<attribute name='occi.networkinterface.interface' value='eth0'/>
	<attribute name='occi.networkinterface.mac' value='00:80:41:ae:fd:32'/>
	<mixin scheme='http://schemas.ogf.org/occi/infrastructure/networkinterface#' term='ipnetworkinterface'/>
</link>
\end{lstlisting}

\subsection{Action Invocation Format Example}
\label{sec:example_action_invocation}

\begin{lstlisting}
<occi:action xmlns="http://schemas.ogf.org/occi"
term="stop" scheme="http://schemas.ogf.org/occi/infrastructure/compute/action#" >
	<occi:attribute name="method" value="graceful"/>
</occi:action>
\end{lstlisting}

\subsection{Kind Format Example}
\label{sec:example_kind}
\begin{lstlisting}
<occi:kind scheme="http://schemas.ogf.org/occi/infrastructure#" term="compute" title="Compute Ressource">
	<occi:attribute name="occi.compute.hostname.mutable" value="true"/>
	<occi:attribute name="occi.compute.hostname.required" value="false"/>
	<occi:attribute name="occi.compute.hostname.type" value="xs:string"/>
	<occi:attribute name="occi.compute.hostname.description" value="Hostname of the compute ressource"/>
	<occi:attribute name="occi.compute.state.mutable" value="false"/>
	<occi:attribute name="occi.compute.hostname.required" value="false"/>
	<occi:attribute name="occi.compute.hostname.type" value="xs:string"/>
	<occi:attribute name="occi.compute.hostname.default" value="inactive"/>
	<occi:attribute name="occi.compute.hostname.description" value="State the compute ressource is in"/>
	<occi:action sechme="http://schemas.ogf.org/occi/infrastructure/compute/action#start"/>
	<occi:action sechme="http://schemas.ogf.org/occi/infrastructure/compute/action#stop"/>
	<occi:action sechme="http://schemas.ogf.org/occi/infrastructure/compute/action#restart"/>
	<occi:action sechme="http://schemas.ogf.org/occi/infrastructure/compute/action#suspend"/>
</occi:kind>
\end{lstlisting}

\subsection{Mixin Format Example}
\label{sec:example_mixin}

\begin{lstlisting}
<occi:mixin scheme="http://schemas.ogf.org/occi/infrastructure#" term="ressource_tpl" title="Medium VM">
	<occi:attribute name="occi.compute.speed.type" value="number"/>
	<occi:attribute name="occi.compute.speed.default" value="2.8">		
</occi:mixin>
\end{lstlisting}

\subsection{Action Format Example}
\label{sec:example_action} 

\begin{lstlisting}
<occi:action scheme="http://schemas.ogf.org/occi/infrastructure/compute/action#" term="stop" title="Stop Compute instance">
	<occi:attribute name="method.mutable" value="true"/>
	<occi:attribute name="method.required" value="false"/>
	<occi:attribute name="method.type" value="string"/>
	<occi:attribute name="method.default" value="poweroff"/>
</occi:action>
\end{lstlisting}

\section{Glossary}
\label{sec:glossary}
\todo{update glossary}

\begin{tabular}{l|p{12cm}}
Term & Description \\
\hline
\hl{Action} & An OCCI base type. Represents an invocable operation on a \hl{Entity} sub-type instance or collection thereof. \\

\hl{Attribute} & A type in the OCCI Core Model. Describes the name and properties of attributes found in \hl{Entity} types. \\

\hl{Category} & A type in the OCCI Core Model and the basis of the OCCI type identification mechanism. The parent type of \hl{Kind}. \\

capabilities & In the context of \hl{Entity} sub-types {\bf  capabilities} refer
  to the OCCI \hl{Attribute}s and OCCI \hl{Action}s exposed by an {\bf entity
  instance}. \\

\hl{Client} & An OCCI client.\\

\hl{Collection} & A set of \hl{Entity} sub-type instances all associated to a particular \hl{Kind} or \hl{Mixin} instance. \\

\hl{Entity} & An OCCI base type. The parent type of \hl{Resource} and \hl{Link}. \\

entity instance & An instance of a sub-type of \hl{Entity} but not an instance
  of the \hl{Entity} type itself.  The OCCI model defines two sub-types of
  \hl{Entity}, the \hl{Resource} type and the \hl{Link} type.  However, the
  term {\em entity instance} is defined to include any instance of a
  sub-type of \hl{Resource} or \hl{Link} as well. \\

\hl{Kind} & A type in the OCCI Core Model. A core component of the OCCI classification system. \\

\hl{Link} & An OCCI base type. A \hl{Link} instance associates one \hl{Resource} instance with another. \\

\hl{Mixin} & A type in the OCCI Core Model. A core component of the OCCI classification system. \\

mix-in & An instance of the \hl{Mixin} type associated with an {\em entity
 instance}. The ``mix-in'' concept as used by OCCI {\em only} applies to
 instances, never to \hl{Entity} types. \\

model attribute & An internal attribute of a the Core Model which is {\em not}
  client discoverable. \\

\hl{OCCI} & Open Cloud Computing Interface. \\

OCCI base type & One of \hl{Entity}, \hl{Resource}, \hl{Link} or \hl{Action}. \\

OCCI Action & see \hl{Action}. \\
OCCI Attribute & A client discoverable attribute identified by an instance of the \hl{Attribute} type. Examples are \hl{occi.core.title} and \hl{occi.core.summary}. \\
OCCI Category & see \hl{Category}. \\
OCCI Entity & see \hl{Entity}. \\
OCCI Kind & see \hl{Kind}. \\
OCCI Link & see \hl{Link}. \\
OCCI Mixin & see \hl{Mixin}. \\

OGF & Open Grid Forum. \\

\hl{Resource} & An OCCI base type. The parent type for all domain-specific \hl{Resource} sub-types. \\

resource instance & See {\em entity instance}. This term is considered obsolete. \\

tag & A \hl{Mixin} instance with no attributes or actions defined. \\

template & A \hl{Mixin} instance which if associated at instance
creation-time pre-populate certain attributes. \\

type & One of the types defined by the OCCI Core Model.  The Core Model types are
 \hl{Category}, \hl{Attribute},
 \hl{Kind}, \hl{Mixin}, \hl{Action}, \hl{Entity}, \hl{Resource}
 and \hl{Link}. \\

concrete type/sub-type & A concrete type/sub-type is a type that can be instantiated.\\

URI & Uniform Resource Identifier. \\
URL & Uniform Resource Locator. \\
URN & Uniform Resource Name. \\
\end{tabular}


\section{Contributors}

We would like to thank the following people who contributed to this
document:

\begin{tabular}{l|p{2in}|p{2in}}
Name & Affiliation & Contact \\
\hline
Michael Behrens & R2AD & behrens.cloud at r2ad.com \\
Andy Edmonds & Intel - SLA@SOI project & andy at edmonds.be \\
Sam Johnston & Google & samj at samj.net \\
Gary Mazzaferro & OCCI Counselour - Exxia, Inc. &  garymazzaferro at gmail.com \\ 
Thijs Metsch & Platform Computing, Sun Microsystems & tmetsch at platform.com \\
Ralf Nyrén & Aurenav & ralf at nyren.net \\
Alexander Papaspyrou & TU-Dortmund & alexander.papaspyour at tu\-dortmund.de \\
Shlomo Swidler & Orchestratus & shlomo.swidler at orchestratus.com \\
\end{tabular}

Next to these individual contributions we value the contributions from
the OCCI working group.



% FIXME: Insert an up-to-date table here!

\section{Intellectual Property Statement}
\input{include/ip}

\section{Disclaimer}
This document and the information contained herein is provided on an
``As Is'' basis and the OGF disclaims all warranties, express or
implied, including but not limited to any warranty that the use of the
information herein will not infringe any rights or any implied
warranties of merchantability or fitness for a particular purpose.


\section{Full Copyright Notice}
Copyright \copyright ~Open Grid Forum (2009-2016). All Rights Reserved.

This document and translations of it may be copied and furnished to
others, and derivative works that comment on or otherwise explain it
or assist in its implementation may be prepared, copied, published and 
distributed, in whole or in part, without restriction of any kind,
provided that the above copyright notice and this paragraph are
included as references to the derived portions on all such copies
and derivative works. The published OGF document from which such works
are derived, however, may not be modified in any way, such as by removing
the copyright notice or references to the OGF or other organizations,
except as needed for the purpose of developing new or updated OGF documents
in conformance with the procedures defined in the OGF Document Process,
or as required to translate it into languages other than English. OGF,
with the approval of its board, may remove this restriction for inclusion
of OGF document content for the purpose of producing standards in cooperation
with other international standards bodies. 

The limited permissions granted above are perpetual and will not be
revoked by the OGF or its successors or assignees.


\bibliographystyle{IEEEtran}
\bibliography{references}

\end{document}
