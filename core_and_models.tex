\documentclass[10pt,a4paper,british]{article}
\usepackage[utf8]{inputenc}
\usepackage{fullpage}
\usepackage{graphicx}
\usepackage{fancyhdr}
\usepackage{comment}
\usepackage{occi}
\setlength{\headheight}{13pt}
\pagestyle{fancy}

% default sans-serif
\renewcommand{\familydefault}{\sfdefault}

% no lines for headers and footers
\renewcommand{\headrulewidth}{0pt}
\renewcommand{\footrulewidth}{0pt}

% header
\fancyhf{}
\lhead{GWD-R}
\rhead{\today}

% footer
\lfoot{occi-wg@ogf.org}
\rfoot{\thepage}

% paragraphs need some space...
\setlength{\parindent}{0pt}
\setlength{\parskip}{1ex plus 0.5ex minus 0.2ex}

% some space between header and text...
\headsep 13pt

\setcounter{secnumdepth}{4}

\begin{document}

% header on first page is different
\thispagestyle{empty}

GWD-R \hfill  Thijs Metsch, Platform Computing\\
OCCI-WG \hfill  Andy Edmonds, Intel\\
\rightline {Ralf Nyrén, Aurenav}\\
\rightline {October 14, 2010}\\
\rightline {Updated: \today}

\vspace*{0.5in}

\begin{Large}
\textbf{Open Cloud Computing Interface - Core and Models}
\end{Large}

\vspace*{0.5in}

\underline{Status of this Document}

This document provides information to the community regarding the
specification of the Open Cloud Computing Interface. Distribution is
unlimited.

\underline{Obsoletes}

This document obsoletes GFD-xxx [REFERENCE].

\underline{Copyright Notice}

Copyright \copyright Open Grid Forum (2009-2010). All Rights Reserved.

\underline{Trademarks}

OCCI is a trademark of the Open Grid Forum.

\underline{Abstract}

This document, part of a document series, produced by the OCCI working
group within the Open Grid Forum (OGF), provides a high-level
definition of a Protocol and API. The document is based upon
previously gathered requirements and focuses on the scope of important
capabilities required to support modern service offerings.

\newpage
\tableofcontents
\newpage

\section{Introduction}
The Open Cloud Computing Interface (OCCI) is a RESTful Protocol and
API for all kinds of Management tasks. OCCI was originally initiated
to create a remote management API for IaaS%
\footnote{Infrastructure as a Service}
model based Services, allowing for the development of interoperable tools for
common tasks including deployment, autonomic scaling and monitoring.
%
It has since evolved into an flexible API with a strong focus on
interoperability while still offering a high degree of extensibility. The
current release of the Open Cloud Computing Interface is suitable to serve many
other models in addition to IaaS, including e.g.~PaaS and SaaS.

In order to be modular and extensible the current OCCI specification is
released as a suite of complimentary documents which together form the complete
specification.
%
The documents are divided into three categories consisting of the OCCI Core,
the OCCI Renderings and the OCCI Extensions.
%
\begin{itemize}
\item The OCCI Core specification consist of a single document defining the
 OCCI Core Model. The OCCI Core Model can be interacted with {\em
 renderings} (including associated behaviours) and expanded through {\em extensions}.
\item The OCCI Rendering specifications consist of multiple documents each
 describing a particular rendering of the OCCI Core Model. Multiple renderings can
 interact with the same instance of the OCCI Core Model and will automatically support
 any additions to the model which follow the extension rules defined in OCCI
 Core.
\item The OCCI Extension specifications consist of multiple documents each
 describing a particular extension of the OCCI Core Model. The extension documents
 describe additions to the OCCI Core Model defined within the OCCI specification
 suite.
\end{itemize}
%
The current specification consist of three documents.
Future releases of OCCI may include additional rendering and extension
specifications. The documents of the current OCCI specification suite are:

\begin{description}
\item[OCCI Core] describes the formal definition of the the OCCI Core Model
\cite{occi:core}.
\item[OCCI HTTP Rendering] defines how to interact with the OCCI Core Model using the
RESTful OCCI API \cite{occi:http_rendering}. The document defines how the OCCI Core Model can
be communicated and thus serialised using the HTTP protocol.
\item[OCCI Infrastructure] contains the definition of the OCCI Infrastructure
extension for the IaaS domain \cite{occi:infrastructure}. The document defines
additional resource types, their attributes and the actions that can be taken
on each resource type.
\end{description}


\section{Notational Conventions}
All these parts and the information within are mandatory for
implementors (unless otherwise specified). The key words "MUST", "MUST
NOT", "REQUIRED", "SHALL", "SHALL NOT", "SHOULD", "SHOULD NOT",
"RECOMMENDED", "MAY", and "OPTIONAL" in this document are to be
interpreted as described in RFC 2119 \cite{rfc2119}.


\section{OCCI model}
OCCI is a boundary protocol and API
that acts as a service front-end to a provider's internal management
framework. Figure~\ref{fig:placement} shows OCCI's place in a
provider's architecture.

\begin{figure}[!hp]
	\centering
	\includegraphics[scale=0.5]{figs/occi-intro.png}
	\caption{OCCI's place in a provider's architecture}
	\label{fig:placement}
\end{figure}

The heart of the OCCI model is the \hl{Resource} type. Any resource exposed
through OCCI is a \hl{Resource} or sub-type thereof.
A resource can be e.g.~a virtual machine, a job in a job submission system, a
user, etc.
%
The \hl{Resource} type contains a number of common attributes that
domain-specific \hl{Resource} types inherit. The \hl{Resource} type is
complemented by the \hl{Link} type which associates one \hl{Resource} instance
with another.
%
The \hl{Link} type also contains a number of common attributes that
domain-specific \hl{Link} types inherit.

\hl{Kind} is an abstract type which both \hl{Resource} and \hl{Link} inherit.
Each sub-type of \hl{Kind} is identified by a unique \hl{Type} instance.
%
The \hl{Type} type comprise the classification system built into the OCCI
model. \hl{Type} is a specialisation of \hl{Category} and introduce additional
capabilities in terms of \hl{Action} types.

\begin{figure}[!h]
{\centering \resizebox*{0.9\columnwidth}{!}{\rotatebox{270}
      {\includegraphics{figs/core_model.pdf}}} \par}
\caption{UML class diagram of the OCCI model. The diagram provides an
overview of the OCCI model but is not a standalone definition thereof}
\label{fig:occi_model}
\end{figure}

The UML class diagram shown in figure~\ref{fig:occi_model} gives an overview of
the OCCI model.  For compliance with OCCI Core, all of the types defined in
the OCCI model MUST be implemented.
The following sections of the specification define the details of the OCCI
model.

\subsection{Classification and Identification}
\label{sec:classification}
The OCCI model provides a built in classification system allowing for safe
extension towards domain-specific usage. This system is like a ``type system''
but with the possibility of being easily exposed over a text based protocol.
%
The classification system can be summarised with the following key features:
\begin{itemize}
\item Each OCCI base type and extension thereof is assigned a unique
 identifier, a structural \hl{Type}, which allow for dynamic discovery of available
 types.
\item The relationship of structural \hl{Type}s is part of the system and thus
 the inheritance model is also discoverable.
\item The classification system allows non-structural \hl{Type}s to be assigned
 to resource instances adding new capabilities using a mix-in like model.
\item Tagging of resource instances is supported through mix-in of
 non-structural \hl{Type}s which have no additional capabilities defined.
\item A collection of associated resources is implicitly defined for each
 structural and non-structural \hl{Type}. I.e.~all resource instances
 associated with a particular \hl{Type} instance form a collection.
\end{itemize}

\subsubsection{Type}
\label{sec:type}
The \hl{Type} type comprises the classification system provided by the OCCI
model. It MUST be implemented. A \hl{Type} instance can be either structural or
non-structural.

\mytablefloat{\label{tbl:type}Attributes defined for the \hl{Type} type}{
\begin{tabular}{llllp{2.7in}}
\toprule
Attribute & Type & Multiplicity & Client Mutability & Description \\
\colrule
actions & \hl{Action} & 0..* & Immutable & Set of \hl{Action}s defined by the \hl{Type} instance. \\
related & \hl{Type} & 0..* & Immutable & Set of related \hl{Type} instances. \\
kind & \hl{Kind} & 0..1 & Immutable & \hl{Kind} type uniquely identified by the \hl{Type} instance. \\
\botrule
\end{tabular}
}
The \hl{Type} type inherits the \hl{Category} type and all inherited
attributes MUST be implemented. Table~\ref{tbl:type} defines the
additional attributes the \hl{Type} type MUST implement to be compliant.

\paragraph*{Structural Type}
A structural \hl{Type} is an instance of \hl{Type} assigned as the unique
identifier of a \hl{Kind} sub-type. The following rules apply:
\begin{itemize}
\item A structural \hl{Type} define the capabilities of a \hl{Kind} sub-type in terms of attributes and \hl{Action}s.
\item A unique structural \hl{Type} MUST be assigned to each and every sub-type
 of \hl{Kind}.
\item A structural \hl{Type} MUST be related, either directly or indirectly, to
 the structural \hl{Type} of \hl{Kind},
 i.e.~\textit{http://schemas.ogf.org/occi/core\#kind}.
 %
 See section~\ref{sec:type_relationship} for the definition of \hl{Type}
 relationship.
\item If type {\bf B} inherit type {\bf A}, where {\bf A} is a sub-type of
 \hl{Kind}, the structural \hl{Type} of {\bf B} MUST be directly related to the
 structural \hl{Type} of {\bf A}.
\end{itemize}

\paragraph*{Non-structural Type} A non-structural \hl{Type} is an instance of
\hl{Type} {\em not} assigned as the unique identifier of any \hl{Kind} sub-type.
The following rules apply:
\begin{itemize}
\item A non-structural \hl{Type} define additional capabilities for each
 \hl{Kind} sub-type instance it is associated with. A non-structural \hl{Type}
 add capabilities using a mix-in like model.
\item A non-structural \hl{Type} MUST only be associated with \hl{Kind}
 sub-type {\em instances}, either at creation-time or run-time.
\item A non-structural \hl{Type} MUST NOT be related, neither directly nor
 indirectly, to the structural \hl{Type} of \hl{Kind},
 i.e.~\textit{http://schemas.ogf.org/occi/core\#kind}.
 %
 See section~\ref{sec:type_relationship} for the definition of \hl{Type}
 relationship.
\item A non-structural \hl{Type} defining no additional capabilities in terms
 of attributes or \hl{Action}s is considered to be a tag.
\end{itemize}

\subsubsection{Category}
\label{sec:category}
The \hl{Category} type comprises the basis of the identification mechanism used
by the OCCI classification system. It MUST be implemented. Instances of the
\hl{Category} type are only used to identify \hl{Action} types. All other uses
of \hl{Category} properties are managed through its sub-type \hl{Type}.
%
Table~\ref{tbl:category} defines the attributes the \hl{Category} type MUST
implement to be compliant.

\mytablefloat{\label{tbl:category}Attributes defined for the \hl{Category} type}{
\begin{tabular}{llllp{2.7in}}
\toprule
Attribute & Type & Multiplicity & Client Mutability & Description \\
\colrule
term & String & 1 & Immutable & Unique identifier of the \hl{Category} instance within the categorisation scheme. \\
scheme & URI & 1 & Immutable & The categorisation scheme. \\
title & String & 0..1 & Immutable & The display name of an instance. \\
attributes & String & 0..* & Immutable & The set of resource attribute names defined by the \hl{Category} instance. \\
\botrule
\end{tabular}
}

A \hl{Category} is uniquely identified by concatenating the categorisation
scheme with the category term,
e.g.~\textit{http://example.com/category/scheme\#term}.
This is done to enable discovery of \hl{Category} definitions in text based
renderings such as HTTP. All renderings MUST make use of and understand
concatenated unique identifiers of \hl{Category} types.
%
Sub-types of \hl{Category} such as \hl{Type} inherit this property.

The categorisation schemes defined in the OCCI specification all use the
\textit{http://schemas.ogf.org/occi/} base URL. This base URL is reserved for
OCCI an MUST NOT be used by domain-specific extensions.

Attribute names defined by \hl{Category} instances%
\footnote{Also applies to \hl{Type} instances.}
use the \texttt{occi.}~prefix.  This prefix is reserved for OCCI and MUST NOT
be used by domain-specific extensions.

\subsubsection{Type relationship}
\label{sec:type_relationship}
As previously defined a structural \hl{Type} MUST be related, either
either directly or indirectly, to the structural \hl{Type} of \hl{Kind},
i.e.~\textit{http://schemas.ogf.org/occi/core\#kind}.
%
The OCCI base types \hl{Resource} and \hl{Link} extend \hl{Kind}.  This
together with any further sub-typing implies a hierarchy of related structural
\hl{Type} instances.  The \hl{Type} relationships thus mirror the type
inheritance structure of the OCCI model and any extension thereof.

In an example where a domain-specific ``Custom Compute Resource'' is a sub-type
the OCCI infrastructure type Compute, which in turn is a sub-type of the
\hl{Resource} type, four related structural \hl{Type}s would be involved.
%
Table~\ref{tbl:relationship} illustrates the exemplified hierarchy of \hl{Type}
instances relating the domain-specific structural \hl{Type} to the structural
\hl{Type} of \hl{Kind}.

\mytablefloat{\label{tbl:relationship}Example of the \hl{Type} relationship involved
for a domain-specific extension of the OCCI infrastructure type \hl{Compute}.}{
\begin{tabular}{p{7cm}p{7cm}}
\toprule
Structural Type & Related Structural Type \\
\colrule
\textit{http://example.com/occi/custom\#compute} & \textit{http://schemas.ogf.org/occi/infrastructure\#compute} \\
\textit{http://schemas.ogf.org/occi/infrastructure\#compute} & \textit{http://schemas.ogf.org/occi/core\#resource} \\
\textit{http://schemas.ogf.org/occi/core\#resource} & \textit{http://schemas.ogf.org/occi/core\#kind} \\
\botrule
\end{tabular}
}

\subsubsection{Type assignment}
\label{sec:type_assignment}
A structural \hl{Type} MUST be statically assigned to each sub-type of
\hl{Kind} defined by an implementation. A \hl{Kind} sub-type instance MUST be
automatically associated with its structural \hl{Type} at creation-time.  The
structural \hl{Type} associated with an instance MUST remain associated with the
instance during its lifetime.

A non-structural \hl{Type}, also known as a mix-in, MAY be associated with a
\hl{Kind} sub-type instance, either at creation-time or run-time. An OCCI
implementation MAY restrict which instances can be associated with a particular
non-structural \hl{Type}.

\subsubsection{Collections}
\label{sec:collection}
One or more \hl{Kind} sub-type instances associated with the same \hl{Type},
may it be structural or non-structural, automatically form a collection.
Each \hl{Type} instance in the system identifies a collection consisting of all
different \hl{Kind} sub-type instances associated with the \hl{Type}.

A \hl{Kind} sub-type instance is always a member of the \hl{Kind}'s structural
\hl{Type} collection since a \hl{Kind} sub-type instance MUST be associated
with the structural \hl{Type} of the \hl{Kind} sub-type.
Since a non-structural \hl{Type} can be assigned to any \hl{Kind} sub-type
instance a collection can contain instances of different \hl{Kind} sub-types.
%
For example, an instance of the \hl{Resource} type will always be associated
to the structural \hl{Type}
\textit{http://scheme.ogf.org/occi/core\#resource} and thus part of the
\hl{Resource} \hl{Type} collection.
\begin{description}
\item[Adding an instance] to a collection is accomplished by associating the
corresponding non-structural \hl{Type} to the \hl{Kind} sub-type instance.
\item[Removing an instance] from a collection is accomplished by disassociating the
corresponding non-structural \hl{Type} from the \hl{Kind} sub-type instance.
\end{description}
%
An OCCI implementation MUST allow a client to navigate collections. The
following basic navigation operations MUST be supported:
\begin{itemize}
\item Retrieve the whole collection.
\item Retrieve a specific item in a collection.
\item Retrieve a subset of a collection.
\end{itemize}
The details of collection navigation is rendering specific.

\subsubsection{Discovery}
\label{sec:discovery}
An OCCI client MUST be able to discover all instances of \hl{Type} and
\hl{Category} a particular service provider's OCCI implementation support. By
examining these instances a client MUST be able to, at a minimum, deduce the
following information:
\begin{itemize}
\item The \hl{Kind} sub-types available from a the service provider, including domain-specific extensions.
\item The attributes associated with each \hl{Kind} sub-type.
\item The invocable operations, i.e. \hl{Action}s, defined for each \hl{Kind} sub-type.
\item Additional mix-ins or tags, i.e. non-structural \hl{Type}s, applicable to
 \hl{Kind} sub-type instances.
\end{itemize}
The above requirements comprise the OCCI discovery mechanism. It MUST be
implemented.
%
The details of exactly how the \hl{Category} and \hl{Type} instances are
exposed to an OCCI client is specific to the particular rendering used.
\marginpar{References?}
The relevant details can be found in the OCCI rendering documents.

%%%%%%%%%%%%%%%%%%%%%%%%%%%%% Bits and pieces %%%%%%%%%%%%%%%%%%%%%%%%%%%%%%%%%%

% Stuff which may or may not be useful to add somewhere in the spec...

\begin{comment}
The central component of the classification system is the \hl{Type} type which
is a specialisation of the \hl{Category} type. The following sections describe
the OCCI classification system in detail.

Each OCCI base type, see section~\ref{sec:base_types}, is assigned a
unique identifier.  This identifier is an instance of \hl{Type} and comprise
the structural \hl{Type} of the associated base type%
\footnote{The \hl{Action} type is an exception and is instead uniquely
identified by a \hl{Category} instance. An \hl{Action} is a capability of \hl{Type}
and can therefore not be identified by \hl{Type}.}.
\end{comment}


\begin{comment}
%%% CLIENT Interaction, do we need to specify this explicitly?!
\subsubsection{Creating Instances}
A client MUST supply the concrete Kind type as a Category. All OCCI
implementations MUST understand these requests containing concrete
Kind types, i.e. Resource, Link, Action or a subtype thereof.

When a client attempts to add a non-structural
\hl{Type} at a stage not supported by a particular provider's OCCI
implementation, the provider MUST notify the client it has issued a bad
request.
\end{comment}

%%%%%%%%%%%%%%%%%%%%%%%%%%%%%%%%%%%%%%%%%%%%%%%%%%%%%%%%%%%%%%%%%%%%%%%%%%%%%%%


\subsection{The OCCI base types}
\label{sec:base_types}
The following sections describe the OCCI base types defined by the OCCI model.
The base types are \hl{Kind}, \hl{Resource}, \hl{Link} and \hl{Action}. All
base types MUST be implemented.

\subsubsection{Kind}
\label{sec:kind}
The \hl{Kind} type is the abstract base type for \hl{Resource} and \hl{Link}
and any domain-specific sub-types thereof. It MUST be implemented.
%
Table~\ref{tbl:kind} defines the attributes the \hl{Kind} type MUST implement to
be compliant.

\mytablefloat{\label{tbl:kind}Attributes defined for the \hl{Kind} type.}{
\begin{tabular}{llllp{7cm}}
\toprule
Attribute & Type & Multiplicity & Client Mutability & Description \\
\colrule
id & URI & 1 & Immutable & A unique identifier (within the service provider's name-space) of the \hl{Kind} sub-type instance. \\
title & String & 0..1 & Mutable & The display name of the instance. \\
structural type & \hl{Type} & 1 & Immutable & The structural \hl{Type} of the instance. \\
non-structural types & \hl{Type} & 0..* & Mutable & The non-structural \hl{Type}s associated to this instance. Consumers can expect the attributes and \hl{Action}s of the associated non-structural \hl{Type}s to be exposed by the instance. \\ 
\botrule
\end{tabular}
}

\hl{Kind} enforces for all sub-types a required \texttt{id} attribute and an
optional \texttt{title} attribute.
%
Every sub-type of \hl{Kind} MUST be assigned a structural \hl{Type}, see
section~\ref{sec:type}. \hl{Kind} itself is assigned the structural \hl{Type}
\textit{http://schemas.ogf.org/occi/core\#kind}.
%
A \hl{Kind} sub-type instance MAY be associated with one or more non-structural
\hl{Type}s.

A \hl{Kind} sub-type instance MUST expose its structural \hl{Type} and any
associated non-structural \hl{Type}s together with their associated attributes
and \hl{Action}s.

\subsubsection{Resource}
\label{sec:resource}
The \hl{Resource} type inherit \hl{Kind} and describes a concrete resource that
can be inspected and manipulated. It represents a general object in the OCCI
model and MUST be implemented.
%
The \hl{Resource} type MUST implement all attributes inherited from the
\hl{Kind} type together with the attributes defined in table~\ref{tbl:resource}
in order to be compliant.
%
The Resource type is assigned the structural \hl{Type}
\textit{http://schemas.ogf.org/occi/core\#resource}.

\mytablefloat{\label{tbl:resource}Attributes defined for the \hl{Resource} type.}{
\begin{tabular}{llllp{8cm}}
\toprule
Attribute & Type & Multiplicity & Client Mutability & Description \\
\colrule
summary & String & 0..1 & Mutable & A summarising description of the \hl{Resource} instance.\\
links & \hl{Link} & 0..* & Mutable & A set of \hl{Link} compositions. Being a composite relation the removal of a \hl{Link} from the set MUST also remove the \hl{Link} instance.\\
\botrule
\end{tabular}
}

\hl{Resource} enforces the inheritance of a set of common attributes into
sub-types. Moreover, it introduces relationships to other \hl{Resource}
instances through instances of the \hl{Link} type.
%
The \hl{Resource} type is the entry point for domain-specific extensions of the
OCCI model, see section~\ref{sec:extensibility}.

\subsubsection{Link}
\label{sec:link}
An instance of the \hl{Link} type defines a base association between two
\hl{Resource} instances. It MUST be implemented. A \hl{Link} instance indicates
that one \hl{Resource} instance is connected to another.
%
The \hl{Link} type MUST implement all attributes inherited from the
\hl{Kind} type together with the attributes defined in table~\ref{tbl:link}
in order to be compliant.
%
The Link type is assigned the structural \hl{Type}
\textit{http://schemas.ogf.org/occi/core\#link}.

\mytablefloat{\label{tbl:link}Attributes defined for the \hl{Link} type.}{
\begin{tabular}{llllp{7.5cm}}
\toprule
Attribute & Type & Multiplicity & Client Mutability & Description \\
\colrule
source & \hl{Resource} & 1 & Mutable & The \hl{Resource} instances the \hl{Link} instance originates from.\\
target & \hl{Resource} & 1 & Mutable & The \hl{Resource} instances the \hl{Link} instance points to.\\
\botrule
\end{tabular}
}

An instance of the \hl{Link} type MUST NOT refer to an external resource.  A
provider MAY however create a sub-type of \hl{Link} with different semantics,
e.g.~have a target attribute containing an URI and thus the ability of linking
with external resources.
%
The \hl{Link} type can be sub-typed for domain-specific extensions of the
OCCI model, see section~\ref{sec:extensibility}.

\subsubsection{Action}
The \hl{Action} type defines an invocable operation applicable to a \hl{Kind}
sub-type instance or a collection thereof. It MUST be implemented. In general,
\hl{Action}s modify state by e.g.~performing a complex operation such as
rebooting a virtual machine.
%
Table~\ref{tbl:action} defines the attributes the \hl{Action} type MUST
implement to be compliant.

\mytablefloat{\label{tbl:action}Attributes defined for the \hl{Action} type.}{
\begin{tabular}{llllp{7.5cm}}
\toprule
Attribute & Type & Multiplicity & Client Mutability & Description \\
\colrule
category & \hl{Category} & 1 & Immutable & The identifying \hl{Category} of the \hl{Action}. \\
parameters & String & 0..* & Immutable & Enumeration of valid parameters for the \hl{Action}. \\
\botrule
\end{tabular}
}

An \hl{Action} is always bound to a \hl{Type} instance through a composite
association. An \hl{Action} is considered a capability of the \hl{Type}.  An
\hl{Action} MAY be invoked on any \hl{Kind} sub-type instance associated with
the \hl{Type} instance defining the \hl{Action}. An OCCI implementation MAY
refuse an \hl{Action} from being invoked if currently not applicable.

An \hl{Action} MAY be invoked on a collection of \hl{Kind} sub-type instances.
The \hl{Action} is only considered valid if all instances of the collection are
associated with the \hl{Type} defining the \hl{Action}.

The Action type is assigned the \hl{Category} identifier
\textit{http://schemas.ogf.org/occi/core\#action}.
%
The \hl{Action} type can be sub-typed for domain-specific extensions of the
OCCI model, see section~\ref{sec:extensibility}.

\subsection{Mutability}
Attributes of an OCCI model type instance, a resource instance, are
either client mutable or client immutable. If an attribute is noted to
be mutable this MUST be interpreted that a client can create a
resource instance that is parametrised by the attribute. Likewise, if
an attribute is mutable, a client can update that resource instance's
mutable attribute value and the server side MUST support this. If an
attribute is marked as immutable, it indicates that the server side
implementation MUST manage these exclusively. Immutable attributes
MUST NOT be modifiable by clients under any circumstance.

\subsection{Extensibility}
\label{sec:extensibility}
The OCCI model has a flexible yet fairly simple extension mechanism based on
the classification system described in section~\ref{sec:classification}.
%
The OCCI model can be extended using two different methods, sub-typing and
mix-in. Both methods involve the use of domain-specific \hl{Type} or
\hl{Category} instances. The following sections define the requirements for
extensions of the OCCI model.
%
The rules defined in section~\ref{sec:classification} and \ref{sec:base_types}
are REQUIRED for all extensions of the OCCI model.

\subsubsection{\hl{Type} and \hl{Category} instances}
\label{sec:ext:category}
Domain-specific \hl{Type} and \hl{Category} instances MAY be introduced by an
OCCI implementation.
%
A \hl{Type} or \hl{Category} instance defined outside of the OCCI
specification MUST use a categorisation scheme unique to the provider,
e.g.~\textit{http://example.com/occi\#}.
%
An attribute introduced by a domain-specific \hl{Type} or \hl{Category} MUST
use an attribute name prefix. This prefix MUST NOT be the ``\texttt{occi.}''~prefix
which is reserved for the OCCI specification. Domain-specific attribute names
SHOULD use a prefix consisting of the provider's reverse domain name,
e.g.~``\texttt{com.example.}''.

\subsubsection{Sub-typing}
The OCCI model MAY be extended through sub-typing for domain-specific purposes.
Three OCCI model types MAY be sub-typed, those are \hl{Resource}, \hl{Link} and
\hl{Action}.

In order to define a sub-type of \hl{Resource} or \hl{Link} a domain-specific
structural \hl{Type} MUST be defined and assigned to the sub-type. This
structural \hl{Type} MUST be directly related to the structural \hl{Type} of
the type extended.

In order to define a sub-type of \hl{Action} a domain-specific \hl{Category}
instance MUST be assigned to the \hl{Action} sub-type as its unique identifier.
Furthermore the \hl{Action} sub-type MUST be associated as a capability of a
domain-specific \hl{Type} instance.

\subsubsection{Mix-ins}
The OCCI model MAY be extended through domain-specific mix-ins,
i.e.~non-structural \hl{Type}s.  A non-structural \hl{Type} MAY be associated
with any \hl{Kind} sub-type instance although a provider MAY apply
restrictions.

In order to support user-defined tags an OCCI implementation must allow
non-structural \hl{Type}s to be created and destroyed by request of a client.
There is no limitation in the OCCI model from doing so but it is RECOMMENDED to
assign a separate categorisation scheme for each user's non-structural
\hl{Type}s%
\footnote{A tag is a non-structural \hl{Type} which do not introduce additional
capabilities.}.

\section{Contributors}
Editors: Andy Edmonds, Thijs Metsch, Ralf Nyrén \\
Contributors: Alexander Papaspyrou, Sam Johnston

\textbf{TBD: Bunch op people missing here - create table\ldots}

\section{Glossary}
\label{sec:glossary}
\begin{tabular}{l|p{11cm}}
Term & Description \\
\hline
\hl{Action} & An OCCI base type. Represent an invocable operation on a \hl{Entity} sub-type instance or collection thereof. \\
\hl{Category} & A type in the OCCI model. The parent type of \hl{Kind}. \\
Client & An OCCI client.\\
Collection & A set of \hl{Entity} sub-type instances all associated to a particular \hl{Kind} instance. \\
\hl{Entity} & An OCCI base type. The parent type of \hl{Resource} and \hl{Link}. \\
\hl{Kind} & A type in the OCCI model. The central piece in the OCCI classification system. \\
\hl{Link} & An OCCI base type. A \hl{Link} instance associate one \hl{Resource} instance with another. \\
Mix-in & A non-structural \hl{Kind}. The ``mix-in like'' concept in OCCI only
  support binding of new attributes and \hl{Action}s at run-time. A
  non-structural \hl{Kind} can only be associated with an {\em instance} of an
  \hl{Entity} sub-type. \\
Non-structural \hl{Kind} & An instance of \hl{Kind} {\em not} used as an unique identifier of an OCCI base type. \\
OCCI & Open Cloud Computing Interface \\
OCCI base type & One of \hl{Entity}, \hl{Resource}, \hl{Link} or \hl{Action}. \\
OGF & Open Grid Forum \\
\hl{Resource} & An OCCI base type. The parent type for all domain-specific resource types. \\
Structural \hl{Kind} & An instance of \hl{Kind} assigned as the unique identifier of an OCCI base type. \\
Tag & A non-structural \hl{Kind} with no attributes or actions defined. \\
URI & Uniform Resource Identifier \\
URL & Uniform Resource Locator \\
URN & Uniform Resource Name \\
\end{tabular}



\section{Intellectual Property Statement}
The OGF takes no position regarding the validity or scope of any
intellectual property or other rights that might be claimed to pertain
to the implementation or use of the technology described in this
document or the extent to which any license under such rights might or
might not be available; neither does it represent that it has made any
effort to identify any such rights. Copies of claims of rights made
available for publication and any assurances of licenses to be made
available, or the result of an attempt made to obtain a general
license or permission for the use of such proprietary rights by
implementers or users of this specification can be obtained from the
OGF Secretariat.

The OGF invites any interested party to bring to its attention any
copyrights, patents or patent applications, or other proprietary
rights which may cover technology that may be required to practice
this recommendation. Please address the information to the OGF
Executive Director.


\section{Disclaimer}
\input{include/disclaimer}

\section{Full Copyright Notice}
Copyright \copyright ~Open Grid Forum (2009-2016). All Rights Reserved.

This document and translations of it may be copied and furnished to
others, and derivative works that comment on or otherwise explain it
or assist in its implementation may be prepared, copied, published and
distributed, in whole or in part, without restriction of any kind,
provided that the above copyright notice and this paragraph are
included on all such copies and derivative works. However, this
document itself may not be modified in any way, such as by removing
the copyright notice or references to the OGF or other organizations,
except as needed for the purpose of developing Grid Recommendations in
which case the procedures for copyrights defined in the OGF Document
process must be followed, or as required to translate it into
languages other than English.

The limited permissions granted above are perpetual and will not be
revoked by the OGF or its successors or assignees.


\section{References}

Note that only permanent documents should be cited as
references. Other items, such as Web pages or working groups, should
be cited inline (i.e., see the Open Grid Forum,
http://www.ogf.org). References should conform to a standard such as
used by IEEE/ACM, MLA, Chicago or similar. Include an author, year,
title, publisher, place of publication. For online materials, also add
a URL. It is acceptable to separate out ''normative references,'' as
IETF documents typically do. Some sample citations:

\end{document}
