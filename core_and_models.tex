\documentclass[10pt,a4paper]{article}
\usepackage{fullpage}
\usepackage{graphicx}
\usepackage{fancyhdr}
\setlength{\headheight}{13pt}
\pagestyle{fancy}

% default sans-serif
\renewcommand{\familydefault}{\sfdefault}

% no lines for headers and footers
\renewcommand{\headrulewidth}{0pt}
\renewcommand{\footrulewidth}{0pt}

% header
\fancyhf{}
\lhead{GWD-R}
\rhead{\today}

% footer
\lfoot{occi-wg@ogf.org}
\rfoot{\thepage}

% paragraphs need some space...
\setlength{\parindent}{0pt}
\setlength{\parskip}{1ex plus 0.5ex minus 0.2ex}

% some space between header and text...
\headsep 13pt

\setcounter{secnumdepth}{4}

\begin{document}

% header on first page is different
\thispagestyle{empty}

GWD-R \hfill  Thijs Metsch, Platform Computing\\
OCCI-WG \hfill  Andy Edmonds, Intel\\
\rightline {October 14, 2010}\\
\rightline {Updated: \today}

\vspace*{0.5in}

\begin{Large}
\textbf{Open Cloud Computing Interface - Core and Models}
\end{Large}

\vspace*{0.5in}

\underline{Status of this Document}

This document provides information to the community regarding the
specification of the Open Cloud Computing Interface. Distribution is
unlimited.

\underline{Obsoletes}

This document obsoletes GFD-xxx [REFERENCE].

\underline{Copyright Notice}

Copyright \copyright Open Grid Forum (2009-2010). All Rights Reserved.

\underline{Trademarks}

OCCI is a trademark of the Open Grid Forum.

\underline{Abstract}

This document, part of a document series, produced by the OCCI working
group within the Open Grid Forum (OGF), provides a high-level
definition of a Protocol and API. The document is based upon
previously gathered requirements and focuses on the scope of important
capabilities required to support modern service offerings.

\newpage
\tableofcontents
\newpage

\section{Introduction}
The Open Cloud Computing Interface (OCCI) is a RESTful Protocol (and
API) \marginpar{Probably no need for the brackets} for all kinds of
Management tasks. OCCI was originally initiated to create a remote
management API for IaaS model based Services, allowing for the
development of interoperable tools for common tasks including
deployment, autonomic scaling and monitoring. It now can be used to
severe other models as well. To be modular and extensible the current
specification itself is currently split into three complimentary
documents:

\begin{itemize}
\item Core - this defines the OCCI model
\item HTTP Rendering - this defines how to manipulate the core model
  using the OCCI RESTful API. The document defines how the OCCI model
  can be communicated and thus serialized using HTTP.
\item Infrastructure - this defines the infrastructure domain resource
  types, the required attributes for each and the actions that can be
  taken on each.
\end{itemize}

\section{Notational Conventions}
All these parts and the information within are mandatory for
implementors (unless otherwise specified). The key words "MUST", "MUST
NOT", "REQUIRED", "SHALL", "SHALL NOT", "SHOULD", "SHOULD NOT",
"RECOMMENDED", "MAY", and "OPTIONAL" in this document are to be
interpreted as described in RFC 2119.

\section{OCCI model}
OCCI is a boundary protocol/API \marginpar{Better as protocol and API}
that acts as a service front-end to a provider’s internal management
framework.  Diagram \ref{fig:placement} shows OCCI's place in a
provider’s architecture:

\begin{figure}[!hp]
	\centering
	\includegraphics[scale=0.5]{figs/occi-intro.png}
	\caption{OCCI Overview}
	\label{fig:placement}
\end{figure}

The core of the OCCI model is simple. The central resource type is
Resource and contains a number of common attributes that
domain-specific Resource types inherit. These Resource types are
complemented by two additional types, Link and Action. Accompanying
these, Kind and Category provide the fabric for a safe extension model
towards domain-specific usage. For compliance with OCCI Core, all
types MUST be implemented. The UML class diagram \ref{fig:occi_core}
gives an overview of the model.

\clearpage
\begin{figure}[!h]
\centering \includegraphics[scale=0.5]{figs/core_model.png}
\caption{OCCI Overview}
\label{fig:occi_core}
\end{figure}

The following sections of the specification define the foundations of
the OCCI model, including common resource types, linkages and
attributes.

\subsection{The base types}
The following sections describe the base types of the OCCI core
model. The base types are Kind, Category, Resource, Link and
Action. All base types MUST be implemented.

\subsubsection{Kind}
The Kind type is an abstract base type and common to all resources in
the OCCI context. It MUST be implemented. Kind enforces for all
subtypes a required id attribute and an (optional) title
attribute. More importantly Kind introduces one or more type mix-ins
through the type Category (see section \ref{sec:type_system}). The
following table defines the attributes the Kind type MUST implement to
be compliant:

\begin{tabular}{l|l|l|l|p{2.7in}}
Attribute & Type & Multiplicity & Client Mutability & Description \\
\hline
id & URI & 1 & Immutable & Denotes a unique (within the service provider's name-space) identifier of a Kind subtype instance. \\
title & String & 0..1 & Mutable & Denotes the display name of an instance. \\
categories & Category & 1..* & Mutable\footnotemark[1] & Comprises the Category types associated to this instance. Consumers can expect the attributes and actions of the associated Category types to be exposed by the instance. \\ 
\end{tabular}
\footnotetext[1]{Only tag Category types are client mutable.}
\addtocounter{footnote}{1}

\subsubsection{Category}
The Category type represent the classification mechanism used by
OCCI. It MUST be implemented. From a system point of view a Category
is used for two different classification purposes. See also section
\ref{sec:type_system} ``Type System'' and section \ref{sec:collection}
``Collections''.

\begin{description}
\item[Taxonomy] A Category is used to assign static type information
  to each resource type inheriting Kind or a descendant of Kind. This
  use of Category types denotes the OCCI "static type system". A
  unique Category MUST be assigned to every descendant of Kind. See
  the section \ref{sec:hierarchy} "Hierarchy".
\item[Folksonomy] A Category can be used to assign tags to resource
  instances via a mix-in like model. A Category mix-in MUST NOT be
  related to an OCCI base type (or any descendant of Kind) and MUST
  NOT be the unique identifier thereof.  Example use cases are
  collections, location information and templates for virtual machine
  provisioning.
\end{description}

The following table defines the attributes the Category type MUST
implement to be compliant:

\begin{tabular}{l|l|l|l|p{2.7in}}
Attribute & Type & Multiplicity & Client Mutability & Description \\
\hline
term & String & 1 & Immutable & The category to which the resource belongs. \\
scheme & URI & 1 & Immutable & The categorization scheme. \\
title & String & 0..1 & Immutable & Denotes the display name of an instance. \\
attributes & String & 0..* & Immutable & Comprise the resource attributes defined by the Category. \\
related & Category & 0..* & Immutable & Set of related Category types. \\
\end{tabular}

A Category is uniquely identified by concatenating the categorization
scheme with the category term,
e.g. \textit{http://example.com/category/scheme\#term}.  This is done
to enable discovery of Category definitions per HTTP. All renderings
MUST make use and understand concatenated unique identifiers of
Category types.

\paragraph{Hierarchy}
\label{sec:hierarchy}
To be a part of the "static type system" a Category MUST be related,
either directly or indirectly, to a base type Category. Extension of
the OCCI base types through subtyping thus implies a hierarchy of
related Category types.

In an example where "Custom-Compute-Resource" is a subtype of
"Compute-Resource", which in turn is a subtype of the Resource base
type, three related Category types would be involved. The following
table illustrates the exemplified hierarchy of Category types relating
the custom Category to a base type Category.

Example:

\begin{tabular}{p{0.6in}|p{3.1in}|p{2in}}
Kind & Category & Related Category \\
\hline
Custom-Compute-Resource & \textit{http://example.com/occi/custom\#compute} & \textit{http://schemas.ogf.org/occi/infrastructure\#compute} \\
Compute-Resource & \textit{http://schemas.ogf.org/occi/infrastructure\#compute} & \textit{http://schemas.ogf.org/occi/core\#resource} \\
Resource & \textit{http://schemas.ogf.org/occi/core\#resource} & none \\
\end{tabular}

\subsubsection{Resource}
The Resource type describes a concrete resource that can be inspected
and manipulated. It represents a general object in the OCCI core model
and MUST be implemented. Resource enforces the inheritance of a set of
common attributes into subtypes (such as a unique identifier and an
optional descriptive summary). Moreover, it introduces relationships
to other Kind instances, denoted by Links. It also introduces
operations that can be invoked on this instance, denoted by Actions.

The Resource type is assigned the
\textit{http://schemas.ogf.org/occi/core\#resource} Category.

\begin{itemize}
  \item A Resource instance MUST at least expose this or a related
    Category together with any associated attributes.
  \item A Resource instance MUST advertise those of its Actions, if
    any, which are currently applicable.
  \item A Resource instance MUST implement the attributes in the
    following table:
\end{itemize}

\begin{tabular}{l|l|l|l|p{2.7in}}
Attribute & Type & Multiplicity & Client Mutability & Description \\
\hline
summary & String & 0..1 & Mutable & Holds a summarizing description of the Resource instance.\\
links & Link & 0..* & Mutable & Comprises a set of Link compositions. Being a composite relation the removal of a Link from the set MUST also remove the Link instance.\\
actions & Action & 0..* & Immutable & Set of Actions associated with the Resource.\\
\end{tabular}

The OCCI Extension documents define sets of specific subtypes of
Resources, Links and Actions. They inherit Resource and is each
assigned an unique Category.

\subsubsection{Link}
The Link type defines a base association between two Resources. It
MUST be implemented. Link indicates that one resource is connected to
another.

The Link type is assigned the
\textit{http://schemas.ogf.org/occi/core\#link Category}. A Link
instance MUST at least expose this or a related Category together with
any associated attributes. A Link instance MUST also implement the
attributes in the following table:

\begin{tabular}{l|l|l|l|p{2.7in}}
Attribute & Type & Multiplicity & Client Mutability & Description \\
\hline
source & Resource & 1 & Mutable & Denotes the resource the Link instance originates from.\\
target & Resource & 1 & Mutable & Denotes the resource the Link instance points to.\\
\end{tabular}

The Link base type MUST NOT refer to an external resource. A provider
MAY however create a subtype of Link with different semantics that
e.g. have a target attribute containing an URI and thus the ability of
linking with external resources.

\subsubsection{Action}
The Action type defines an invocable operation on an associated
Resource. It MUST be implemented. In general, Actions modify state
(e.g. by performing a complex operation such as rebooting a virtual
machine).

The Action type is assigned the
\textit{http://schemas.ogf.org/occi/core\#action} Category. An Action
instance MUST at least expose this or a related Category together with
any associated attributes.

\subsection{Mutability}
\marginpar{Does 'client access' need to be discussed here?}
Attributes of a OCCI base type instance, a resource instance, are
either client mutable or client immutable. If an attribute is noted to
be mutable this MUST be interpreted that a client can create a
resource instance that is parametrized by the attribute. Likewise, if
an attribute is mutable, a client can update that resource instance's
mutable attribute value and the server side MUST support this. If an
attribute is marked as immutable, it indicates that the server side
implementation MUST manage these exclusively. Immutable attributes
MUST NOT be modifiable by clients under any circumstance.

\subsection{Type System}
\label{sec:type_system}
The Category type is the classification mechanism used by OCCI. This
tells service providers what type of concrete OCCI Kind should be
created. Each domain-specific section of the OCCI specification
(e.g. infrastructure) defines its own concrete Resource, Action and
optionally Link specializations. It should be noted that depending on
a client's credentials, access to creating certain OCCI Kind instances
MAY be restricted.

The type system is a flexible and extensible one allowing for the
addition of "mix-in" Category types. Mix-in Category types are ones
that add additional behavior to OCCI Kinds and are Category types not
defined by the OCCI specifications. The mix-in Category types can be
added to an OCCI Kind both at Kind creation-time and run-time. The
Kinds to which a particular mix-in Category type can be applied, is
defined by the provider implementation. At which time, creation- or
run-time, these mix-in types can be applied is also defined by the
provider implementation. Discovery of mix-in types can be achieved by
querying the OCCI Query interface. Mix-in Category types are those
that are NOT related to an OCCI base type Category and MUST be
implemented as such. When a client attempts to add a mix-in at a stage
not supported by the provider, the provider MUST notify the client
that it has issued a bad request.

\subsubsection{Creating Instances}
A client MUST supply the concrete Kind type as a Category. All OCCI
implementations MUST understand these requests containing concrete
Kind types, i.e. Resource, Link, Action or a subtype thereof.

\subsection{Collections}
\label{sec:collection}
One or more Kind sub-type instances assigned to the same Category
automatically form a collection. Each Category in the system
identifies a collection consisting of all different Kind instances
associated with the Category.

A Kind sub-type instance is always a member of the Kind's identifying
Category collection since a Kind instance MUST be assigned the Kind
sub-type Category.  Since a tag Category can be assigned to any Kind
instance a collection can contain instances of different Kind
sub-types.

For example an instance of the Resource type will be assigned the
\textit{http://scheme.ogf.org/occi/core\#resource} Category and thus
part of the Resource Category collection.

\subsection{Extensibility}
The OCCI Core model has a flexible yet fairly simple extension
mechanism through the use of sub-typing and Category types.

\subsubsection{Sub-types and Category Types}
The OCCI Core model has a flexible yet fairly simple extension
mechanism through the use of sub-typing and Category types.  A
provider MAY extend the OCCI core model by creating a subtype of
either an OCCI Core base type or a provider defined subtype.

A provider MUST assign a new unique Category to each new subtype
defined. The new Category MUST be, either directly or indirectly,
related to one of the Core base types, i.e. Resource, Link or
Action. If a provider extends a subtype the new Category SHOULD be
related to the subtype extended, provided the previous rule holds
true.

Providers MAY also define new Category types for tagging and
templating purposes. Such a Category MUST NOT be the unique identifier
of a resource type, i.e. must not be part of the "static type system".

The scheme of provider-defined Category types MUST reside in a unique
namespace different from those in the OCCI specification. Any
attributes defined by a new Category MUST NOT have a name starting
with occi. The occi namespace prefix is reserved for extensions which
are standardized. A provider SHOULD use an attribute namespace
comprising the reverse domain name as prefix to every attribute name.

A provider-defined Category identifying a custom subtype of Resource
could be exemplified as follows. The term and scheme is defined by the
provider. The related set MUST include a, direct or indirect (through
a hierarchy), reference to the Resource base type category.

\begin{itemize}
\item term = custom\_resource.
\item scheme = \textit{http://example.com/occi/resource\#}
\item related = \textit{http://schemas.ogf.org/occi/core\#resource}
\end{itemize}

\subsubsection{Extension of the base types}
A provider MAY define subtypes of the Core base types Resource, Link
and Action for domain-specific purposes. A unique Category MUST be
assigned to each new subtype. The assigned Category MUST be directly
related to the Category of the base type extended.

\subsubsection{Extension of subtypes}
A provider MAY define subtypes of any existing subtype, either defined
in OCCI or by someone else. A unique Category MUST be assigned to each
new subtype. The assigned Category MUST be either directly or
indirectly related to a base type Category. It is RECOMMENDED the
Category is related to the Category of the subtype extended.

\subsection{Discovery}
An OCCI client MUST be able to discover all Category types a
particular provider supports before interacting with the
provider. Since the rendering of the Category types is depended of the
protocol the rendering documents give more information on how
information is send and can be parsed.

\section{Contributors}
Editors: Andy Edmonds, Thijs Metsch \\
Contributors: Alexander Papaspyrou, Ralf Nyrén, Sam Johnston

\textbf{TBD: Bunch op people missing here - create table\ldots}

\section{Glossary}

\begin{tabular}{l|l}
Term & Description \\
\hline
OCCI & Open Cloud Computing Interface \\
\end{tabular}

\section{Intellectual Property Statement}
The OGF takes no position regarding the validity or scope of any
intellectual property or other rights that might be claimed to pertain
to the implementation or use of the technology described in this
document or the extent to which any license under such rights might or
might not be available; neither does it represent that it has made any
effort to identify any such rights. Copies of claims of rights made
available for publication and any assurances of licenses to be made
available, or the result of an attempt made to obtain a general
license or permission for the use of such proprietary rights by
implementers or users of this specification can be obtained from the
OGF Secretariat.

The OGF invites any interested party to bring to its attention any
copyrights, patents or patent applications, or other proprietary
rights which may cover technology that may be required to practice
this recommendation. Please address the information to the OGF
Executive Director.

\section{Disclaimer}
This document and the information contained herein is provided on an
''As Is'' basis and the OGF disclaims all warranties, express or
implied, including but not limited to any warranty that the use of the
information herein will not infringe any rights or any implied
warranties of merchantability or fitness for a particular purpose.

\section{Full Copyright Notice}
Copyright \copyright Open Grid Forum (2009-2010). All Rights Reserved.

This document and translations of it may be copied and furnished to
others, and derivative works that comment on or otherwise explain it
or assist in its implementation may be prepared, copied, published and
distributed, in whole or in part, without restriction of any kind,
provided that the above copyright notice and this paragraph are
included on all such copies and derivative works. However, this
document itself may not be modified in any way, such as by removing
the copyright notice or references to the OGF or other organizations,
except as needed for the purpose of developing Grid Recommendations in
which case the procedures for copyrights defined in the OGF Document
process must be followed, or as required to translate it into
languages other than English.

The limited permissions granted above are perpetual and will not be
revoked by the OGF or its successors or assignees.

\section{References}

Note that only permanent documents should be cited as
references. Other items, such as Web pages or working groups, should
be cited inline (i.e., see the Open Grid Forum,
http://www.ogf.org). References should conform to a standard such as
used by IEEE/ACM, MLA, Chicago or similar. Include an author, year,
title, publisher, place of publication. For online materials, also add
a URL. It is acceptable to separate out ''normative references,'' as
IETF documents typically do. Some sample citations:

\end{document}
