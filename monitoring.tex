\documentclass[10pt,a4paper]{article}
\usepackage[utf8]{inputenc}
\usepackage{fullpage}
\usepackage{graphicx}
\usepackage{fancyhdr}
\usepackage{occi}
\setlength{\headheight}{13pt}
\pagestyle{fancy}

\newcommand{\doccode}{XXXXX}

% default sans-serif
\renewcommand{\familydefault}{\sfdefault}

% no lines for headers and footers
\renewcommand{\headrulewidth}{0pt}
\renewcommand{\footrulewidth}{0pt}

% header
\fancyhf{}
\lhead{\doccode}
\rhead{\today}

% footer
\lfoot{occi-wg@ogf.org}
\rfoot{\thepage}

% paragraphs need some space...
\setlength{\parindent}{0pt}
\setlength{\parskip}{1ex plus 0.5ex minus 0.2ex}

% some space between header and text...
\headsep 13pt

\setcounter{secnumdepth}{4}

\begin{document}

% header on first page is different
\thispagestyle{empty}

\doccode \hfill Augusto Ciuffoletti, Università di Pisa\\ 
OCCI-WG\\
\rightline {September 22, 2014}\\
\rightline {Updated: \today}

\vspace*{0.5in}

\begin{Large}
\textbf{Open Cloud Computing Interface - Monitoring Extension}
\end{Large}

\vspace*{0.5in}

\underline{Status of this Document}

This document provides information to the community regarding the
specification of the Open Cloud Computing Interface. Distribution is
unlimited.

\underline{Copyright Notice}

Copyright \copyright ~Open Grid Forum (2009-2011). All Rights Reserved.

\underline{Trademarks}

OCCI is a trademark of the Open Grid Forum.

\underline{Abstract}

This document, part of a document series, produced by the OCCI working
group within the Open Grid Forum (OGF), provides a high-level
definition of a Protocol and API. The document is based upon
previously gathered requirements and focuses on the scope of important
capabilities required to support modern service offerings.


\newpage
\tableofcontents
\newpage

\section{Introduction}
The Open Cloud Computing Interface (OCCI) is a RESTful Protocol and
API for all kinds of Management tasks. OCCI was originally initiated
to create a remote management API for IaaS%
\footnote{Infrastructure as a Service}
model based Services, allowing for the development of interoperable tools for
common tasks including deployment, autonomic scaling and monitoring.
%
It has since evolved into an flexible API with a strong focus on
interoperability while still offering a high degree of extensibility. The
current release of the Open Cloud Computing Interface is suitable to serve many
other models in addition to IaaS, including e.g.~PaaS and SaaS.

In order to be modular and extensible the current OCCI specification is
released as a suite of complimentary documents which together form the complete
specification.
%
The documents are divided into three categories consisting of the OCCI Core,
the OCCI Renderings and the OCCI Extensions.
%
\begin{itemize}
\item The OCCI Core specification consist of a single document defining the
 OCCI Core Model. The OCCI Core Model can be interacted with {\em
 renderings} (including associated behaviours) and expanded through {\em extensions}.
\item The OCCI Rendering specifications consist of multiple documents each
 describing a particular rendering of the OCCI Core Model. Multiple renderings can
 interact with the same instance of the OCCI Core Model and will automatically support
 any additions to the model which follow the extension rules defined in OCCI
 Core.
\item The OCCI Extension specifications consist of multiple documents each
 describing a particular extension of the OCCI Core Model. The extension documents
 describe additions to the OCCI Core Model defined within the OCCI specification
 suite.
\end{itemize}
%
The current specification consist of three documents.
Future releases of OCCI may include additional rendering and extension
specifications. The documents of the current OCCI specification suite are:

\begin{description}
\item[OCCI Core] describes the formal definition of the the OCCI Core Model
\cite{occi:core}.
\item[OCCI HTTP Rendering] defines how to interact with the OCCI Core Model using the
RESTful OCCI API \cite{occi:http_rendering}. The document defines how the OCCI Core Model can
be communicated and thus serialised using the HTTP protocol.
\item[OCCI Infrastructure] contains the definition of the OCCI Infrastructure
extension for the IaaS domain \cite{occi:infrastructure}. The document defines
additional resource types, their attributes and the actions that can be taken
on each resource type.
\end{description}


\section{Notational conventions}
All these parts and the information within are mandatory for
implementors (unless otherwise specified). The key words "MUST", "MUST
NOT", "REQUIRED", "SHALL", "SHALL NOT", "SHOULD", "SHOULD NOT",
"RECOMMENDED", "MAY", and "OPTIONAL" in this document are to be
interpreted as described in RFC 2119 \cite{rfc2119}.


\newcommand{\smx}{{\bf Notify}}
\newcommand{\ntfl}{{\bf Notification}}
\newcommand{\sens}{{\bf Sensor}}

\section{Motivations}

The provider may want to give the user the tools to undertake actions based on the functional parameters of a resource performance: for instance, depending on how many jobs are running, on how much storage is available, or on the current state of a WAN link.

These actions, as well as the collected functional parameters, are  primarily controlled by the provider, and so is the the way in which they are processed in order to implement the service.

This document defines an OCCI Resource subtype, the {\bf Sensor}, that is in charge of implementing these kinds of activities.

\section{OCCI monitoring}

Resource monitoring is one component of the service the provider gives to the user: so it is naturally configured as an OCCI-resource. We define the {\em Sensor} as the Resource subtype for this purpose.

Among the distinguishing capabilities of a monitoring activity is its relationship with time: it may be limited in time, and exhibit real-time properties.

Other capabilities that are specific for a given monitoring service are added as mixins of the \sens.

A \sens\ instance is the target of {\em Notification} links from the resources it monitors.

%\begin{figure}[!h]
%	{\centering \resizebox*{0.9\columnwidth}{!}{\rotatebox{270}
%	    {\includegraphics{figs/infrastructure_model.pdf}}} \par}
%	\caption{Overview Diagram of OCCI Infrastructure Types.}
%	\label{fig:infra_uml}
%\end{figure}


%\subsection{The {\bf Sensor} resource}

\mytablefloat{
	\label{tbl:smx}The immutable model attributes of the \sens\ resource.
	The base URL {\bf http://schemas.ogf.org/occi} has been replaced with
	{\bf $<$schema$>$} in this table for a better reading experience. 
	} {
	\begin{tabular}{lllllll}
	\toprule
	Term & Scheme & Title & Attributes & Actions & Parent \\
	\colrule
	\sens &  $<$schema$>$/monitoring\# & \sens\ Resource
	& see Table \ref{tbl:sensor} & \{\} & \{\} & $<$schema$>$/core\#Resource \\
	\botrule
	\end{tabular}
}

\mytablefloat{
	\label{tbl:sensor}\hl{Attribute}s defined for the \sens\ type. 
}
{
	\begin{tabular}{lp{2.5cm}p{1cm}lp{6cm}}
	\toprule
	Attribute&Type&Multi\-plicity&Mutability&Description\\
	\colrule
occi.sensor.period & number & 0..1 & true & The time between two following measurements \\
occi.sensor.accuracy & number & 1 & false & of period or delay \\
occi.sensor.timebase & number & 0..1 & true & The server time when the timestart and timestop are modified \\  
occi.sensor.timestart &	number & 0..1 & true & The delay after which the session is planned to start \\
occi.sensor.timestop & number	& 0..1 & true & The delay after which the session is planned to stop \\
	\botrule
	\end{tabular}
}

\section{Application notes and an example}

The sensor is characterized mainly by timing attributes. The only atttribute that MUST be defined is the real-time {\em accuracy} of the monitoring activity. Other non mandatory attributes describe how frequently the monitoring activity is performed, and the time lapse during which the monitoring activity takes place.

The accuracy of the timing represents the maximum distance between an event, and its observation from the \sens.

The \sens\ gathers the functional parameters for its activity using data that is received from resources: therefore, the \sens\ SHOULD be the target of {\em Notification} links, as defined in document \cite{occi:notification}.

The operation of the sensor is defined by mixins: the provider offers the user a choice of mixins, thus allowing the user to inspect the performance of the provision.

The provider uses the capabilities indicated in the mixins associated with a \sens\ instance to implement the {\bf notification} link and to instrument the Resource at the source of the {\bf Notification} link.

For instance, consider the following simple example, where the mixin that specifies the operation of the \sens\ is a simple tag:

\begin{quote}
A provider offers a {\em 3-out-of-k} mixin that can be associated to a \sens\ Resource, that keeps in the {\em active} state 3 of the Compute resources from which it receives notifications.

The user that wants to take advantage of this service instantiates a \sens\ and associates with it a {\em 3-out-of-k} mixin. Next it associated an \smx\ to each of the $k$ Compute resources $C_{1..k}$. For each of them a \ntfl\ link $N_i$ is instantiated that originates from $C_i$ and targets the \sens.
\end{quote}

The schema is portable across any platform that offers the OCCI-infrastructure, OCCI-notification, and OCCI-monitoring, and that provides a {\em 3-out-of-k} mixin.

Distinct providers may interoperate, for instance if one provides the  {\em 3-out-of-k} Resource and another the Compute resources, provided that an agreement exists between the two that allows cross provider information transfer.

\section{Security issues}
The OCCI Notification specification is an extension to the OCCI Core
and Model specification \cite{occi:core}; thus the same security
considerations as for the OCCI Core and Model specification apply
here.

\section{Glossary}
\label{sec:glossary}
\begin{tabular}{l|p{11cm}}
Term & Description \\
\hline
\hl{Action} & An OCCI base type. Represent an invocable operation on a \hl{Entity} sub-type instance or collection thereof. \\
\hl{Category} & A type in the OCCI model. The parent type of \hl{Kind}. \\
Client & An OCCI client.\\
Collection & A set of \hl{Entity} sub-type instances all associated to a particular \hl{Kind} instance. \\
\hl{Entity} & An OCCI base type. The parent type of \hl{Resource} and \hl{Link}. \\
\hl{Kind} & A type in the OCCI model. The central piece in the OCCI classification system. \\
\hl{Link} & An OCCI base type. A \hl{Link} instance associate one \hl{Resource} instance with another. \\
Mix-in & A non-structural \hl{Kind}. The ``mix-in like'' concept in OCCI only
  support binding of new attributes and \hl{Action}s at run-time. A
  non-structural \hl{Kind} can only be associated with an {\em instance} of an
  \hl{Entity} sub-type. \\
Non-structural \hl{Kind} & An instance of \hl{Kind} {\em not} used as an unique identifier of an OCCI base type. \\
OCCI & Open Cloud Computing Interface \\
OCCI base type & One of \hl{Entity}, \hl{Resource}, \hl{Link} or \hl{Action}. \\
OGF & Open Grid Forum \\
\hl{Resource} & An OCCI base type. The parent type for all domain-specific resource types. \\
Structural \hl{Kind} & An instance of \hl{Kind} assigned as the unique identifier of an OCCI base type. \\
Tag & A non-structural \hl{Kind} with no attributes or actions defined. \\
URI & Uniform Resource Identifier \\
URL & Uniform Resource Locator \\
URN & Uniform Resource Name \\
\end{tabular}

 
%\section{Contributors}
%
We would like to thank the following people who contributed to this
document:

\begin{tabular}{l|p{2in}|p{2in}}
Name & Affiliation & Contact \\
\hline
Michael Behrens & R2AD & behrens.cloud at r2ad.com \\
Mark Carlson & Oracle & mark.carlson at oracle.com \\
Andy Edmonds & Intel - SLA@SOI project & andy at edmonds.be \\
Sam Johnston & Google & samj at samj.net \\
Gary Mazzaferro & OCCI Counselour - AlloyCloud, Inc. &  garymazzaferro at gmail.com \\ 
Thijs Metsch & Platform Computing, Sun Microsystems & tmetsch at platform.com \\
Ralf Nyrén & Aurenav & ralf at nyren.net \\
Alexander Papaspyrou & TU Dortmund University & alexander.papaspyrou at tu\-dortmund.de \\
Alexis Richardson & RabbitMQ & alexis at rabbitmq.com \\
Shlomo Swidler & Orchestratus & shlomo.swidler at orchestratus.com \\
Florian Feldhaus & GWDG & florian.feldhaus at gwdg.de \\
Jean Parpaillon & & jean.parpaillon at free.fr \\
\end{tabular}

Next to these individual contributions we value the contributions from
the OCCI working group.


\section{Intellectual Property Statement}
The OGF takes no position regarding the validity or scope of any
intellectual property or other rights that might be claimed to pertain
to the implementation or use of the technology described in this
document or the extent to which any license under such rights might or
might not be available; neither does it represent that it has made any
effort to identify any such rights. Copies of claims of rights made
available for publication and any assurances of licenses to be made
available, or the result of an attempt made to obtain a general
license or permission for the use of such proprietary rights by
implementers or users of this specification can be obtained from the
OGF Secretariat.

The OGF invites any interested party to bring to its attention any
copyrights, patents or patent applications, or other proprietary
rights which may cover technology that may be required to practice
this recommendation. Please address the information to the OGF
Executive Director.


\section{Disclaimer}
\input{include/disclaimer}

\section{Full Copyright Notice}
Copyright \copyright ~Open Grid Forum (2009-2016). All Rights Reserved.

This document and translations of it may be copied and furnished to
others, and derivative works that comment on or otherwise explain it
or assist in its implementation may be prepared, copied, published and
distributed, in whole or in part, without restriction of any kind,
provided that the above copyright notice and this paragraph are
included on all such copies and derivative works. However, this
document itself may not be modified in any way, such as by removing
the copyright notice or references to the OGF or other organizations,
except as needed for the purpose of developing Grid Recommendations in
which case the procedures for copyrights defined in the OGF Document
process must be followed, or as required to translate it into
languages other than English.

The limited permissions granted above are perpetual and will not be
revoked by the OGF or its successors or assignees.


\bibliographystyle{IEEEtran}
\bibliography{references}

\end{document}
